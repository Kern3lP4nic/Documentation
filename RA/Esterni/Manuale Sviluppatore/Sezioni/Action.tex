\documentclass[../ManualeSviluppatore_v2.0.0.tex]{subfiles}

\begin{document}

\section{Aggiunta di Action e QuestionAction a Skill}
	
	Per aggiungere un nuovo file di tipo Action come prima cosa recarsi al seguente percorso:\\
	\textbf{Back-End/LambdaSkill/QuestionActions/QuestionActionModules} (per le QuestionAction)\\
	\textbf{Back-End/LambdaSkill/Actions/ActionModules} (per le Action)\\
	Per sintetizzare, da ora in avanti si farà riferimento solo alle Action, ma il procedimento per le QuestionAction differisce solo per quanto riguarda il percorso dei file sopraindicato.\\
	Il file contenente l'Action appena creata deve avere estensione \underline{.js} e rispettare il seguente template:
	\begin{lstlisting}[language=json,firstnumber=1]
{
  module.exports = {
  id: 5,
  executeAction: function (sessionAttributes, callback) {}}
}
	\end{lstlisting}
	Per evitare inconsistenze fra i file, bisogna assicurarsi, prima di assegnare un numero al campo \underline{id} del template precedente, che quello stesso numero non sia stato già assegnato ad un'Action od una QuestionAction create in precedenza (deve essere univoco per entrambe le cartelle).\\
	A questo punto è necessario caricare l'Action sul Database.\\
	Per terminare, una Action deve richiamare una Callback di tipo:
	\begin{lstlisting}[language=json,firstnumber=1]
{
  callback(err,res)
}
	\end{lstlisting}
	che esegue la funzione che era stata passata precedentemente.
	Il risultato che un'Action restituisce deve essere strutturato nel seguente modo:
	\begin{lstlisting}[language=json,firstnumber=1]
{
  res:{ok:boolean,info:""}
}
	\end{lstlisting}
	
\end{document}