\documentclass[../ManualeSviluppatore_v2.0.0.tex]{subfiles}

\begin{document}

\section{Introduzione}

	\subsection{Scopo del documento}
	Il corrente documento rappresenta il manuale sviluppatore per il software \atavi sviluppato da \kpanic. Il documento ha lo scopo di illustrare allo sviluppatore le modalità di utilizzo e le funzionalità messe a dipsosizione dal prodotto. In particolare verranno descritte le modalità di installazione ed utilizzo, i requisiti necessari per il corretto utilizzo di \atavi, le tecnologie utilizzate e le librerie a loro correlate ed infine verrà fornita una visione generale dell'architettura, allo scopo di aiutare lo sviluppatore nella ricerca delle implementazioni delle funzionalità desiderate.
	
	\subsection{Scopo del prodotto}
	Lo scopo del prodotto è quello di fornire assistenza ad un \gls{ospite} in visita alla sede dell'azienda attraverso un sistema d'interazione che utilizza \gls{tecnologie di sintetizzazione vocale}.
	\\Il prodotto finale dovrà offrire le seguenti funzionalità:
	\begin{itemize}
		\item Registrazione dei dati personali dell'ospite all'interno di un database di supporto;
		\item Identificazione \gls{interlocutore} all'interno dell'azienda;
		\item Inoltro delle informazioni tramite un messaggio \gls{Slack} all'interlocutore;
		\item Accoglienza virtuale dell'ospite durante l'attesa.
	\end{itemize}

	\subsection{Glossario}
	Con lo scopo di rendere più chiara e semplice la lettura e la comprensione di questo documento, è inserito un \textbf{glossario in appendice}, nel quale vengono raccolti termini, anche tecnici, abbreviazioni ed acronimi e solo alla prima istanza.
	
	\subsection{Riferimenti utili}
		\subsubsection{Riferimenti normativi}
		\begin{itemize}
			\item \textit{\normediprogettov};
			\item Regolamento del progetto didattico:\\
			(\url{http://www.math.unipd.it/~tullio/IS-1/2016/Dispense/L09.pdf}(2017/01/26));
		\end{itemize}
	
		\subsubsection{Riferimenti Informativi}
		\begin{itemize}
			\item Capitolato d'appalto C2: \textbf{AtAVi} (Accoglienza tramite Assistente Virtuale):\\ \url{http://www.math.unipd.it/~tullio/IS-1/2016/Progetto/C2.pdf};
			\item Vincoli di organigramma e dettagli tecnico-economici:\\
			(\url{http://www.math.unipd.it/~tullio/IS-1/2016/Progetto/PD01b.html}(2017/01/26));
			\item Software Engineering - Ian Sommerville - 9th Edition 2010;
			\begin{itemize}
				\item Part 1, Chapter 3: Agile Software Developement;
				\item Part 4: Software Management.
			\end{itemize}
			\item Slides del corso di Ingegneria del Software:\\
			(\url{http://www.math.unipd.it/~tullio/IS-1/2016/}(2017/01/26));
			\item Wikipedia, The Free Encyclopedia\\
			(\url{https://it.wikipedia.org/wiki/Scrum_(informatica)}(2017/01/26)).
		\end{itemize}

\end{document}
