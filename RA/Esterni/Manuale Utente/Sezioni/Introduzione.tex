\documentclass[../ManualeUtente_v2.0.0.tex]{subfiles}

\begin{document}

\section{Introduzione}

	\subsection{Scopo del documento}
	Il documento rappresenta il Manuale Utente per il software \gls{AtAVi} sviluppato dal gruppo \kpanic. Esso ha lo scopo di illustrare all'utente le modalità di utilizzo e i servizi offerti dall'applicazione \progetto.
	
	\subsection{Scopo del prodotto}
	Lo scopo del prodotto è quello di fornire assistenza ad un ospite in visita alla sede dell'azienda attraverso un sistema d'interazione che utilizza tecnologie di sintetizzazione vocale.
	\\Il prodotto finale dovrà offrire le seguenti funzionalità:
	\begin{itemize}
		\item Registrazione dei dati personali dell'ospite all'interno di un \gls{database} di supporto;
		\item Identificazione interlocutore all'interno dell'azienda;
		\item Inoltro delle informazioni tramite un messaggio \gls{Slack} all'interlocutore;
		\item Accoglienza virtuale dell'ospite durante l'attesa.
	\end{itemize}

	\subsection{Glossario}
	Con lo scopo di rendere più chiara e semplice la lettura e la comprensione di questo documento, è allegato un \textbf{glossario in appendice} contenente termini tecnici, abbreviazioni o acronimi che hanno bisogno di una spiegazione e solo alla sua prima istanza.

\newpage
	\subsection{Riferimenti utili}
		\subsubsection{Riferimenti normativi}
		\begin{itemize}
			\item \textit{\normediprogettov};
			\item Regolamento del progetto didattico:\\
			(\url{http://www.math.unipd.it/~tullio/IS-1/2016/Dispense/L09.pdf}(2017/01/26));
		\end{itemize}
	
		\subsubsection{Riferimenti Informativi}
		\begin{itemize}
			\item Capitolato d'appalto C2: \textbf{AtAVi} (Accoglienza tramite Assistente Virtuale):\\ \url{http://www.math.unipd.it/~tullio/IS-1/2016/Progetto/C2.pdf};
			\item Vincoli di organigramma e dettagli tecnico-economici:\\
			(\url{http://www.math.unipd.it/~tullio/IS-1/2016/Progetto/PD01b.html}(2017/01/26));
			\item Software Engineering - Ian Sommerville - 9th Edition 2010;
			\begin{itemize}
				\item Part 1, Chapter 3: Agile Software Developement;
				\item Part 4: Software Management.
			\end{itemize}
			\item Slides del corso di Ingegneria del Software:\\
			(\url{http://www.math.unipd.it/~tullio/IS-1/2016/}(2017/01/26));
			\item Wikipedia, The Free Encyclopedia\\
			(\url{https://it.wikipedia.org/wiki/Scrum_(informatica)}(2017/01/26)).
		\end{itemize}

\end{document}
