\documentclass[../ManualeUtente_v2.0.0.tex]{subfiles}

\begin{document}

\section{Requisiti di Sistema}

	\subsection{Dispositivi Supportati}
		\atavi\ risulta essere completamente compatibile con tutti i dispositivi dotati di hardware apposito per l'input vocale e con la maggior parte dei browser più recenti.
	
	\subsection{Browser Supportati}
		Date le limitazioni riscontrate con la compatibilità di alcune librerie \gls{Javascript}, essenziali per la realizzazione del prodotto, con i maggiori web Browser, il team di \kpanic\ ha deciso che \atavi\ sarà completamente compatibile solamente con i browser elencati nella seguente tabella:
		
	\begin{longtable}[c] { >{\centering\arraybackslash}p{3cm} >{\centering\arraybackslash}p{3cm}}
	\toprule
	\centerline{\textbf{Broswer}} & \centerline{\textbf{Versione}} \\
	\midrule
	Edge & 38 e successive  \\
	\addlinespace[0.4em]
	\midrule
	\addlinespace[0.4em]
	Google Chrome & 53 e successive \\
	\addlinespace[0.4em]
	\midrule
	\addlinespace[0.4em]
	Mozilla Firefox (desktop) & 28 e successive \\
	\addlinespace[0.4em]
	\midrule
	\addlinespace[0.4em]
	Mozilla Firefox (mobile) & 52 e successive \\
	\addlinespace[0.4em]
	\midrule
	\addlinespace[0.4em]
	Opera & 44 e successive \\
	\bottomrule
	\caption{Versioni browser supportati - aggiornata}
	\label{tab:browser}
	\end{longtable}
	
	\subsubsection{Estensione}
	Per permette il corretto funzionamento dell'applicativo è necessario installare un'estensione nei browser supportati che permetta di aggiungere all'header delle risposte, delle API, la stringa "Allow-Control-Allow-Origin".
	
\end{document}
