\documentclass[../PianoDiProgetto_v4.0.0.tex]{subfiles}

\begin{document}
	\section{Consuntivo di periodo}
	Di seguito verranno indicate le spese effettivamente sostenute, sia per ruolo che per persona.\\
	Il bilancio potrà risultare:
	\begin{itemize}
		\item \textbf{Positivo}: se il preventivo supera il consuntivo;
		\item \textbf{In pari}: se il preventivo e il consuntivo sono equivalenti;
		\item \textbf{Negativo}: se il consuntivo supera il preventivo.
	\end{itemize}
	
	\subsection{Primo Periodo}
		\subsubsection{Consuntivo}
		Essendo \kpanic\ non ancora stato scelto come fornitore ufficiale per il progetto \progetto, i suoi componenti hanno svolto le ore di lavoro come approfondimento personale.\\
		I dati riportati sulla seguente tabella riguardano le ore non rendicontate.
		\begin{table}[h]
		\centering
		\begin{tabular}{l * {2}{c}}
			\toprule
			\textbf{Ruolo} & \textbf{Ore} & \textbf{Costo (\euro{})} \\
			\midrule
			\responsabilediprogetto &	32 (+3) & 1050,00 \\

			\amministratore & 64 (+2) & 1320,00 \\
			
			\progettista & 0 & 0,00 \\
			
			\analista & 68 (+5) & 1825,00 \\
			
			\programmatore & 0 & 0,00 \\
			
			\verificatore & 46 (+4) & 7500,00 \\
			
			\midrule
			\textbf{Totale Preventivo} & 210 & 4.630,00 \\		
			\textbf{Totale Consuntivo} & 223 & 4945,00 \\
			\midrule
			\textbf{Differenza} & +14 & +315,00 \\
			\bottomrule
		\end{tabular}
		\caption{Primo Periodo - Consuntivo}
		\label{tab:consuntivo1}
	\end{table}
	
		\subsubsection{Conclusioni}
		Come si può vedere dalla tabella \ref{tab:consuntivo1}, che contiene i dati riguardanti il consuntivo del Primo Periodo, è stato impiegato più tempo rispetto a quanto previsto in tutti i ruoli, di conseguenza il bilancio risulta essere \textbf{negativo}.
		
			Vista la mancanza di esperienza nella pianificazione e nella conoscenza di progetti sui quali basare la preventivazione dei costi, da parte dei componenti del gruppo che hanno svolto il ruolo di \responsabilediprogetto, sono state necessarie più ore di lavoro.
			
			Il lavoro svolto dagli \amministratori\ è rimasto in linea con quanto era stato previsto, visto che chi ha svolto questo ruolo aveva già conoscenza della piattaforma su cui sono stati basati i software di supporto utilizzati, come quello per il tracciamento dei requisiti.
			
			L'attività degli \analisti\ ha impiegato più ore di quanto preventivato poiché il capitolato scelto ha richiesto una buona dose di innovazione e di ricerca che, in questo periodo, ha impattato sulla specifica dei casi d'uso e dei requisiti.
			
			Per quanto riguarda i \verificatori, le ore aggiuntive sono state necessarie a causa delle numerose modifiche fatte sui documenti che andavano verificati.
			\\ \\
			Viste le ore di lavoro non preventivate utilizzate per effettuare l'analisi dei requisiti, sicuramente per il Secondo Periodo, ovvero l'Analisi di dettaglio, verrà stanziato un numero maggiore di ore a disposizione degli \analisti. Grazie all'esperienza acquisita prima di questa revisione, i membri che svolgeranno il ruolo di \responsabilediprogetto\ dispongono di una maturità tale da non necessitare ulteriori ore di lavoro, rispetto a quelle preventivate, per svolgere le proprie attività. Grazie alle correzioni apportate alla revisione precedente ed alla minor mole di documenti da verificare, non saranno necessarie ore aggiuntive per svolgere le attività di \verificatori.

	\subsection{Secondo Periodo}
		\subsubsection{Consuntivo}
		Le ore rendicontate per il Secondo Periodo, di Analisi di Dettaglio, sono riportate nella tabella seguente:
		\begin{table}[h]
				\centering
				\begin{tabular}{l * {2}{c}}
					\toprule
					\textbf{Ruolo} & \textbf{Ore} & \textbf{Costo (\euro{})} \\
					\midrule
					Responsabile & 9 (-1) & 240,00 \\
					
					Amministratore & 9 & 180,00 \\
				
					Progettista & 0 & 0,00 \\

					Analista & 18 & 450,00 \\		

					Programmatore & 0 & 0,00 \\		

					Verificatore & 30 (+2) & 480,00 \\
							
					\midrule		
					\textbf{Totale Preventivo} & 66 & 1.350,00 \\
					\textbf{Totale Consuntivo} & 67 & 1.350,00 \\
					\midrule
					\textbf{Differenza} & +1 & 0,00 \\
					\bottomrule
				\end{tabular}
				\caption{Secondo Periodo - Consuntivo}
				\label{tab:consuntivo2}	
			\end{table}

		\subsubsection{Conclusioni}
		Come si può notare dalla tabella \ref{tab:consuntivo2} le ore previste sono state di poco superate, ma per quanto riguarda il bilancio non ci sono stati cambiamenti rispetto a quanto preventivato ed esso risulta quindi essere \textbf{in pari}.
		
			Per il ruolo di \responsabilediprogetto\ é stato sufficiente un numero inferiore di ore rispetto al preventivo per via di un minor numero di documenti da validare.
		
			Viste le numerose correzioni apportate ad alcuni documenti é stato necessario verificarli più di quanto previsto, quindi il ruolo di \verificatore\ ha richiesto un numero di ore leggermente superiore.


	\subsection{Terzo Periodo}
		\subsubsection{Consuntivo}
		Per il Terzo periodo, Progettazione Architetturale e Progettazione di Dettaglio, sono state necessarie le ore di lavoro riportate nella seguente tabella:
		
		\begin{table}[h]
				\centering
				\begin{tabular}{l * {2}{c}}
					\toprule
					\textbf{Ruolo} & \textbf{Ore} & \textbf{Costo (\euro{})} \\
					\midrule
					Responsabile & 17 & 510,00 \\

					Amministratore & 16 & 320,00 \\

					Progettista & 70 (+7) & 1.694,00 \\

					Analista & 36 (-6) & 750,00 \\		

					Programmatore & 0 & 0,00 \\		

					Verificatore & 31 (-2) & 435,00 \\				
					\midrule		
					\textbf{Totale Preventivo} & 171 & 3.765,00 \\
					\textbf{Totale Consuntivo} & 170 & 3.709,00 \\
					\midrule
					\textbf{Differenza} & -1 & -56,00 \\
					\bottomrule
				\end{tabular}
				\caption{Terzo Periodo - Consuntivo}
				\label{tab:consuntivo3}	
			\end{table}
		
		\subsubsection{Conclusioni}
		Come evidenzia la tabella \ref{tab:consuntivo3} in questo periodo le ore effettive di lavoro necessarie sono inferiori a quanto previsto e anche il bilancio ne ha beneficiato risultando \textbf{positivo}.
		
			Vista la grande necessità, in questo periodo, del ruolo di \progettista\ e visto il grande numero di correzioni apportate all'architettura, sono stato necessarie alcune ore aggiuntive rispetto a quanto preventivato. 
			
			Per il ruolo di \analista\ le ore totali necessarie sono state diminuite a causa di una riduzione del monte ore in favore del ruolo di \progettista.
			
			Per i \verificatori\ le ore sono risultate inferiori al preventivo in quanto l'attività di verifica ha richiesto un tempo leggermente inferiore. 

	\subsection{Quarto Periodo}
		\subsubsection{Consuntivo}
		Per il Quarto periodo, Revisione di Progettazione e Codifica dei Requisiti Obbligatori, sono state necessarie le ore di lavoro riportate nella seguente tabella:
		
		\begin{table}[h]
				\centering
				\begin{tabular}{l * {2}{c}}
					\toprule
					\textbf{Ruolo} & \textbf{Ore} & \textbf{Costo (\euro{})} \\
					\midrule
					Responsabile & 13 (-2) & 330,00 \\

					Amministratore & 10 (+3) & 260,00 \\

					Progettista & 36 (+7) & 946,00 \\

					Analista & 5 & 125,00 \\		

					Programmatore & 96 (-3) & 1.395,00 \\		

					Verificatore & 51 (-2) & 735,00 \\				
					\midrule		
					\textbf{Totale Preventivo} & 211 & 3.712,00 \\
					\textbf{Totale Consuntivo} & 211  & 3.791,00 \\
					\midrule
					\textbf{Differenza} & 0  & +79,00 \\
					\bottomrule
				\end{tabular}
				\caption{Quarto Periodo - Consuntivo}
				\label{tab:consuntivo4}	
			\end{table}
		
		\subsubsection{Conclusioni}
		Come evidenzia la tabella \ref{tab:consuntivo4} in questo periodo le ore effettive di lavoro necessarie sono uguali a quanto previsto, il bilancio però non ne ha beneficiato risultando \textbf{negativo} a causa della riorganizzazione delle ore dedicate ad alcuni ruoli.
		
			Il ruolo di \progettista\ ha necessitato di alcune ore aggiuntive a causa di numerose modifiche da fare all'architettura scoperte durante l'attività di codifica. 
			
			Per i ruolo di \responsabilediprogetto\ e di \programmatore\ le ore totali necessarie sono state diminuite a causa di una riduzione del monte ore in favore del ruolo di \progettista.
			
			Per gli \amministratori\ le ore previste non sono risultate sufficienti a causa di numerose modifiche apportate alle \normediprogettoRQ\ per correggere le segnalazione fatte alla consegna della Revisione di Progettazione.
			
			Per i \verificatori\ le ore sono risultate inferiori al preventivo in quanto l'attività di verifica ha richiesto un tempo leggermente inferiore.
			\\ \\
			Per sopperire a questo aumento di costi nel successivo periodo le ore verrano riorganizzate per mantenere il preventivo in linea quanto preventivato. Verrano quindi aggiornati i preventivi dei periodi successivi riducendo le ore al ruolo di \progettista\ e di \analista\ essendo a buon punto con la propria attività.\\
			
		\subsubsection{Riorganizzazione preventivo}
		Dopo il consuntivo del Quarto Periodo vengono aggiornate le ore assegnate per i successivi periodi in modo da ammortizzare l'aumento dei costi. I nuovi preventivi sono visibili nelle seguenti tabelle:
		
		\begin{table}[h]
			%\centering
			\begin{tabularx}{\textwidth}{l * {6}{C} c}
			\toprule
			\textbf{Nominativo} & \textbf{Rp} & \textbf{Am} & \textbf{Pt} & \textbf{An} & \textbf{Pm} & \textbf{Ve} & \textbf{Ore totali} \\
			\midrule
			Berto Filippo &	0 & 0 & 0 & 4 & 5 & 7 & 17 \\
			%\midrule
			Fasolato Francesco & 0 & 0 & 4 & 3 & 0 & 7 & 15 \\
			%\midrule
			Favaro Daniele & 0 & 4 & 0 & 5 & 0 & 6 & 16 \\
			%\midrule
			Franceschini Marco & 0 & 0 & 0 & 0 & 10 & 7 & 17 \\
			%\midrule
			Macrì Antonino & 0 & 0 & 0 & 0 & 9 & 5 & 15 \\
			%\midrule
			Zanon Edoardo &	10 & 0 & 0 & 0 & 5 & 0 & 15 \\
			%\midrule
			Zecchin Giacomo & 0 & 0 & 3 & 0 & 3 & 7 & 14 \\
			\midrule		
			\textbf{Ore Totali Ruolo} & 10 & 5 & 7 & 11 & 32 & 39 & 109 \\
			\bottomrule
			\end{tabularx}
			\caption{Quinto Periodo - Suddivisione delle ore di lavoro aggiornata}		
		\end{table}
		
		\paragraph{Conslusione} \mbox{} \\
		Rispetto al preventivo presente nella sezione \textit{Preventivo} quello presente nella tabella soprastante sono state tolte due ore al ruolo di \progettista\, visto che non saranno necessarie drastiche modifiche all'architettura come per il Quarto Periodo. Altre due ore sono state tolte al ruolo di \analista\ visto che la maggior parte del lavoro sui documenti è stato fatto.
		
		\begin{table}[h]
			\centering
			\begin{tabular}{l * {2}{c}}
			\toprule
			\textbf{Ruolo} & \textbf{Ore} & \textbf{Costo (\euro{})} \\
			\midrule
			Responsabile & 10 & 300,00 \\
			%\midrule
			Amministratore & 5 & 100,00 \\
			%\midrule
			Progettista & 7 & 154,00 \\
			%\midrule
			Analista & 11 & 2.75,00 \\		
			%\midrule
			Programmatore & 33 & 495,00 \\		
			%\midrule
			Verificatore & 39 & 585,00 \\				
			\midrule		
			\textbf{Totale} & 105 & 1.909,00 \\
			\bottomrule	
			\end{tabular}
			\caption{Quinto Periodo - Prospetto economico aggiornato}	
		\end{table}
		
		Come è possibile notare, confrontanto quest'ultima tabella con l'equivalente presente nella sezione \textit{Preventivo}, sono state ridotte le ore così come il costo per il Quito Periodo che è calato di 94,00 \euro{}.
		
	\subsection{Quinto Periodo}
		\subsubsection{Consuntivo}
		Per il Quinto Periodo, Codifica dei Requisiti Desiderabili, sono state necessarie le ore di lavoro riportate nella seguente tabella:
		
		\begin{table}[h]
				\centering
				\begin{tabular}{l * {2}{c}}
					\toprule
					\textbf{Ruolo} & \textbf{Ore} & \textbf{Costo (\euro{})} \\
					\midrule
					Responsabile & 10 & 330,00 \\

					Amministratore & 5 & 260,00 \\

					Progettista & 9 & 946,00 \\

					Analista & 13 & 325,00 \\		

					Programmatore & 32 & 480,00 \\		

					Verificatore & 39 & 585,00 \\				
					\midrule		
					\textbf{Totale Preventivo} & 105 & 1.988,00 \\
					\textbf{Totale Consuntivo} & 105  & 1.988,00 \\
					\midrule
					\textbf{Differenza} & 0  & 00,00 \\
					\bottomrule
				\end{tabular}
				\caption{Quarto Periodo - Consuntivo}
				\label{tab:consuntivo5}	
			\end{table}
		
		\subsubsection{Conclusioni}
		Come evidenzia la tabella \ref{tab:consuntivo5} in questo periodo le ore effettive di lavoro necessarie sono risultate uguali rispetto a quanto previsto, il  bilancio di conseguenza è risultato \textbf{in pari}.
		
		\newpage
		\subsection{Sesto Periodo}
		\subsubsection{Consuntivo}
		Per il Sesto Periodo, Codifica dei Requisiti Opzionali, sono state necessarie le ore di lavoro riportate nella seguente tabella:
		
		\begin{table}[h]
				\centering
				\begin{tabular}{l * {2}{c}}
					\toprule
					\textbf{Ruolo} & \textbf{Ore} & \textbf{Costo (\euro{})} \\
					\midrule
					Responsabile & 5 (-1) & 120,00 \\

					Amministratore & 6 (-1) & 100,00 \\

					Progettista & 22 (-4) & 396,00 \\

					Analista & 5 & 125,00 \\		

					Programmatore & 13 (+2) & 225,00 \\		

					Verificatore & 23 & 345,00 \\				
					\midrule		
					\textbf{Totale Preventivo} & 74 & 1.419,00 \\
					\textbf{Totale Consuntivo} & 72  & 1.311,00 \\
					\midrule
					\textbf{Differenza} & -4  & -108,00 \\
					\bottomrule
				\end{tabular}
				\caption{Sesto Periodo - Consuntivo}
				\label{tab:consuntivo6}	
			\end{table}
		
		\subsubsection{Conclusioni}
		Come evidenzia la tabella \ref{tab:consuntivo6} in questo periodo le ore effettive di lavoro necessarie sono state leggermente inferiori rispetto a quanto previsto, di conseguenza il  bilancio è diminuito, risultando essere \textbf{positivo}.\\
		I ruoli di \responsabilediprogetto, di \amministratore\ e di \progettista\ hanno richiesto meno ore di quelle previste, data il minor impiego, in questo periodo, delle loro attività, anche grazie ad una quantità di errori segnalati in sede di Revisione di Qualifica non eccessiva.\\
		L'attività del \programmatore\ ha, invece, richiesto più ore rispetto a quanto preventivato a causa della ricerca di una soluzione al problema riscontrato con i server AVS, inoltre l'attività di codifica del front-end che riguarda la gestione delle domande ha richiesto più tempo del previsto per essere codificata.\\
		\newpage
		\subsubsection{Riorganizzazione preventivo}
		Dopo il consuntivo del Sesto Periodo vengono aggiornate le ore assegnate per i successivi periodi in modo riorganizzare il tempo per ruolo. I nuovi preventivi sono visibili nella seguente tabella:
		
		\begin{table}[h]
			%\centering
			\begin{tabularx}{\textwidth}{l * {6}{C} c}
			\toprule
			\textbf{Nominativo} & \textbf{Rp} & \textbf{Am} & \textbf{Pt} & \textbf{An} & \textbf{Pm} & \textbf{Ve} & \textbf{Ore totali} \\
			\midrule
			Berto Filippo &	0 & 4 & 0 & 0 & 11 & 0 & 15 \\
			%\midrule
			Fasolato Francesco & 0 & 0 & 0 & 0 & 9 & 5 & 14 \\
			%\midrule
			Favaro Daniele & 0 & 0 & 0 & 0 & 0 & 13 & 13 \\
			%\midrule
			Franceschini Marco & 0 & 0 & 6 & 0 & 0 & 8 & 14 \\
			%\midrule
			Macrì Antonino & 0 & 0 & 6 & 0 & 0 & 9 & 15 \\
			%\midrule
			Zanon Edoardo &	3 & 0 & 0 & 0 & 4 & 8 & 15 \\
			%\midrule
			Zecchin Giacomo & 5 & 0 & 3 & 0 & 0 & 8 & 13 \\
			\midrule		
			\textbf{Ore Totali Ruolo} & 8 & 4 & 12 & 0 & 24 & 51 & 98 \\
			\bottomrule
			\end{tabularx}
			\caption{Settimo Periodo - Suddivisione delle ore di lavoro aggiornata}		
		\end{table}
		
		\paragraph{Conslusione} \mbox{} \\
		Rispetto al preventivo presente nella sezione \textit{Preventivo}, in quello presente nella tabella soprastante sono state aggiunte due ore al ruolo di \programmatore, dato che si svolgeranno test di collaudo e molto probabilmente bisognerà corregge gli eventuali errori emersi, quindi è necessario assegnare più tempo per questo ruolo. Sono state aumentate anche le ore dei \verificatori, di tre unità, visto che il lavoro di verifica richiederà molto tempo per essere svolto nel successivo periodo. È stato possibile aumentare le ore grazie ad un bilancio positivo presentatosi nel Sesto Periodo.
		
		\begin{table}[h]
			\centering
			\begin{tabular}{l * {2}{c}}
			\toprule
			\textbf{Ruolo} & \textbf{Ore} & \textbf{Costo (\euro{})} \\
			\midrule
			Responsabile & 8 & 240,00 \\
			%\midrule
			Amministratore & 4 & 80,00 \\
			%\midrule
			Progettista & 12 & 264,00 \\
			%\midrule
			Analista & 0 & 0,00 \\		
			%\midrule
			Programmatore & 25 & 375,00 \\		
			%\midrule
			Verificatore & 52 & 780,00 \\				
			\midrule		
			\textbf{Totale} & 101 & 1.739,00 \\
			\bottomrule	
			\end{tabular}
			\caption{Settimo Periodo - Prospetto economico aggiornato}	
		\end{table}
		
		Come è possibile notare, confrontanto quest'ultima tabella con l'equivalente presente nella sezione \textit{Preventivo}, sono state aumentate le ore così come il costo per il Settimo Periodo che è aumentato di 45,00 \euro{}, a fronte di un risparmio di 58,00 \euro{} nel Sesto Periodo.
		
		\subsection{Settimo Periodo}
		\subsubsection{Consuntivo}
		Per il Settimo Periodo, il periodo di Validazione, sono state necessarie le ore di lavoro riportate nella seguente tabella:
		
		\begin{table}[h]
				\centering
				\begin{tabular}{l * {2}{c}}
					\toprule
					\textbf{Ruolo} & \textbf{Ore} & \textbf{Costo (\euro{})} \\
					\midrule
					Responsabile & 8 & 240,00 \\

					Amministratore & 4 & 80,00 \\

					Progettista & 12 (-2) & 220,00 \\

					Analista & 0 & 0,00 \\		

					Programmatore & 23 (+8) & 465,00 \\		

					Verificatore & 49 (+4) & 795,00 \\				
					\midrule		
					\textbf{Totale Preventivo} & 96 & 1.664,00 \\
					\textbf{Totale Consuntivo} & 107  & 1.800,00 \\
					\midrule
					\textbf{Differenza} & +10  & +136,00 \\
					\bottomrule
				\end{tabular}
				\caption{Settimo Periodo - Consuntivo}
				\label{tab:consuntivo7}	
			\end{table}
		
		\subsubsection{Conclusioni}
		Come evidenzia la tabella \ref{tab:consuntivo7} in questo periodo le ore effettive di lavoro necessarie sono risultate essere incrementate rispetto a quanto previsto, il  bilancio di conseguenza è aumentato risultando essere \textbf{negativo}.\\
		Il tempo dedicato ai \progettisti\ è stato mal calcolato e l'attività è stata svolta in un numero di ore inferiore rispetto a quanto preventivato.\\
		L'attività dei \verificatori\ è aumentata a causa di un numero non indifferente di test da eseguire, di funzionalità da verificare e di errori riscontrati.\\
		A causa di ciò anche l'attività dei \programmatori\ ha subito un incremento di ore, durante le quali sono stati risolti alcuni errori emersi.\\
		Visto il risparmio ottenuto nel periodo precedente e tenendo anche conto della riorganizzazione delle ore di lavoro, le perdite vengono ammoetizzate, ma il bilancio risulta comunque essere \textbf{negativo} rispetto a quanto preventivato.\\
		
		\newpage
		\subsection{Riepilogo}
		Viene di seguito fatto una considerazione finale prendendo in esame tutti i consuntivi di periodo.
		
		\subsubsection{Ore persona}
		Ogni componente del gruppo \kpanic\ ha speso le seguenti ore rendicontate per ruolo:
		\begin{table}[h]
				%\centering
				\begin{tabularx}{\textwidth}{l * {6}{C} c}
				\toprule
				\textbf{Nominativo} & \textbf{Rp} & \textbf{Am} & \textbf{Pt} & \textbf{An} & \textbf{Pm} & \textbf{Ve} & \textbf{Ore totali} \\
				\midrule
				Berto Filippo &	13 & 5 & 24 & 14 & 37 & 12 & 105 \\
				%\midrule
				Fasolato Francesco & 0 & 13 & 26 & 12 & 16 & 37 & 104 \\
				%\midrule
				Favaro Daniele & 0 & 9 & 22 & 5 & 16 & 53 & 105 \\
				%\midrule
				Franceschini Marco & 15 & 0 & 21 & 3 & 31 & 35 & 105 \\
				%\midrule
				Macrì Antonino & 0 & 16 & 18 & 17 & 17 & 37 & 105 \\
				%\midrule
				Zanon Edoardo &	16 & 0 & 27 & 10 & 32 & 20 & 105 \\
				%\midrule
				Zecchin Giacomo & 14 & 9 & 19 & 10 & 22 & 31 & 105 \\
				\midrule			
				\textbf{Ore Totali Ruolo} & 58 & 52 & 157 & 71 & 171 & 225 & 734 \\
				\bottomrule
				\end{tabularx}
				\caption{Riepilogo Consuntivo - Ore di lavoro per ruolo}		
		\end{table}
		
		\subsubsection{Prospetto economico}
		Riassumento tutti i consuntivi di periodo, per le ore spese per ogni ruolo si ottengono i valori riportati nella seguente tabella:
		
			\begin{table}[h]
				\centering
				\begin{tabular}{l * {2}{c}}
				\toprule
				\textbf{Ruolo} & \textbf{Ore} & \textbf{Costo (\euro{})} \\
				\midrule
				Responsabile & 62 (-4) & 1.740,00 \\
				%\midrule
				Amministratore & 50 (+2) & 1.040,00 \\
				%\midrule
				Progettista & 149 (+8) & 3.454,00 \\
				%\midrule
				Analista & 77 (-6) & 1.775,00 \\		
				%\midrule
				Programmatore & 164 (+7) & 2.565,00 \\		
				%\midrule
				Verificatore & 223 (+2)  & 3.375,00 \\				
				\midrule		
				\textbf{Totale Preventivo} & 725 & 13.868,00 \\
				\textbf{Totale Consuntivo} & 734  & 13.949,00 \\
				\midrule
				\textbf{Differenza} & +9  & +81,00 \\
				\bottomrule
				\end{tabular}
				\caption{Prospetto economico totale}		
			\end{table}
			
		\subsubsection{Conclusioni}
		Si può notare che per la realizzazione del progetto \atavi\ sono state necessarie 9 ore in più rispetto a quanto preventivato, con un conseguente aumento del bilancio, pari a 81,00 \euro{}, che quindi è risultato \textbf{negativo}.
		Le motivazioni di questo risultato ricadono principalmente sul problema tecnologico riscontrato con AVS, che ha richiesto un maggiore sforzo da parte dei \programmatori\ Daniele Favaro e 
		Francesco Fasolato nel cercare di risolvere la problematica, interagendo anche con il supporto Amazon.
		A casua dell'incapacità del team di risolvere la problematica è stato richiesto anche un supplemento da parte di altre figure come l'amministratore per negoziare una soluzione comune 
		con l'azienda (citata nel verbale esterno del giorno 2017-05-10 discussione \#4).
		Il bilancio negativo è quindi stato causato un'eccessiva sicurezza da parte del team nel pensare di risolvere velocemente il problema con AVS.
		Il numero maggiore di ore assegnate ai verificatori e progettisti, per le problematiche sopra descritte, non sono state quindi riassorbite da altri ruoli.
		L'approccio corretto sarebbe stato essere più prudenti nell'affrontare la problematica, gestendo dei tempi "cuscinetto" nella pianificazione post consuntivo del Quinto Periodo.
		
\end{document}