\documentclass[../PianoDiProgetto_v4.0.0.tex]{subfiles}

\section{Meccanismi di Controllo e Rendicontazione}

\begin{document}
Per controllare e valutare lo stato di avanzamento del lavoro e delle attività previste dal progetto e facilitare lo svolgimento del ruolo di \responsabilediprogetto, si è scelto di utilizzare i seguenti strumenti:
	\begin{itemize}
	\item \textbf{Diagrammi, tabelle e grafici}: per rendere più efficace la pianificazione e la distribuzione del lavoro sono stati realizzati diagrammi di \gl{Gantt}, tabelle e diagrammi riassuntivi;
	\item \textbf{\gl{Sistema di Tasking}}: per avere sempre sotto controllo lo stato di avanzamento dei lavori assegnati ai vari componenti del gruppo viene utilizzato il sistema di tasking offerto da Teamwork che permette di visualizzare anche la percentuale e il numero dei lavori assegnati, completati o meno;
	\item \textbf{Rendicontazione delle ore di lavoro}: Teamwork dispone di un meccanismo per la rendicontazione delle ore di lavoro. In questo modo, è possibile controllare l'avanzamento del lavoro. Il suo utilizzo, inoltre, facilita la stesura del Consuntivo. Nel caso fosse impossibile utilizzare questo strumento, verrà scritto a mano il numero di ore persona impiegate nel task eseguito;
	\item \textbf{Riunioni interne}: vengono tenute ad intervalli regolari per valutare lo stato di avanzamento dei lavori e per porre eventuali modifiche in base alle opinioni dei membri del team.
	\end{itemize}
	
\end{document}
