\documentclass[../DefinizioneDiProdotto.tex]{subfiles}

\begin{document}

\section{Introduzione}

	\subsection{Scopo del documento}
	Il presente documento ha lo scopo di definire in dettaglio la struttura e il funzionamento delle componenti del progetto AtAVi. Questo documento servirà come guida per i \programmatori\ del gruppo \kpanic\ durante lo sviluppo e fornendo direttive e vincoli per la realizzazione del \gl{progetto}.
	
	\subsection{Scopo del prodotto}
	Lo scopo del prodotto è quello di fornire assistenza ad un \gl{ospite} in visita alla sede dell'azienda attraverso un sistema d'interazione che utilizza tecnologie di sintetizzazione vocale.
	\\Il prodotto finale dovrà offrire le seguenti funzionalità:
	\begin{itemize}
		\item Registrazione dei dati personali dell'ospite all'interno di un \gl{database} di supporto;
		\item Identificazione \gl{interlocutore} all'interno dell'azienda;
		\item Inoltro delle informazioni tramite un messaggio \gl{Slack} all'interlocutore;
		\item Accoglienza virtuale dell'ospite durante l'attesa.
	\end{itemize}

	\subsection{Glossario}
	Con lo scopo di rendere più chiara e semplice la lettura e la comprensione di questo documento, viene allegato il \glossariov, nel quale vengono raccolti termini, anche tecnici, abbreviazioni ed acronimi. Per evidenziare un termine presente in tale documento, esso verrà marcato con il \gl{pedice}, e solo alla sua prima istanza.
	
	\subsection{Riferimenti utili}
		\subsubsection{Riferimenti normativi}
		\begin{itemize}
			\item \textit{\normediprogettov};
			\item Regolamento del progetto didattico:\\
			(\url{http://www.math.unipd.it/~tullio/IS-1/2016/Dispense/L09.pdf}(2017/01/26));
		\end{itemize}
	
		\subsubsection{Riferimenti Informativi}
		\begin{itemize}
			\item Capitolato d'appalto C2: \textbf{AtAVi} (Accoglienza tramite Assistente Virtuale):\\ \url{http://www.math.unipd.it/~tullio/IS-1/2016/Progetto/C2.pdf};
			\item Vincoli di organigramma e dettagli tecnico-economici:\\
			(\url{http://www.math.unipd.it/~tullio/IS-1/2016/Progetto/PD01b.html}(2017/01/26));
			\item Software Engineering - Ian Sommerville - 9th Edition 2010:
			\begin{itemize}
				\item Part 1, Chapter 3: Agile Software Developement;
				\item Part 4: Software Management.
			\end{itemize}
			\item Slides del corso di Ingegneria del Software:\\
			(\url{http://www.math.unipd.it/~tullio/IS-1/2016/}(2017/01/26));
			\item Wikipedia, The Free Encyclopedia\\
			(\url{https://it.wikipedia.org/wiki/Scrum_(informatica)}(2017/01/26)).
			\item Riferimenti tecnologici:
			\begin{itemize}
				\item (\url{https://aws.amazon.com/it/lambda/faqs/}(2017/03/05));
				\item (\url{https://developer.amazon.com/public/solutions/alexa/alexa-voice-service/getting-started-with-the-alexa-voice-service}(2017/03/05));
				\item (\url{https://developer.amazon.com/alexa-voice-service}(2017/03/05));
				\item (\url{https://aws.amazon.com/it/documentation/sns/}(2017/03/05));
				\item (\url{http://docs.aws.amazon.com/lambda/latest/dg/API_Reference.html}(2017/03/05));
				\item (\url{https://aws.amazon.com/it/documentation/sdk-for-javascript/}(2017/03/05));
				\item (\url{https://angular.io/docs/js/latest/}(2017/03/05));
				\item (\url{https://aws.amazon.com/it/documentation/apigateway/}(2017/03/05));
				\item (\url{https://nodejs.org/it/docs/}(2017/03/05));
				\item (\url{http://swagger.io/docs/}(2017/03/05));
				\item (\url{http://docs.aws.amazon.com/amazondynamodb/latest/APIReference/API_Operations_Amazon_DynamoDB.html}(2017/03/05)).
			
			\end{itemize}
			
			
			
		\end{itemize}

\end{document}
