\documentclass[../PianoDiProgetto.tex]{subfiles}

\begin{document}
	\section{Consuntivo di periodo}
	Di seguito verranno indicate le spese effettivamente sostenute, sia per ruolo che per persona.\\
	Il bilancio potrà risultare:
	\begin{itemize}
		\item \textbf{Positivo}: se il preventivo supera il consuntivo;
		\item \textbf{In pari}: se il preventivo e il consuntivo sono equivalenti;
		\item \textbf{Negativo}: se il consuntivo supera il preventivo.
	\end{itemize}
	
	\subsection{Primo Periodo}
		\subsubsection{Consuntivo}
		Essendo \kpanic\ non ancora stato scelto come fornitore ufficiale per il progetto \progetto, i suoi componenti hanno svolto le ore di lavoro come approfondimento personale.\\
		I dati riportati sulla seguente tabella riguardano le ore non rendicontate.
		\begin{table}[h]
		\centering
		\begin{tabular}{l * {2}{c}}
			\toprule
			\textbf{Ruolo} & \textbf{Ore} & \textbf{Costo (\euro{})} \\
			\midrule
			\responsabilediprogetto &	32 (+3) & 1050,00 \\

			\amministratore & 64 (+2) & 1320,00 \\
			
			\progettista & 0 & 0,00 \\
			
			\analista & 68 (+5) & 1825,00 \\
			
			\programmatore & 0 & 0,00 \\
			
			\verificatore & 46 (+4) & 7500,00 \\
			
			\midrule
			\textbf{Totale Preventivo} & 210 & 4.630,00 \\		
			\textbf{Totale Consuntivo} & 223 & 4945,00 \\
			\midrule
			\textbf{Differenza} & +14 & +315,00 \\
			\bottomrule
		\end{tabular}
		\caption{Primo Periodo - Consuntivo}
		\label{tab:consuntivo1}
	\end{table}
	
		\subsubsection{Conclusioni}
		Come si può vedere dalla tabella \ref{tab:consuntivo1}, che contiene i dati riguardanti il consuntivo del Primo Periodo, è stato impiegato più tempo rispetto a quanto previsto in tutti i ruoli, di conseguenza il bilancio risulta essere \textbf{negativo}.
		
			Vista la mancanza di esperienza nella pianificazione e nella conoscenza di progetti sui quali basare la preventivazione dei costi, da parte dei componenti del gruppo che hanno svolto il ruolo di \responsabilediprogetto, sono state necessarie più ore di lavoro.
			
			Il lavoro svolto dagli \amministratori\ è rimasto in linea con quanto era stato previsto, visto che chi ha svolto questo ruolo aveva già conoscenza della piattaforma su cui sono stati basati i software di supporto utilizzati, come quello per il tracciamento dei requisiti.
			
			L'attività degli \analisti\ ha impiegato più ore di quanto preventivato poiché il capitolato scelto ha richiesto una buona dose di innovazione e di ricerca che, in questo periodo, ha impattato sulla specifica dei casi d'uso e dei requisiti.
			
			Per quanto riguarda i \verificatori, le ore aggiuntive sono state necessarie a causa delle numerose modifiche fatte sui documenti che andavano verificati.
			\\ \\
			Viste le ore di lavoro non preventivate utilizzate per effettuare l'analisi dei requisiti, sicuramente per il Secondo Periodo, ovvero l'Analisi di dettaglio, verrà stanziato un numero maggiore di ore a disposizione degli \analisti. Grazie all'esperienza acquisita prima di questa revisione, i membri che svolgeranno il ruolo di \responsabilediprogetto\ dispongono di una maturità tale da non necessitare ulteriori ore di lavoro, rispetto a quelle preventivate, per svolgere le proprie attività. Grazie alle correzioni apportate alla revisione precedente ed alla minor mole di documenti da verificare, non saranno necessarie ore aggiuntive per svolgere le attività di \verificatori.

	\subsection{Secondo Periodo}
		\subsubsection{Consuntivo}
		Le ore rendicontate per il Secondo Periodo, di Analisi di Dettaglio, sono riportate nella tabella seguente:
		\begin{table}[h]
				\centering
				\begin{tabular}{l * {2}{c}}
					\toprule
					\textbf{Ruolo} & \textbf{Ore} & \textbf{Costo (\euro{})} \\
					\midrule
					Responsabile & 9 (-1) & 240,00 \\
					
					Amministratore & 9 & 180,00 \\
				
					Progettista & 0 & 0,00 \\

					Analista & 18 & 450,00 \\		

					Programmatore & 0 & 0,00 \\		

					Verificatore & 30 (+2) & 480,00 \\
							
					\midrule		
					\textbf{Totale Preventivo} & 66 & 1.350,00 \\
					\textbf{Totale Consuntivo} & 67 & 1.350,00 \\
					\midrule
					\textbf{Differenza} & +1 & 0,00 \\
					\bottomrule
				\end{tabular}
				\caption{Secondo Periodo - Consuntivo}
				\label{tab:consuntivo2}	
			\end{table}

		\subsubsection{Conclusioni}
		Come si può notare dalla tabella \ref{tab:consuntivo2} le ore previste sono state di poco superate, ma per quanto riguarda il bilancio non ci sono stati cambiamenti rispetto a quanto preventivato ed esso risulta quindi essere \textbf{in pari}.
		
			Per il ruolo di \responsabilediprogetto\ é stato sufficiente un numero inferiore di ore rispetto al preventivo per via di un minor numero di documenti da validare.
		
			Viste le numerose correzioni apportate ad alcuni documenti é stato necessario verificarli più di quanto previsto, quindi il ruolo di \verificatore\ ha richiesto un numero di ore leggermente superiore.


	\subsection{Terzo Periodo}
		\subsubsection{Consuntivo}
		Per il Terzo periodo, Progettazione Architetturale, sono state necessarie le ore di lavoro riportate nella seguente tabella:
		
		\begin{table}[h]
				\centering
				\begin{tabular}{l * {2}{c}}
					\toprule
					\textbf{Ruolo} & \textbf{Ore} & \textbf{Costo (\euro{})} \\
					\midrule
					Responsabile & 18 & 540,00 \\

					Amministratore & 16 & 320,00 \\

					Progettista & 70 (+7) & 1.694,00 \\

					Analista & 36 (-6) & 690,00 \\		

					Programmatore & 0 & 0,00 \\		

					Verificatore & 31 (-2) & 435,00 \\				
					\midrule		
					\textbf{Totale Preventivo} & 171 & 3.765,00 \\
					\textbf{Totale Consuntivo} & 170 & 3.679,00 \\
					\midrule
					\textbf{Differenza} & -1 & -86,00 \\
					\bottomrule
				\end{tabular}
				\caption{Terzo Periodo - Consuntivo}
				\label{tab:consuntivo3}	
			\end{table}
		
		\subsubsection{Conclusioni}
		Come evidenzia la tabella \ref{tab:consuntivo3} in questo periodo le ore effettive di lavoro necessarie sono inferiori a quanto previsto e anche il bilancio ne ha beneficiato risultando \textbf{positivo}.
		
			Vista la grande necessità, in questo periodo, del ruolo di \progettista\ e visto il grande numero di correzioni apportate all'architettura, sono stato necessarie alcune ore aggiuntive rispetto a quanto preventivato. 
			
			Per il ruolo di \analista\ le ore totali necessarie sono state diminuite a causa di una riduzione del monte ore in favore del ruolo di \progettista.
			
			Per i \verificatori\ le ore sono risultate inferiori al preventivo in quanto l'attività di verifica ha richiesto un tempo leggermente inferiore. 
		
\end{document}