\documentclass[../PianoDiQualifica.tex]{subfiles}

\begin{document}
\appendix
\section{Resoconto delle attività di verifica}
Di seguito verranno descritti ed analizzati gli esiti delle attività di verifica svolte su tutti i documenti che vengono consegnati nelle varie revisioni di avanzamento del progetto.
	
	\subsection{Revisione dei requisiti}
		\subsubsection{Tracciamento}
		\kpanic\ ha deciso di utilizzare l'applicativo \gl{Trender} per facilitare il tracciamento sia delle relazioni fra casi d'uso e requisiti, che quelle fra requisiti e classi.
			
		\subsubsection{Analisi statica dei documenti}
		L'analisi statica dei documenti è stata fatta mediante Walkthrough ed ha portato all'individuazione di alcuni errori. Tra gli errori individuati quelli più frequenti sono stati:
		\begin{itemize}
			\item Errori nei concetti esposti;
			\item Violazioni di quanto stabilito nelle norme tipografiche;
			\item Aggettivi o verbi utilizzati in modo scorretto;
			\item Periodi troppo lunghi o complessi da capire ed interpretare.
		\end{itemize}
		Durante la verifica manuale sono stati anche individuati i termini da aggiungere al \glossarioRR. Nello specifico sono stati individuati 161 termini. Successivamente per il \glossarioRP\, sono stati individuati altri 17 termini da aggiungervi per un totale di 178 termini.
			
		\subsubsection{Verifiche automatizzate}
		Come spiegato nelle \normediprogetto\ sono stati utilizzati degli applicativi software per l'individuazione dei valori delle metriche per garantire la qualità del prodotto.\\
		In seguito ai test effettuati sui documenti sono stati ottenuti i risultati riportati di seguito. \\ \\
		
		\textbf{Disponibilità di Trender}
		\begin{longtable}[c] { >{\centering\arraybackslash}p{3cm} >{\centering\arraybackslash}p{3cm} }
			\toprule
				\textbf{Valore} & \textbf{Esito} \\
			\midrule
				100\% & Superato \\
			\bottomrule
			\caption{Disponibilità di Trender - RR}
		\end{longtable}\mbox{}\\
		
		\textbf{Tempo di correzione di Trender}
		\begin{longtable}[c] { >{\centering\arraybackslash}p{3cm} >{\centering\arraybackslash}p{3cm} }
			\toprule
					\textbf{Valore} & \textbf{Esito} \\
				\midrule
					0 & Superato \\
				\bottomrule
			\caption{Tempo di correzione di Trender - RR}
		\end{longtable}\mbox{}\\
		
		\textbf{Rischi non preventivati}
		\begin{longtable}[c] { >{\centering\arraybackslash}p{3cm} >{\centering\arraybackslash}p{3cm} }
			\toprule
					\textbf{Valore} & \textbf{Esito} \\
				\midrule
					3 & Superato \\
				\bottomrule
			\caption{Rischi non preventivati - RR}
		\end{longtable}\mbox{}\\
		
		\textbf{Indice Gulpease}
		\begin{longtable}[c] { p{5cm} >{\centering\arraybackslash}p{3cm} >{\centering\arraybackslash}p{3cm}}
			\toprule
					\textbf{Documento} & \textbf{Gulpease} & \textbf{Esito} \\
				\midrule
					\analisideirequisitiRR & 54 & Superato \\
					\glossarioRR & 60 & Superato \\
					\normediprogettoRR & 54 & Superato \\
					\pianodiprogettoRR & 53 & Superato \\
					\pianodiqualificaRR & 53 & Superato \\
					\studiodifattibilitaRR & 57 & Superato \\		
				\bottomrule
			\caption{Indice Gulpease - RR}
		\end{longtable}\mbox{}\\
		
		\textbf{Errori ortografici rilevati e non corretti}
		\begin{longtable}[c] { >{\centering\arraybackslash}p{3cm} >{\centering\arraybackslash}p{3cm} }
			\toprule
					\textbf{Valore} & \textbf{Esito} \\
				\midrule
					0\% & Superato \\
				\bottomrule
			\caption{Errori ortografici rilevati e non corretti - RR}
		\end{longtable}\mbox{}\\
		
		\textbf{Errori concettuali rilevati e non corretti}
		\begin{longtable}[c] { >{\centering\arraybackslash}p{3cm} >{\centering\arraybackslash}p{3cm} }
			\toprule
					\textbf{Valore} & \textbf{Esito} \\
				\midrule
					0\% & Superato \\
				\bottomrule
			\caption{Errori concettuali rilevati e non corretti - RR}
		\end{longtable}\mbox{}\\



	\subsection{Revisione di progettazione}
		Per quanto riguarda le sezioni \textit{Tracciamento} e \textit{Analisi statica dei documenti} non vi sono state variazioni dalla RR.
		\subsubsection{Verifiche automatizzate}
		Come spiegato nelle \normediprogetto\ sono stati utilizzati degli applicativi software per l'individuazione dei valori delle metriche per garantire la qualità del prodotto.\\
		In seguito ai test effettuati sui documenti sono stati ottenuti i risultati di seguito riportati.\\ \\
		
		\textbf{Disponibilità di Trender}
		\begin{longtable}[c] { >{\centering\arraybackslash}p{3cm} >{\centering\arraybackslash}p{3cm} }
			\toprule
					\textbf{Valore} & \textbf{Esito} \\
				\midrule
					83\% & Superato \\
				\bottomrule
			\caption{Disponibilità di Trender - RP}
		\end{longtable}\mbox{}\\
		
		\textbf{Tempo di correzione di Trender}
		\begin{longtable}[c] { >{\centering\arraybackslash}p{3cm} >{\centering\arraybackslash}p{3cm} }
			\toprule
					\textbf{Valore} & \textbf{Esito} \\
				\midrule
					5 & Non superato \\
				\bottomrule
			\caption{Tempo di correzione di Trender - RP}
		\end{longtable}\mbox{}\\
		
		\textbf{Rischi non preventivati}
		\begin{longtable}[c] { >{\centering\arraybackslash}p{3cm} >{\centering\arraybackslash}p{3cm} }
			\toprule
					\textbf{Valore} & \textbf{Esito} \\
				\midrule
					5 & Superato \\
				\bottomrule
			\caption{Rischi non preventivati - RP}
		\end{longtable}\mbox{}\\
		
		\textbf{Numero di metodi per classe}
		Per la classe Question dentro il \gl{package} ManageQuestion si ha il seguente risultato:
		\begin{longtable}[c] { >{\centering\arraybackslash}p{3cm} >{\centering\arraybackslash}p{3cm} }
			\toprule
					\textbf{Valore} & \textbf{Esito} \\
				\midrule
					14 & Non superato \\
				\bottomrule
			\caption{Numero di metodi per classe (Question) - RP}
		\end{longtable}\mbox{}\\
		
		Per le classi rimanenti si ha invece:
		\begin{longtable}[c] { >{\centering\arraybackslash}p{3cm} >{\centering\arraybackslash}p{3cm} }
			\toprule
					\textbf{Valore} & \textbf{Esito} \\
				\midrule
					massimo 9 & Superato \\
				\bottomrule
			\caption{Numero di metodi per classe (generale) - RP}
		\end{longtable}\mbox{}\\
		
		\textbf{Numero di parametri per metodo}
		\begin{longtable}[c] { >{\centering\arraybackslash}p{3cm} >{\centering\arraybackslash}p{3cm} }
			\toprule
					\textbf{Valore} & \textbf{Esito} \\
				\midrule
					massimo 5 & Superato \\
				\bottomrule
			\caption{Numero di parametri per metodo - RP}
		\end{longtable}\mbox{}\\			

		\textbf{Indice Gulpease}
		\begin{longtable}[c] {p{5cm} >{\centering\arraybackslash}p{3cm} >{\centering\arraybackslash}p{3cm}}
			\toprule
					\textbf{Documento} & \textbf{Gulpease} & \textbf{Esito} \\
				\midrule
					\analisideirequisitiRP & 71 & Superato \\
					\definizionediprodottoRP & 60 & Superato \\
					\glossarioRP & 63 & Superato \\
					\normediprogettoRP & 78 & Superato \\
					\pianodiprogettoRP & 67 & Superato \\
					\pianodiqualificaRP & 70 & Superato \\		
				\bottomrule
			\caption{Indice Gulpease - RP}
		\end{longtable}\mbox{}\\
	
		\textbf{Errori ortografici rilevati e non corretti}
		\begin{longtable}[c] { >{\centering\arraybackslash}p{3cm} >{\centering\arraybackslash}p{3cm} }
			\toprule
					\textbf{Valore} & \textbf{Esito} \\
				\midrule
					0\% & Superato \\
				\bottomrule
			\caption{Errori ortografici rilevati e non corretti - RP}
		\end{longtable}\mbox{}\\
		
		\textbf{Errori concettuali rilevati e non corretti}
		\begin{longtable}[c] { >{\centering\arraybackslash}p{3cm} >{\centering\arraybackslash}p{3cm} }
			\toprule
					\textbf{Valore} & \textbf{Esito} \\
				\midrule
					0\% & Superato \\
				\bottomrule
			\caption{Errori concettuali rilevati e non corretti - RP}
		\end{longtable}\mbox{}\\
\end{document}
