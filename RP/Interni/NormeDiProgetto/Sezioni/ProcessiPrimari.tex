\documentclass[../NormeDiProgetto.tex]{subfiles}
\begin{document}

\section{Processi primari}
   
      \subsection{Processo di fornitura}
            \subsubsection{Studio di fattibilità}
                  Alla pubblicazione dei capitolati d'appalto il responsabile di progetto dovrà fissare un numero di riunioni volte alla discussione e al confronto tra i membri del team. In seguito, gli \analisti\ dovranno redigere lo \studiodifattibilita\ in base a quanto emerso nelle riunioni. \\ Lo  \studiodifattibilita\ sarà articolato nei seguenti punti:
                  \begin{itemize}
                        \item \textbf{Dominio applicativo}: descrizione dell'ambito di utilizzo del prodotto richiesto;
                        \item \textbf{Dominio tecnologico}: descrizione delle tecnologie impiegate nello sviluppo del progetto richiesto;
                        \item \textbf{Criticità}: elenco delle possibili problematiche che potrebbero sorgere durante lo sviluppo del prodotto richiesto, individuando quindi punti critici ed eventuali rischi;
                        \item \textbf{Valutazione finale}: piccolo riassunto finale nel quale verranno spiegate le motivazioni per cui sia stato scelto o scartato il suddetto capitolato.
                  \end{itemize}

      \subsection{Analisi}
            \subsubsection{Analisi dei requisiti}
                  Ultimato lo \studiodifattibilita\ gli \analisti\ dovranno redigere l'\analisideirequisiti\ che dovrà obbligatoriamente essere strutturata nel seguente modo:
                  
                  \paragraph{Classificazione dei requisiti}
                  Dovrà essere redatto un elenco di requisiti emersi durante le riunioni interne e/o esterne. Questo compito spetta agli \analisti. I requisiti dovranno essere classificati secondo la seguente codifica:
                  \begin{equation*}
                        R[Tipo][Importanza][Codice]
                  \end{equation*}
                  dove:
                  \begin{itemize}
                        \item \textbf{Tipo}: può avere i seguenti valori:
                        \begin{itemize}
                              \item \textbf{F}: indica un requisito funzionale;
                              \item \textbf{Q}: indica un requisito di qualità;
                              \item \textbf{P}: indica un requisito prestazionale;
                              \item \textbf{V}: indica un requisito di vincolo.
                        \end{itemize}
                        \item \textbf{Importanza}: può assumere i seguenti valori:
                        \begin{itemize}
                              \item \textbf{O}: indica un requisito obbligatorio;
                              \item \textbf{D}: indica un requisito desiderabile;
                              \item \textbf{F}: indica un requisito facoltativo.
                        \end{itemize}
                        \item \textbf{Codice}: indica il codice identificativo del requisito, è univoco e deve essere indicato in forma gerarchica.
                  \end{itemize}
                  Per ogni requisito si dovrà inoltre indicare:
                  \begin{itemize}
                        \item \textbf{Descrizione}: una breve descrizione, deve essere meno ambigua possibile;
                        \item \textbf{Fonte}: la fonte può essere una delle seguenti:
                        \begin{itemize}
                              \item \textbf{Capitolato}: deriva direttamente dal testo del capitolato;
                              \item \textbf{Verbale}: deriva da un incontro verbalizzato;
                              \item \textbf{Interno}: deriva da discussioni interne al team;
                              \item \textbf{Casi d'uso}: deriva da uno o più casi d'uso.
                        \end{itemize}
                  \end{itemize}

                  \paragraph{Classificazione dei casi d'uso}
                  I casi d'uso identificati devono essere descritti nel seguente modo:
                  \begin{equation*}
                       UC[Codice padre].[Codice identificativo]
                  \end{equation*}
                  dove:
                  \begin{itemize}
                        \item \textbf{Codice padre}: indica il codice del caso d'uso padre di quello in esame, se non è identificabile è da omettere;
                        \item \textbf{Codice identificativo}: codice univoco e progressivo del caso d'uso in esame.
                  \end{itemize}
                  Per ogni caso d'uso devono inoltre essere identificate le seguenti informazioni:
                  \begin{itemize}
                        \item \textbf{Nome}: indica il nome del caso d'uso;
                        \item \textbf{Attori}: indica gli attori coinvolti nel caso d'uso;
                        \item \textbf{Descrizione}: chiara, precisa e concisa descrizione del caso d'uso;
                        \item \textbf{Precondizione}: indica la situazione che deve essere vera prima dell'esecuzione del caso d'uso;
                        \item \textbf{Postcondizione}: indica la situazione che deve essere vera dopo l'esecuzione del caso d'uso;
                        \item \textbf{Scenario principale}: descrizione composta dal flusso dei casi d'uso figli;
                        \item \textbf{Scenari alternativi}: descrizione composta dai casi d'uso che non appartengono al flusso principale di esecuzione;
                        \item \textbf{Estensioni}: indica quali sono tutte le estensioni, se presenti;
                        \item \textbf{Inclusioni}: indica quali sono tutte le inclusioni, se presenti;
                        \item \textbf{Generalizzazioni}: indica quali sono tutte le generalizzazioni, se presenti.
                  \end{itemize}

      \subsection{Progettazione}
            \subsubsection{Descrizione}
                  L'attività di progettazione descrive come deve essere realizzata la struttura dell'architettura software. I requisiti delineati, all'interno del documento \analisideirequisiti, devono essere utili a realizzare la documentazione specifica e a determinare le linee guida da seguire durante l'attività di codifica. Tale attività deve essere svolta in maniera ottimale e precisa dai Progettisti.

            \subsubsection{Diagrammi}
                  La progettazione deve utilizzare le seguenti tipologie di diagrammi \gl{UML}:
                  \begin{itemize}
                        \item \textbf{Diagrammi di classe}: illustrano una collezione di elementi dichiarativi di un modello come classi e tipi, assieme ai loro contenuti e alle loro relazioni;
                        \item \textbf{Diagrammi dei package}: raggruppamenti di classi in una unità di livello più alto;
                        \item \textbf{Diagrammi di attività}: illustrano il flusso di operazioni relativo ad un'attività; utilizzati soprattutto per descrivere la logica di un algoritmo;
                        \item \textbf{Diagrammi di sequenza}: descrivono una determinata sequenza di azioni dove tutte le scelte sono già state effettuate; in pratica nel diagramma non compaiono scelte, né flussi alternativi.
                  \end{itemize}

            \subsubsection{Requisiti per i progettisti}
                  I progettisti sono responsabili delle attività di progettazione. Essi sono tenuti ad avere:
                  \begin{itemize}
                        \item Profonda conoscenza di tutto ciò che riguarda il processo di sviluppo del software;
                        \item Capacità di saper anticipare i cambiamenti;
                        \item Notevole inventiva per riuscire a trovare una soluzione progettuale accettabile anche in mancanza di una metodologia che sia sufficientemente espressiva;
                        \item Capacità di individuare con rapidità e sicurezza le soluzioni più opportune.
                  \end{itemize}

            \subsubsection{Obiettivi della progettazione}
                  I tipi di diagrammi che dovranno essere prodotti sono:
                  \begin{itemize}
                        \item Progettare un software con le caratteristiche di qualità che sono state dettagliate nella fase di analisi e specifica dei requisiti;
                        \item Capacità di poter far fronte a modifiche da effettuare senza che l'intera struttura del software già costruita debba essere messa nuovamente in discussione ed elaborata;
                        \item Soddisfare i requisiti di qualità fissati dal committente.
                  \end{itemize}

            \subsubsection{Specifica tecnica}
                  La specifica tecnica contiene la progettazione architetturale di alto livello, generale e di ogni singolo componente.
                  Questo documento viene redatto dai Progettisti, che ne faranno da base per la successiva Definizione di prodotto.
                  La specifica tecnica deve basarsi sulla precedente analisi dei requisiti.
                  I contenuti principali del documento saranno quindi:
                  \begin{itemize}
                        \item \textbf{Design pattern}: i \progettisti\ devono fornire una descrizione dei \gl{design pattern} adottati
                        nella definizione dell’architettura. Questa descrizione dovrà essere accompagnata da un
                        diagramma UML, che ne esemplifichi il funzionamento, e dalle motivazioni che hanno
                        portato all’adozione di tale pattern;
                        \item \textbf{Tracciamento delle componenti}: ogni componente dovrà essere tracciato ed associato
                        ad almeno un requisito. In tal modo sarà possibile avere la certezza che tutti i requisiti
                        siano soddisfatti. Tale tracciamento dovrà essere effettuato tramite Trender, che si occupa
                        di generare in modo automatico le relative tabelle;
                        \item \textbf{Test d’integrazione}: i Progettisti devono definire delle strategie di verifica per poter
                        dimostrare la corretta integrazione tra le varie componenti definite.
                  \end{itemize}

                  I tipi di diagrammi che dovranno essere prodotti sono:
                  \begin{itemize}
                        \item Diagrammi dei package;
                        \item Diagrammi di sequenza;
                        \item Diagrammi di attività.
                  \end{itemize}

            \subsubsection{Definizione di prodotto}
                  La Definizione di Prodotto contiene la progettazione di dettaglio del sistema. Lo scopo di questo documento è quello di
                  definire dettagliatamente ogni singola unità di cui è composto il sistema in modo da semplificare
                  l’attività di codifica e allo stesso tempo di non fornire libertà al Programmatore.
                  Parallelamente alla progettazione di dettaglio dei componenti software dovranno essere progettati
                  i relativi test di unità che verranno descritti nel Piano di Qualifica.
                  I contenuti principali del documento saranno quindi:
                  \begin{itemize}
                        \item \textbf{Definizione delle classi}: ogni classe precedentemente progettata viene descritta più nel
                        dettaglio, fornendo una descrizione più approfondita dello scopo, delle sue funzionalità e del
                        suo funzionamento. Per ogni classe dovranno essere anche definiti i vari metodi e attributi
                        che la caratterizzano;
                        \item \textbf{Tracciamento delle classi}: ogni classe deve essere tracciata ed associata ad almeno un
                        requisito, in questo modo è possibile avere la certezza che tutti i requisiti accettati siano
                        soddisfatti e che ogni classe presente nell’architettura soddisfi almeno un requisito;
                        \item \textbf{Test di unità}: i Progettisti devono definire le strategie di verifica delle varie classi in modo
                        che durante l’attività di codifica sia possibile verificare che la classe si comporti in modo
                        corretto.
                  \end{itemize}

                  I tipi di diagrammi che potranno essere contenuti nel documento sono:
                  \begin{itemize}
                        \item Diagrammi dei package;
                        \item Diagrammi delle classi;
                        \item Diagrammi di sequenza.
                  \end{itemize}

            \subsubsection{Norme progettuali}
                  Durante la fase di progettazione i progettisti dovranno rispettare le seguenti direttive.
                  \paragraph{Modularizzazione}
                        I progettisti devono adottare delle tecniche che consentano la scomposizione del sistema in moduli e 
                        definire una descrizione precisa della struttura e delle relazioni che esistono tra i moduli.
                        Questo metodo di sviluppo comporterà i seguenti vantaggi:
                        \begin{itemize}
                              \item Semplificazione della verifica della correttezza semantica e nella correzione di errori;
                              \item Riusabilità del software;
                              \item Leggibilità del codice;
                              \item Testabilità delle singole unità;
                              \item Semplificazione dell’attività di manutenzione.
                        \end{itemize}
                   \paragraph{Modularizzazione AngularJS}
                        Se il progetto prevede l'utilizzo di AngularJS, devono vigere le seguenti linee guida:
                        \begin{itemize}
                              \item Le iterazione di rete devono essere eseguite mediante l'utilizzo di appositi service;
                              \item L'applicazione deve essere suddivisa in moduli indipendenti per ogni utilizzo seguendo il pattern della dependecy injection;
                              \item L'integrazione tra Moduli deve essere testata da appositi test di integrazione;
                              \item Ogni modulo deve essere testato mediante un test di unità;
                              \item E' vietato l'utilizzo di variabili globali o di scope globale.
                        \end{itemize}

                  \paragraph{Modularizzazione Node.js}
                        Se il progetto prevede l'utilizzo di Node.js, devono vigere le seguenti linee guida:
                        \begin{itemize}
                              \item Le iterazione di rete devono essere eseguite mediante l'utilizzo di appositi moduli e classi;
                              \item L'applicazione deve essere suddivisa in moduli indipendenti, ognuno per una diversa funzione;
                              \item L'integrazione tra Moduli deve essere testata da appositi test di integrazione;
                              \item Ogni modulo deve essere testato mediante un test di unità;
                              \item E' vietato l'utilizzo di variabili globali o di scope globale;
                              \item Per la mappa di path, è previsto l'uso di Total.js;
                              \item Per agevolare la compatibilità con i servizi AWS Lambda il modulo principale sarà chiamato main;
                              \item I moduli Node.js saranno rappresentati sui diagrammi UML come classi.
                        \end{itemize}

      \subsection{Codifica}
            \subsubsection{Descrizione}
                  L'attività di codifica ha come obiettivo quello di passare dalla descrizione della soluzione in termini di architettura alla descrizione della soluzione in formato eseguibile da un calcolatore. \\ I \programmatori, responsabili di questa attività, sono tenuti a seguire le linee guida, delineate nell'attività di progettazione, con lo scopo di produrre in output il software designato.

            \subsubsection{Tecniche di scrittura}
                  La definizione di tecniche di scrittura è importante per avere una maggiore comprensione del codice sorgente. Le tecniche di scrittura del codice sono suddivise in tre categorie:
                  \begin{itemize}
                        \item Nomi;
                        \item Commenti;
                        \item Formattazione.
                  \end{itemize}
                  Si è scelto di adottare i metodi proposti nel primo capitolo \textit{Style Guidelines} del libro \gl{\textit{Maintainable JavaScript}} di Nicholas C. Zakas (\url{http://shop.oreilly.com/product/0636920025245.do}(2017-01-29)).
                  
                  \paragraph{Nomi}
                  Lo schema di denominazione rappresenta uno dei supporti più determinanti per la comprensione del flusso logico del software. Di seguito sono riportate le tecniche di denominazione raccomandate:
                  \begin{itemize}
                        \item Assegnare ad ogni elemento un nome univoco e consono alla funzione svolta;
                        \item Evitare nomi poco chiari e suscettibili di interpretazioni soggettive; questo tipo di nome può contribuire a creare ambiguità piuttosto che astrazione;
                        \item Utilizzare la struttura verbo - nome per la denominazione di routine che consentono di eseguire operazioni specifiche su determinati oggetti;
                        \item Poiché la maggior parte dei nomi viene creata concatenando più parole, utilizzare una combinazione di caratteri maiuscoli e minuscoli per semplificarne la lettura;
                        \item Utilizzare un nome significativo anche per le variabili che vengono visualizzate solo in poche righe di codice;
                        \item Utilizzare nomi di variabili composti da una singola lettera, come "i" o "j", esclusivamente per gli indici a ciclo breve;
                        \item Ridurre l'utilizzo delle abbreviazioni ma utilizzare quelle create in maniera coerente. È opportuno che ad ogni abbreviazione corrisponda un solo significato e che a ciascuna parola abbreviata sia associata una sola abbreviazione. Se, ad esempio, si utilizza min come abbreviazione di minimum, in altri contesti non è possibile utilizzare la stessa abbreviazione per minute;
                        \item Nella denominazione di funzioni inserire una descrizione del valore restituito;
                        \item Evitare di utilizzare gli stessi nomi per elementi diversi, ad esempio una routine e una variabile denominate rispettivamente Getname() e iGetname;
                        \item Nella denominazione degli elementi non utilizzare omonimi per evitare ambiguità durante le revisioni del codice;
                        \item Per la denominazione degli elementi evitare l'uso di parole ortograficamente errate;
                        \item È opportuno che nei nomi di file e cartelle sia contenuta una descrizione precisa della relativa funzione.
                  \end{itemize}

                  \paragraph{Commenti}
                  All'interno del codice sorgente è di fondamentale importanza l'inserimento di commenti al fine di
                  facilitare la comprensione del flusso logico. Di seguito sono suggeriti alcuni metodi di inserimento
                  dei commenti:
                  \begin{itemize}
                        \item Quando si modifica il codice mantenere sempre aggiornati i relativi commenti;
                        \item All'inizio di ogni routine è utile fornire commenti predefiniti standard in cui siano indicati
                        le limitazioni, i presupposti e lo scopo della routine. Per commento predefinito si intende
                        una breve introduzione in cui siano illustrate le funzionalità della routine;
                        \item Evitare l'aggiunta di commenti alla fine di una riga di codice; la presenza di commenti a fine
                        riga può rendere più difficoltosa la lettura del codice. Tuttavia questo tipo di commento è
                        valido per l'annotazione di dichiarazioni di variabili. In tal caso è necessario allineare tutti
                        i commenti di fine riga a una tabulazione comune;
                        \item Evitare commenti confusi come righe intere di asterischi. Utilizzare invece spazi vuoti per
                        separare i commenti dal codice;
                        \item Evitare di racchiudere blocchi di commenti in cornici grafiche. Si tratta di un espediente
                        interessante ma di difficile gestione;
                        \item Quando si scrivono commenti, utilizzare frasi di senso compiuto. La funzione dei commenti
                        consiste nel chiarire il significato del codice senza aggiungere alcun tipo di ambiguità;
                        \item Inserire commenti in fase di scrittura del codice, in quanto ciò potrebbe non essere possibile
                        in un secondo momento. Inoltre, se dovesse presentarsi l'opportunità di rivedere il codice
                        scritto, tenere presente che ciò che può essere evidente al momento della stesura potrebbe
                        non esserlo più in futuro;
                        \item Evitare commenti superflui o inappropriati, come annotazioni umoristiche;
                        \item Indicare il funzionamento di tutto ciò che non è chiaro nel codice tramite commenti;
                        \item Per evitare problemi ricorrenti, è opportuno utilizzare sempre i commenti nel caso di codice
                        relativo a correzioni di errori e a potenziali soluzioni;
                        \item Aggiungere commenti al codice costituito da cicli e diramazioni logiche. Si tratta di aree
                        di fondamentale importanza che facilitano la lettura del codice sorgente;
                        \item Creare commenti adottando uno stile uniforme e una struttura e una punteggiatura coerenti;
                        \item Separare i commenti dai delimitatori di commento con spazi vuoti. In questo modo i
                        commenti saranno chiari e facili da individuare.
                  \end{itemize}

                  \paragraph{Formattazione}
                  Una buona e uniforme formattazione facilita la comprensione dell'organizzazione logica del codice. Per consentire agli sviluppatori la decifrazione del codice sorgente è fondamentale una formattazione logica e coerente.
                  Di seguito sono suggeriti alcuni metodi di formattazione:
                  \begin{itemize}
                        \item Definire una dimensione standard per i rientri e utilizzarla in maniera coerente. Allineare le sezioni di codice utilizzando il rientro predefinito;
                        \item Allineare le parentesi di apertura e chiusura utilizzando lo stile \gl{JavaScript}, ovvero con la parentesi di apertura alla fine della linea e quella di chiusura all'inizio della linea. Vari esempi di questo stile possono essere trovati nel libro di riferimento Maintainable JavaScript;
                        \item Rientrare le righe di codice secondo la relativa costruzione logica. Se non viene utilizzato il rientro, il codice risulterà di difficile comprensione;
                        \item Utilizzare spazi prima e dopo la maggior parte degli operatori, se ciò non altera la funzione del codice, per migliorare la leggibilità;
                        \item Utilizzare linee vuote per separare strutturalmente i blocchi di codice sorgente. In questo modo sarà possibile creare "paragrafi" di codice, che consentono di semplificare la comprensione della segmentazione logica del software da parte del lettore;
                        \item Suddividere logicamente il codice sorgente tra diversi file fisici;
                        \item Suddividere le sezioni complesse ed estese di codice in moduli comprensibili di dimensioni minori.
                  \end{itemize}

                  \paragraph{Intestazione File}
                  L’intestazione di ogni file deve contenere obbligatoriamente le seguenti informazioni:
                  \begin{itemize}
                        \item textbf{Name}: nome del file, con riferimento alla Definizione di Prodotto;
                        \item textbf{Description}: descrizione del file, con riferimento alla Definizione di Prodotto;
                        \item textbf{Creation data}: indica la data di creazione del file;
                        \item textbf{Author}: indica l’autore del file;
                        \item textbf{License}: indica il nome della licenza del file;
                        \item textbf{Update history}: indica la cronologia delle modifiche effettuate al file, è così strutturata:
                              \begin{itemize}
                              \item textbf{Update data}: data corrispondente all’ultimo aggiornamento del file;
                              \item textbf{Description}: descrizione della modifica effettuata;
                              \item textbf{Author}: autore dell’ultima modifica.
                              \end{itemize}
                  \end{itemize}

            \subsubsection{Ricorsione}
            La ricorsione va evitata quando possibile. Per ogni funzione ricorsiva sarà necessario fornire una prova di terminazione e sarà necessario valutare il costo in termini di occupazione della memoria. \\
            Nel caso l'utilizzo di memoria risulti troppo elevato la ricorsione dovrà essere rimossa.

\end{document}
