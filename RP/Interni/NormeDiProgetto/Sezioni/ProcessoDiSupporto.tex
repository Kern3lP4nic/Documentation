\documentclass[../NormeDiProgetto.tex]{subfiles}

\begin{document}

\section{Processi di supporto}
\subsection{Sviluppo documentazione}
\subsubsection{Descrizione}
I documenti verranno classificati in due categorie che vengono distinte a scopi puramente semplificativi:
\begin{itemize}
\item Interni: documenti riguardati l'organizzazione e le regole imposte ai membri del team\ e destinati quindi ad un uso interno;
\item Esterni: documenti destinati al committente.
\end{itemize}
\subsubsection{Denominazione}
I documenti approvati devono avere il nome strutturato nel seguente modo:
\begin{itemize}
\item Il nome deve essere scritto seguendo la forma \gl{PascalCase} ovvero con la prima lettera maiuscola per ogni parola del nome inclusa la prima;
\item Ogni documento durante lo sviluppo deve avere un file.tex che unisce tutte le sezioni, e ogni sezione, deve essere un file assestante;
\item Il documento prodotto deve essere nel formato PDF.
\end{itemize}
\subsubsection{Directory}
Le cartelle relative alla documentazione devono avere il nome scritto nello stesso modo dei file, ma non hanno versione. La forma da seguire è sempre la PascalCase.
L'organizzazione sarà ad albero: il primo livello rappresenterà le consegne e saranno:
\begin{itemize}
\item RR;
\item RQ;
\item RA;
\item RP.
\end{itemize}
Il secondo livello sarà invece la documentazione, ci sarà quindi una cartella per ogni documento.
In ogni cartella del documento, sarà presente un file \LaTeX\ che rappresenta l'intero documento e una sottodirectory che contiene le sezioni.

\subsubsection{Strumenti di sviluppo}
Per la stesura della documentazione è stato scelto di utilizzare il \gl{linguaggio di markup} \LaTeX\ per poter standardizzare meglio i documenti. Quest'ultimo difatti permette di gestire in maniera centralizzata il \gl{template} e lo stile della documentazione e di riusarne le parti comuni. Inoltre supporta nativamente gli indici e i glossari.

\subsubsection{Ciclo di vita dei documenti}
I documenti in tutto il loro ciclo di vita avranno uno dei seguenti stati:
\begin{itemize}
\item Sviluppo: sono quelli in fase di stesura dal relativo redattore;
\item Verifica: sono quelli del quale è terminata la stesura e sono in attesa di verifica che verrà effettuata dal \gl{\responsabilediprogetto};
\item Approvati: è la versione finale del documento che verrà presentata in via ufficiale.
\end{itemize}
Ad ognuno di questi passaggi deve corrispondere un cambio di versione di secondo livello.

\subsubsection{Documenti finali}
\begin{itemize}
\item \textbf{Studio di fattibilità}: questo documento riporta l'analisi dei punti principali del progetto, per verificarne la fattibilità nei tempi prestabiliti, cercando anche eventuali soluzioni tecniche già presenti nel mercato, per ridurne costi e tempi di sviluppo;
\item \textbf{Norme di progetto}: questo documento riporta tutte le convenzioni, strumenti e norme che si dovranno usare durante lo sviluppo di un qualsiasi progetto del team;
\item \textbf{Piano di Progetto}: questo documento descrive come i membri del team hanno intenzione di gestire le risorse umane e temporali al fine dello sviluppo del progetto;
\item \textbf{Piano di Qualifica}: l'obbiettivo di questo documento è descrivere il modo in cui l'intero team punta a soddisfare gli obiettivi di qualità. Questo documento è utilizzato anche esternamente e la lista di distribuzione
comprende committenti e \gl{proponenti};
\item \textbf{Analisi dei Requisiti}: questo documento ha lo scopo di dare una visione generale dei requisiti e dei casi d'uso del progetto.
Esso conterrà quindi la lista di tutti i casi d'uso, i diagrammi delle attività di interazione tra utente e sistema sviluppato e i servizi offerti dal prodotto finale. Questo documento è utilizzato anche esternamente e la lista di distribuzione comprende committenti e proponenti;
\item \textbf{Definizione del prodotto}: in questo documento verrà fornita una progettazione dettagliata del prodotto. Verranno quindi forniti i dettagli implementativi che comprenderanno diagrammi UML delle classi e i relativi metodi;
\item \textbf{Glossario}: questo documento dovrà dare una definizione dettagliata di tutti i termini tecnici e acronimi utilizzati nell'intera documentazione. Questo documento è utilizzato anche esternamente e la lista di distribuzione comprende committenti e proponenti.
\item \textbf{Manuale Utente}: questo documento dovrà fornire una guida di tutte le funzionalità del prodotto sviluppato.
Come linea interna, la verifica finale, sarà definita verificando che tutti i casi d'uso siano nel manuale.
\item \textbf{Verbali}: questo documento ha lo scopo di riassumere in modo formale le discussioni effettuate e le decisioni
prese durante le riunioni. I verbali, come le riunioni, sono classificati in: interni ed esterni. In particolare i verbali esterni, essendo documenti ufficiali, devono essere redatti dal \responsabilediprogetto.
Ogni verbale dovrà essere denominato nel seguente modo:
\begin{equation*}
	Verbale-Data\ del\ Verbale
\end{equation*}
dove Data del Verbale identifica la data nella quale si è svolta la riunione corrispondente al verbale con i formato
\begin{equation*}
	YYYY-MM-DD
\end{equation*}

Nella parte introduttiva del verbale devono essere specificate le seguenti informazioni:
\begin{itemize}
  \item \textbf{Data incontro};
  \item \textbf{Ora inizio incontro};
  \item \textbf{Ora termine incontro};
  \item \textbf{Luogo incontro};
  \item \textbf{Oggetto};
  \item \textbf{Segretario};
  \item \textbf{Partecipanti}.
\end{itemize}
\end{itemize}
\subsubsection{Struttura del documento}
\paragraph{Prima pagina}

In ogni prima prima pagina di ogni documento, devono essere contenute le seguenti informazioni per chiarezza:
\begin{itemize}
  \item Nome del gruppo;
  \item Logo del progetto;
  \item Nome del progetto;
  \item Nome del documento;
  \item Versione del documento;
  \item Data di creazione del documento;
  \item Data di ultima modifica del documento;
  \item Stato del documento;
  \item Nome e cognome del redattore del documento;
  \item Nome e cognome del \verificatore\ del documento;
  \item Nome e cognome del \responsabilediprogetto\ approvatore del documento;
  \item Uso del documento;
  \item Lista di distribuzione del documento;
  \item Destinatari del documento;
  \item Un sommario contenente una breve descrizione del documento.
\end{itemize}

\paragraph{Diario delle modifiche}
La seconda sezione del documento è assegnata al diario delle modifiche. Questa sezione contiene il report di tutte le versioni del documento, in quale data ne è stata cambiata la versione e cosa è stato modificato. I dati verranno espressi in maniera tabulare, con i seguenti campi:
\begin{itemize}
  \item \textbf{Versione}: versione del documento;
  \item \textbf{Descrizione}: descrizione della modifica;
  \item \textbf{Autore e Ruolo}: autore della modifica e ruolo che esso ricopre;
  \item \textbf{Data}: data dell'approvazione.
\end{itemize}
Questa operazione di versionamento verrà fatta automaticamente da uno \gl{script} \gl{Perl} elaborato dal team.

\paragraph{Indice}
In ogni documento come terza sezione, dopo il diario delle modifiche, deve essere presente l'indice di tutte le sezioni, sottosezioni, tabelle e paragrafi.
L'indice verrà generato automaticamente dal compilatore di \LaTeX .

\paragraph{Formattazione generale delle pagine}
Nella formattazione delle pagine è previsto un header e un piè di pagina.
L'header della pagina contiene:
\begin{itemize}
	\item Logo del gruppo;
 	\item Nome del documento.
\end{itemize}
Il piè di pagina contiene:
\begin{itemize}
  \item Il nome del gruppo;
  \item Nome del progetto;
  \item Numero della pagina corrente.
\end{itemize}

\subsubsection{Norme tipografiche}
Le seguenti norme tipografiche indicano i criteri riguardanti la tipografia dei documenti.

\paragraph{Stili}
\begin{itemize}
  \item \textbf{Grassetto}: il grassetto deve essere utilizzato per evidenziare parole importanti;
  \item \textbf{Corsivo}: il corsivo deve essere utilizzato nelle seguenti
  situazioni:
  \begin{itemize}
    \item Citazioni: ogni citazione va scritta in corsivo;
    \item Nomi di capitolati: ogni capitolato preso in esame, deve avere il nome scritto in corsivo.
  \end{itemize}
\end{itemize}

\paragraph{Punteggiatura}
\begin{itemize}
  \item \textbf{Punteggiatura}: non deve seguire spazi;
  \item \textbf{Lettere maiuscole}: vanno utilizzate dopo il punto, il punto interrogativo,
  il punto esclamativo e all'inizio di ogni elemento di un elenco puntato.
\end{itemize}

\paragraph{Composizione del testo}
\begin{itemize}
  \item \textbf{Elenchi puntati}: all'inizio di ogni punto, la lettera deve essere maiuscola. Ogni punto termina con il punto e virgola, tranne l'ultima che deve terminare con il punto;
   \item \textbf{Glossario}: ogni termine presente nel \glossario\ deve essere munito di pedice utilizzando la segnatura \verb|\gl{parola}|, quest'ultimo termine si troverà nell'apposito documento.
\end{itemize}

\paragraph{Formati}
\begin{itemize}
   \item \textbf{Date}: le date presenti nei documenti devono seguire lo standard \textit{ISO 8601:2004}:\\
   \begin{equation*}
     YYYY-MM-DD
   \end{equation*}
   Corrispondenti alla data americana, con anno-mese-giorno;
   \item \textbf{Ore}: le ore presenti nei documenti devono seguire lo standard \textit{ISO 8601:2004}
   con il sistema a 24 ore:\\
     \begin{equation*}
     		hh:mm
     \end{equation*}
   \item \textbf{Link}: i link presenti nei documenti devono avere indicata la data corrispondente all'ultimo controllo di effettiva presenza dei contenuti con il seguente formato:\\
   \begin{equation*}
   	https://www.example.com/example.pdf(YYYY-MM-DD)
  	\end{equation*}
\end{itemize}

\subsubsection{Componenti grafiche}
\begin{itemize}

\item \textbf{Tabelle}: tutte le tabelle presenti devono avere una descrizione ed un indice univoco nel documento per il loro tracciamento;
\item \textbf{Immagini}: l'inserimento di immagini nel documento è sconsigliato quando è possibile evitarne l'uso, ma nel caso sia necessario per rendere chiaro un concetto come diagrammi o screenshot, il formato deve essere PNG.
\end{itemize}

\subsubsection{Versionamento}
Ogni documento prodotto deve essere identificato, oltre che dal nome, dal numero
di versione nel seguente modo:
\begin{equation*}
  vA.B.C
\end{equation*}
dove:
\begin{itemize}
  \item \textbf{A}: indica il numero di uscite formali del documento e viene
  incrementato in seguito all'approvazione finale da parte del \responsabilediprogetto.
  L'incremento dell'indice \textbf{A} comporta l'azzeramento degli indici
  \textbf{B} e \textbf{C};
  \item \textbf{B}: indica il numero crescente delle verifiche. L'incremento viene 	eseguito dal \verificatore\ e comporta l'azzeramento dell'indice \textbf{C};
  \item \textbf{C}: indica il numero di modifiche minori ma percepibili apportate al documento prima della sua verifica.
  \end{itemize}

\subsubsection{Composizione email}
In questo paragrafo verranno descritte le norme da applicare nella
composizione delle email.
\begin{itemize}
	\item \textbf{Destinatario}:
	\begin{itemize}
  		\item Interno: l'indirizzo da utilizzare è \textit{kern3lp4nic.team@gmail.com} che, mediante l'inoltro, arriverà a tutti i componenti del team;
  		\item Esterno: l'indirizzo del destinatario è variabile.
	\end{itemize}
	\item \textbf{Mittente}:
	\begin{itemize}
  		\item Interno: l'indirizzo è di colui che scrive e spedisce la email;
  		\item Esterno: l'indirizzo da utilizzare è \textit{kern3lp4nic.team@gmail.com} ed è utilizzabile unicamente dal \responsabilediprogetto.
	\end{itemize}
	\item \textbf{Oggetto}: l'oggetto della mail deve essere breve e contenere la motivazione dell'invio della mail in forma breve;
	\item \textbf{Corpo}: il testo deve essere chiaro, e deve riguardare esclusivamente l'oggetto;
	\item \textbf{Allegati}: è sconsigliato inviare allegati attraverso l'email, è invece richiesto, ove possibile, di condividere il file in \gl{Google Drive} e di inserire il link al file nel corpo dell'email.
\end{itemize}

\subsection{Processo di verifica}
\subsubsection{Analisi}
\begin{itemize}
\item \textbf{Analisi statica}:
l'analisi statica è una tecnica che permette di trovare eventuali anomalie nella
documentazione o nel software prodotto. Ci sono due modi con la quale essa viene
impiegata:
\begin{itemize}
  \item \textbf{\gl{Walkthrough}}: questa tecnica consiste nella lettura a largo spettro del documento o del codice, al fine di trovare anomalie, senza avere un'idea precisa degli errori da cercare.
Questa tecnica risulta molto utile nel periodo iniziale dello sviluppo del prodotto data la scarsa esperienza. Il \verificatore\ dovrà stilare una lista degli errori più frequenti trovati nel documento da lui analizzato. Quando la lista sarà abbastanza corposa, potrà essere allegata ad una versione di questo documento come indice per l'utilizzo della più sottile tecnica \gl{Inspection};
  \item \textbf{Inspection}: questa tecnica di analisi statica consiste in una lettura più mirata dei documenti o del codice, utilizzando come supporto fondamentale la lista di controllo contenente gli errori più frequenti. Questa tecnica è migliorativa grazie al continuo aggiornamento della lista degli errori, e può essere implementata in un algoritmo per quanto riguarda il codice.
\end{itemize}
\item \textbf{Analisi dinamica}:
questo tipo di analisi, viene applicata solamente al codice, dato che ne richiede la compilazione per un'esecuzione. Essa serve a verificare che il software prodotto produca il risultato aspettato.
\end{itemize}
\subsubsection{Test}
\begin{itemize}
	\item \textbf{Test di unità}:
		i test di unità verificano che una singola componente funzioni correttamente. Effettuando questi test si riduce al minimo la presenza di errori di singole funzioni o moduli. Essi verranno marcati con seguente sintassi:\\
		\begin{equation*}
  			TU[Codice \ Test]
		\end{equation*}
	\item \textbf{Test di integrazione}:
		i test di integrazione verificano che più unità, validate singolarmente, funzionino correttamente una volta assemblate. Essi verranno marcati con seguente sintassi:\\
\begin{equation*}
  TI[Codice \ Test]
\end{equation*}
\item \textbf{Test di regressione}:
il test di regressione viene effettuato su tutte le componenti software che subiscono delle modifiche e a tutte le sue dipendenze.
Essi verranno marcati con seguente sintassi:\\
\begin{equation*}
  TR[Codice \ Test]
\end{equation*}
\item \textbf{Test di sistema}:
i test di sistema vengono eseguiti sul prodotto che verrà rilasciato. E conterrà tutti i requisiti richiesti al prodotto. Se questo test verrà superato con successo, il prodotto verrà considerato completo.
Essi verranno marcati con seguente sintassi:\\
\begin{equation*}
  TS[Codice \ Requisito]
\end{equation*}
\item \textbf{Test di validazione}:
il test di validazione viene effettuato con il proponente. Se tutti i requisiti sono soddisfatti e il proponente lo approva, allora il prodotto verrà rilasciato.
Essi verranno marcati con seguente sintassi:\\
\begin{equation*}
  TV[Codice \ Requisito]
\end{equation*}
\end{itemize}

\subsubsection{Verifica dei documenti}
	La verifica dei documenti deve essere eseguita dai \verificatori\ assegnati a ciascun documento. Qualsiasi tipo di errore, incongruenza o dubbio dovrà essere segnalato nel sistema di gestione dei task interno \gl{Teamwork} nel seguente modo:
	\begin{itemize}
		\item Se non presente, aggiungere una tasklist con nome:\\
		\begin{equation*}
			Verifica [TitoloDelDocumento] vX.X.X
		\end{equation*}
		dove X.X.X indica la versione del documento in esame;
		\item Ciascun problema dovrà essere segnalato con un subtask identificato dalla frase in cui si trova e una breve descrizione del problema;
		\item La tasklist dovrà essere assegnata ai redattori del documento, al fine di applicare le modifiche, dopo attenta valutazione;
		\item Se ci fossero dubbi su come risolvere un subtask, il redattore deve richiedere maggiori chiarimenti ai \verificatori\ tramite lo strumento di comunicazione interna Slack nell'apposito canale \#documenti.
	\end{itemize}

\subsection{Strumenti}
	\subsubsection{Strumenti per l'analisi statica}
		\begin{itemize}
  			\item \textbf{\gl{JSHint}};
 			\item \textbf{\gl{W3C markup}};
  			\item \textbf{Analisi con \gl{Gulpease}};
 			\item \textbf{Texmaker}
        \item \textbf{\gl{SonarQube}}.

		\end{itemize}
	\subsubsection{Strumenti per l'analisi dinamica}
		\begin{itemize}
 			\item \textbf{\gl{Google Chrome DevTools}};
 			\item \textbf{\gl{Jenkins} + \gl{Mocha}}.
	\end{itemize}
\end{document}
