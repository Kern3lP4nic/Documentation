\documentclass[../NormeDiProgetto.tex]{subfiles}

\begin{document}

\section{Processi organizzativi}
\subsection{Processo di coordinamento}
\subsubsection{Comunicazioni}
\begin{itemize}
\item \textbf{Responsabilità}:
possiamo suddividere le comunicazioni in due gruppi, uno interno ed uno esterno;
\item \textbf{Comunicazioni interne}:
a seconda della quantità e importanza delle informazioni che si vogliono trasmettere.
In base a questi criteri, le forme di comunicazione interna che possono essere adottate sono:
\begin{itemize}
\item Comunicazione formale scritta: questa modalità di comunicazione viene adottate per tutte le comunicazioni di fondamentale importanza per la corretta gestione del progetto. Per questo tipo di comunicazione si utilizzeranno i verbali riunione o il documento \gl{\pianodiprogettov};
\item Comunicazione verbale formale: le comunicazioni a carattere divulgativo o di presentazione. Per questo tipo comunicazione verranno indetti dei discorsi all'interno delle riunioni, in seguito specificate nel verbale;
\item Comunicazione informale scritta: comunicazioni che riguardano due o più persone del quale si vuole dare tracciamento o che possono servire a puntualizzare o chiarire alcuni aspetti. Per questo tipo comunicazione si utilizzeranno email o la piattaforma di messaggistica Slack;
\item Comunicazione informale orale: comunicazioni che sono di minor importanza nel quale vengono scambiate opinioni o consigli. Per questo tipo di conversazione non sono necessari particolari protocolli.
\end{itemize}
\item \textbf{Comunicazioni esterne}:
il team ha ritenuto necessario creare una casella di posta elettronica per comunicare uniformemente con terze parti denominata:\\
\begin{equation*}
	kern3lp4nic.team@gmail.com
\end{equation*}
La redazione delle mail per comunicare con i committenti è affidata al \responsabilediprogetto, mentre la comunicazione a tutti i membri avverrà mediante inoltro automatico.
\end{itemize}
\subsubsection{Riunioni}
	\paragraph{Riunioni interne}
		La convocazione delle riunioni avrà cadenza settimanale, il \responsabilediprogetto\ invierà una email ad ogni membro del gruppo informandolo della riunione oltre che informare ove possibile mediante altre piattaforme.
Affinché una riunione abbia validità, è necessario che siano presenti almeno la metà più uno dei membri del team, in caso contrario la riunione sarà annullata.
Le riunioni interne si dividono in quattro tipi:
\begin{itemize}
	\item Riunioni per informare;
	\item Riunioni per valutare;
	\item Riunioni per decidere;
	\item Riunioni per progettare.
\end{itemize}
	\paragraph{Riunioni esterne}
		Le riunioni esterne, che sono tenute con il/i committente/i, sono ritenute di fondamentale importanza per instaurare un rapporto di fiducia tra le parti, condividere gli obiettivi e i bisogni.
L'organizzazione delle riunioni esterne sarà affidata al \responsabilediprogetto\ in carica o da qualunque altro membro del team che abbia avuto l'incarico.
Al termine di ogni riunione, il \responsabilediprogetto\ o un suo delegato dovrà verbalizzare quanto emerso, eventuali richieste di cambiamenti riguardanti elementi che possono avere degli impatti negativi sulla realizzazione del progetto e porre le necessarie misure ove necessario;
	\paragraph{Ruoli}
		\begin{itemize}
			\item \textbf{Moderatore}:
				il \responsabilediprogetto\ ha il compito di convocare le riunioni interne, incontri nel quale partecipano esclusivamente i membri del team. Deve fare da guida nella discussione degli argomenti, essere fonte di ispirazione e avere l'autorità per seguire l'ordine del giorno.
Essendo il coordinatore della riunione, il \responsabilediprogetto, è tenuto ad essere serio, ma di cedere alla mediazione ove possibile;
			\item \textbf{Segretario}:
				il \segretario\ ha il compito di tenere la minuta dell'incontro, controllare che siano stati discussi tutti i punti previsti dalla riunione e di redigere il verbale. Alla fine di ogni riunione deve inviare ai partecipanti una copia del verbale dell'incontro. Ciò consentirà di mantenere informato anche chi non ha potuto partecipare al \gl{meeting};
			\item \textbf{Partecipanti}:
				ad ogni incontro i membri del gruppo devono sentirsi responsabili ad arrivare informati per affrontare al meglio l'ordine del giorno.
Ogni membro del gruppo può richiedere al \responsabilediprogetto\ di indire una riunione interna extra a quelle già previste, motivando la richiesta e spetterà sempre al \responsabilediprogetto\ decidere se approvare o meno la richiesta.

		\end{itemize}

\subsection{Processo di pianificazione}

\subsubsection{Ruoli}
Durante l'intero sviluppo del progetto didattico ogni componente del gruppo
dovrà obbligatoriamente cimentarsi in tutti i ruoli elencati di seguito.
Inoltre non potrà mai accadere che un membro del gruppo risulti redattore e verificatore di un medesimo documento o lavoro.
Un membro può inoltre ricoprire più ruoli contemporaneamente.

\paragraph{Responsabile di progetto}
Il \responsabilediprogetto\ è colui che si assume la responsabilità del
lavoro svolto. Rappresenta inoltre colui che mantiene i contatti tra il fornitore e il cliente e funge da mediatore tra i membri del gruppo. Più in dettaglio, ha responsabilità su:
\begin{itemize}
  \item Pianificazione, coordinamento e controllo generale delle attività;
  \item Gestione delle risorse;
  \item Gestione e approvazione della documentazione;
  \item Contatti con i soggetti esterni.
\end{itemize}
Il \responsabilediprogetto\ ha l'incarico di creare, assegnare ad ogni membro e gestire i singoli task. É l'unica persona in grado di approvare in modo definitivo un documento.
Può essere abbreviato con \textbf{Rp}.

\paragraph{Amministratore}
L'\amministratore\ è il responsabile di tutto ciò che riguarda l'ambiente di lavoro. Più in dettaglio, egli si occupa di:
\begin{itemize}
  \item Controllo dell'ambiente di lavoro;
  \item Gestione del versionamento della documentazione;
  \item Controllo delle versioni e delle configurazioni del prodotto;
  \item Risoluzione dei problemi legati alla gestione dei processi.
\end{itemize}
Può essere abbreviato con \textbf{Am}.

\paragraph{Progettista}
Il \progettista\ è il responsabile di tutto ciò che riguarda la progettazione.
Più in dettaglio, egli si occupa di:
\begin{itemize}
  \item Produrre una soluzione attuabile e robusta;
  \item Effettuare scelte progettuali volte a garantire la manutenibilità e la modularità del prodotto software.
\end{itemize}
Può essere abbreviato con \textbf{Pt}.

\paragraph{Analista}
L'\analista\ si occupa di tutto ciò che riguarda l'analisi del problema da affrontare. Le mansioni principali sono quelle di:
\begin{itemize}
  \item Studiare a fondo e capire le problematiche del prodotto da realizzare;
  \item Produrre una specifica di progetto comprensibile per il proponente, per il committente e per il Progettista.
\end{itemize}
Può essere abbreviato con \textbf{An}.

\paragraph{Verificatore}
Il \verificatore\ è responsabile di tutto ciò che riguarda l'attività di verifica.
Effettua la verifica dei documenti utilizzando gli strumenti e i metodi proposti nel
\pianodiqualificav.
Egli ha il compito di garantire la conformità rispetto le \normediprogettov dei documenti verificati.
Può essere abbreviato con \textbf{Ve}.

\paragraph{Programmatore}
Il \programmatore\ si occuperà di implementare le soluzioni, è quindi
responsabile dell'attività di codifica. In dettaglio, i suoi compiti sono:
\begin{itemize}
  \item Implementare le soluzioni descritte dal progettista in maniera
  rigorosa;
  \item Scrivere il codice rispettando le convenzioni prese nel presente documento;
  \item Implementare i test per il codice scritto da utilizzare per l'attività di verifica.
\end{itemize}
Può essere abbreviato con \textbf{Pm}.

\subsubsection{Gestione delle attività del progetto}
Le attività di progetto, verranno divise in unità logiche denominate \gl{task}. Un'attività sarà quindi rappresentata da una lista di task e ne verrà creata una per ogni attività da svolgere.
Le liste di task e gli stessi task vengono creati dal \responsabilediprogetto\ e sono assegnati ad un singolo o a più membri del gruppo, in quest'ultimo caso, essi potranno suddividerseli in task più ristretti ed assegnarseli tra loro.
Verrà affiancato un \verificatore\ ad ogni lista di task, esso sarà segnalato come partecipante e ci sarà un task dedicato alla verifica, che sarà dipendente dalla terminazione di tutti i task di quella lista.
Una volta che si ritiene il proprio task completato, si deve segnalarlo come completato nello strumento di gestione dei task.
Quando tutti i task di una lista saranno completati, il task di verifica verrà sbloccato e i \verificatori\ ed il \responsabilediprogetto\ riceveranno una notifica.
Terminati tutti i task di quella lista quel modulo può considerarsi concluso e verrà chiuso dal \responsabilediprogetto.
Ogni task dovrà integrare il numero di ore previste per quel lavoro.
A seguito del completamento di ogni task, dovrà essere noto quanto tempo è stato effettivamente utilizzato per completare quel task attraverso l'apposita funzione integrata nello strumento o come elemento della descrizione.
Lo strumento utilizzato per questa gestione di task sarà \gl{Teamwork} che verrà descritto nella sezione strumenti.
\subsection{Strumenti}

\subsubsection{Google Drive}
Drive è un servizio di \gl{cloud storage} offerto da Google e utilizzato dal team per condividere file e documenti che non necessitano di tracciamento come: guide, documenti informali e bozze.

\subsubsection{GitHub}
Lo strumento di controllo di versione, è fondamentale nello sviluppo di un software, specialmente se in gruppo, per evitare duplicazioni di codice o errori difficili da recuperare.
Per questo motivo, si è deciso di utilizzare uno di questi strumenti chiamato \gl{Git}, che traccia ogni modifica effettuata ai singoli file. La scelta è motivata da diversi fattori:
\begin{itemize}
\item Git permette di lavorare anche in assenza di connettività;
\item Era già famigliare a diversi componenti del gruppo ed è inoltre possibile reperire moltissima documentazione a riguardo per un rapido apprendimento;
\item Esistono molti servizi di \gl{repository} online ben sviluppati e gratuiti.
\end{itemize}
Il servizio scelto per il mantenimento dei dati è \gl{GitHub} che per progetti open \gl{source} offre un servizio di qualità e gratuito; è inoltre possibile integrarlo con diversi altri strumenti.
I \gl{branch} di Git saranno sempre almeno due:
\begin{itemize}
\item \textbf{Master}: dovrà contenere i \gl{commit} delle sole \gl{release} principali dopo l'approvazione del \responsabilediprogetto;
\item \textbf{Dev}: può essere utilizzato da tutti e i commit sono liberi.
\end{itemize}
Entrambi i branch della repository saranno sottoposti ad analisi statica e dinamica nel caso del codice e, se non conformi, verranno rigettati.
Per questa operazione chiamata \gl{integrazione continua} verrà utilizzato uno strumento centralizzato chiamato \gl{Jenkins}, che ogni qualvolta verrà effettuato un commit provvederà alla verifica.
L'utilizzo di Git per il versionamento del codice permette a più persone di lavorare contemporaneamente anche sullo stesso file. Per ottimizzare la produzione si è scelto di creare un nuovo branch ad ogni modifica rilevante al  codice. Una volta che l'incremento sarà stato completato si procederà al \gl{merge}. \\
Per evitare conflitti ed errori deve essere richiesto il merge tramite GitHub utilizzando le apposite \gl{pull request} assegnando la verifica ad un membro verificatore del team. Il verificatore assegnato alle modifiche dovrà procedere alla verifica e successivamente accettare il merge o riportare gli errori tramite Teamwork.

\subsubsection{Teamwork}
Per gestire nella maniera più opportuna e centralizzata la divisione del lavoro, si è scelto di utilizzare il sistema di pianificazione delle attività chiamato Teamwork.
Quest'ultimo è di supporto in diversi modi:
\begin{itemize}
\item Permette di suddividere i task in diversi sottogruppi;
\item Permette di gestire le \gl{milestone};
\item Permette di avere un calendario comune dove annotare gli impegni che vanno rispettati e gestirne i promemoria;
\item Aiuta nella pianificazione, grazie alla presenza di un Gantt chart;
\item Permette di registrare il tempo reale utilizzato per eseguire un determinato task.
\end{itemize}

\subsection{Lista di controllo}
Per controllare i documenti è stata adottata la tecnica del Walkthrough.
Durante i controlli gli errori più frequentemente trovati, sono stati:
\begin{itemize}
	\item Punteggiatura;
	\item Mancanza di termini nel glossario;
	\item Mancanza di lettere maiuscole all'inizio degli elementi delle liste;
	\item Link rotti o mal formattati.
\end{itemize}

\end{document}
