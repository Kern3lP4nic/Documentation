\documentclass[../PianoDiQualifica.tex]{subfiles}

\begin{document}
	\section{Qualità di prodotto}
	Oltre che nei processi, \kpanic\ si prefigge di mantenere una buona qualità del prodotto cercando di seguire al meglio lo standard di qualità ISO/IEC 9126. Con prodotto, il team intende l'insieme di documenti e del software realizzato.
	
		\paragraph{Misure}
			Ogni misura che viene attuata sui processi o sui prodotti deve essere rapportata in una scala. I valori di quest'ultima che si intendo perseguire sono riportati nel documento \pianodiqualificav. 
			
		\paragraph{Qualità del software}
			Assimilato lo standard ISO/IEC 9126, gli obiettivi di qualità riguardanti il software che il gruppo Kern3lP4nic desidera raggiungere nell'arco dell'intero progetto sono i seguenti:
			\begin{itemize}
				\item Funzionalità: il prodotto funzionerà sulla base dei requisiti indicati nel documento \analisideirequisitiv;
				\item Affidabilità: il prodotto dovrà essere robusto in presenza di eventuali situazioni di difficoltà;
				\item Usabilità: il prodotto dovrà risultare semplice nell'utilizzo per gli utenti a cui è destinato;
				\item Efficienza: il prodotto dovrà essere performante e rispondere alle richieste dell'utente utilizzando il minor numero di risorse e tempo;
				\item Manutenibilità: il prodotto dovrà essere verificato, stabile e testato ad ogni estensione o modifica;
				\item Portabilità: il prodotto si adatterà con facilità ad ogni trasferimento in diversi sistemi.
			\end{itemize}

			\subsection{Funzionalità}
				\subsubsection{Obiettivi di qualità}
				Il gruppo \kpanic\ si impegnerà affinché vi siano:
				\begin{itemize}
					\item \textbf{Adeguatezza}: le funzionalità fornite sono conformi alle aspettative;
					\item \textbf{Accuratezza}: il prodotto fornisce i risultati attesi, con il livello di dettaglio richiesto;
					\item \textbf{Sicurezza}: il prodotto protegge le informazioni e i dati da accessi e modifiche non autorizzate.
				\end{itemize}
			
			\subsubsection{Metriche}
				\paragraph{Completezza dell'implementazione funzionale}
				Indica la percentuale di requisiti funzionali coperti dall'implementazione.
				È possibile calcolarla con la formula seguente:
				\begin{equation*}
					C = (1 - \frac{N_{FM}}{N_{FI}}) * 100
				\end{equation*}
				Dove:
				\begin{itemize}
					\item \textbf{$N_{FM}$}: rappresenta il numero di funzionalità mancanti nell'implementazione;
					 \item \textbf{$N_{FI}$}: indica il numero di funzionalità individuate nell'attività di analisi.
				\end{itemize}

				\paragraph{Accuratezza rispetto alle attese}
				Serve per rilevare la percentuale di risultati che rispecchiano le attese.
				Si può calcolare con la seguente formula:
				\begin{equation*}
					A = (1 - \frac{N_{RD}}{N_{TE}})*100
				\end{equation*}
				Dove:
				\begin{itemize}
					\item \textbf{$N_{RD}$}: rappresenta il numero di test che producono risultati discordanti rispetto alle attese;
					\item \textbf{$N_{TE}$}: indica il numero di test eseguiti.
				\end{itemize}
				

				\paragraph{Controllo degli accessi}
				Permette di conoscere la percentuale di operazioni illegali non bloccate.
				É possibile calcolarla tramite la formula:
				\begin{equation*}
					I = \frac{N_{IE} * 100}{N_{II}}
				\end{equation*}
				Dove:
				\begin{itemize}
					\item \textbf{$N_{IE}$}: rappresenta il numero di operazioni illegali eseguibili dal test;
					\item \textbf{$N_{II}$}: indica il numero di operazioni illegali individuate.
				\end{itemize}

			\paragraph{Copertura requisiti desiderabili}
			Questa metrica permette di verificare in ogni momento lo stato dell'implementazione dei requisiti desiderabili. Essa controlla infatti il rapporto percentuale tra i requisiti desiderabili soddisfatti e il numero totale dei requisiti desiderabili richiesti.\\
			É possibile calcolarla tramite la formula:
				\begin{equation*}
					Copertura \ requisiti \ desiderabili = \frac{Numero \ di \ requisiti \ desiderabili \ soddisfatti \ * \ 100}{Numero \ totale \ di \ requisiti \ desiderabili}
				\end{equation*}
				
	
			\subsection{Affidabilità}
				\subsubsection{Obiettivi di qualità}
				Il prodotto software durante la sua esecuzione dovrà avere le seguenti caratteristiche:
				\begin{itemize}
					\item \textbf{Maturità}: evitare che si verifichino malfunzionamenti, operazioni illegali e restituzione di risultati errati (\gl{failure}) in seguito a \gl{fault};
					\item \textbf{Tolleranza agli errori}: nell'eventualità in cui si verifichino degli errori, a causa di guasti o per un uso scorretto dell'applicativo, questi dovranno essere gestiti in modo da non avere cali di prestazione.
				\end{itemize}

				\subsubsection{Metriche}
				\paragraph{Densità di failure}
				Questa metrica serve per indicare la percentuale di operazioni di testing che si sono concluse in failure. \\La formula per calcolare il valore è la seguente:
				\begin{equation*}
					F = \frac{N_{FR} * 100}{N_{TE}}
				\end{equation*}
				Dove:
				\begin{itemize}
					\item \textbf{$N_{FR}$}: indica il numero di failure rilevati durante la fase di test;
					\item \textbf{$N_{TE}$}: indica il numero di test eseguiti.
				\end{itemize}
				
				\paragraph{Blocco di operazioni non corrette}
				Serve per indicare la percentuale di funzionalità in grado di gestire correttamente i fault che potrebbero verificarsi.
				\begin{equation*}
					B = \frac{N_{FE}}{N_{ON}} * 100
				\end{equation*}
				Dove:
				\begin{itemize}
					\item \textbf{$N_{FE}$}: indica il numero di failure evitati durante la fase di test;
					\item \textbf{$N_{ON}$}: indica il numero di test eseguiti che prevedono l'esecuzione di operazioni non corrette e che possono causare failure.
				\end{itemize}
			
			
			\subsection{Usabilità}
				\subsubsection{Obiettivi di qualità}
				Il team punterà a garantire al prodotto, i seguenti obiettivi di usabilità:
				\begin{itemize}
					\item \textbf{Comprensibilità}: le funzionalità offerte saranno riconoscibili dall'utente, che dovrà essere in grado di comprenderne le modalità di utilizzo per poter raggiungere i risultati attesi;
					\item \textbf{Apprendibilità}: l'utente dovrà avere la possibilità di impararne l'utilizzo senza troppo impegno;
					\item \textbf{Operabilità}: le funzionalità presenti dovranno essere coerenti con le aspettative dell'utente;
					\item \textbf{Attrattiva}: dovrà essere piacevole durante il suo utilizzo.
				\end{itemize}

				\subsubsection{Metriche}
				\paragraph{Comprensibilità delle funzioni offerte}
				Questa metrica indica la percentuale di operazioni che l'utente riesce a comprendere in modo immediato, senza la consultazione di un manuale.\\La formula per calcolare il valore è la seguente:
				\begin{equation*}
					S = \frac{N_{FC} * 100}{N_{FO}}
				\end{equation*}
				Dove:
				\begin{itemize}
					\item \textbf{$N_{FC}$}: indica il numero di funzionalità comprese in modo immediato dall'utente durante l'attività di test del prodotto;
					\item \textbf{$N_{FO}$}: indica il numero di versioni di funzionalità offerte dal sistema.
				\end{itemize}
				
				\paragraph{Facilità di apprendimento delle funzionalità}
				Serve per indicare il tempo medio impiegato dall'utente per imparare ad usare correttamente una data funzionalità del software.\\
			Per il calcolo di questa metrica viene preso in considerazione il tempo, espresso in minuti, che l'utente impiega per apprendere il corretto funzionamento di una funzionalità offerta dal prodotto.
				
				\paragraph{Consistenza operazionale in uso}
				Indica la percentuale di funzionalità offerte all'utente che rispettano le sue aspettative. La misurazione viene effettuata con la seguente formula:
				\begin{equation*}
					C = (1 - \frac{N_{MFI}}{N_{MFO}}) * 100
				\end{equation*}
				Dove:
				\begin{itemize}
					\item \textbf{$N_{MFI}$}:  rappresenta il numero di funzionalità che non rispettano le aspettative dell'utente;
					\item \textbf{$	N_{MFO}$}:  indica le funzionalità offerte dal sistema.
				\end{itemize}
				
			\subsection{Efficienza}
				\subsubsection{Obiettivi di qualità}
				Il prodotto dovrà essere efficiente. In particolare:
				\begin{itemize}
					\item \textbf{Utilizzo delle risorse}: il software, nell'utilizzo delle sue funzionalità, dovrà utilizzare un appropriato numero e tipo di risorse.
				\end{itemize}
			
	
			\subsection{Manutenibilità}
				\subsubsection{Obiettivi di qualità}
				Per garantire il più possibile l'agevolezza delle operazioni di manutenzione verranno adottate le seguenti caratteristiche:
				\begin{itemize}
					\item \textbf{Analizzabilità}: deve essere permessa dal software una rapida identificazione delle possibili cause di errori e malfunzionamenti;
					\item \textbf{Modificabilità}: devono essere permessi, dal software, dei possibili cambianti in alcune sue parti;
					\item \textbf{Stabilità}: dopo delle modifiche al software non devono insorgere degli effetti indesiderati;
					\item \textbf{Testabilità}: il software, dopo eventuali modifiche, deve poter essere facilmente testato.
				\end{itemize}
				
				\subsubsection{Metriche}
					\paragraph{Capacità di analisi di failure}
					È la percentuale di failure registrate delle quali sono state individuate le cause. Si calcola con la seguente formula:
					\begin{equation*}
						I = (\frac{N_{FI}}{N_{FR}}) * 100
					\end{equation*}
					Dove:
					\begin{itemize}
						\item \textbf{$	N_{FI}$}: rappresenta il numero di failure di cui sono state individuate le cause;
						\item \textbf{$N_{FR}$}: indica il numero di failure rilevate.
					\end{itemize}

				\paragraph{Impatto delle modifiche}
					È la percentuale di modifiche effettuate in risposta a failure che hanno portato all'introduzione di nuove failure in altre componenti del sistema. Si calcola con la seguente formula:
					\begin{equation*}
						I = (\frac{N_{FRF}}{N_{FR}}) * 100
					\end{equation*}
					Dove:
					\begin{itemize}
						\item \textbf{$	N_{FRF}$}: rappresenta il numero di failure risolte, ma che hanno introdotto nuove failure;
						\item \textbf{$N_{RF}$}: indica il numero di failure risolte.
					\end{itemize}


			\subsection{Portabilità}
				\subsubsection{Obiettivi di qualità}
				Il gruppo agevolerà la portabilità del prodotto software adottando quanto segue:
				\begin{itemize}
					\item \textbf{Adattabilità}: il prodotto dovrà adattarsi a tutti gli ambienti di lavoro nei quali è stato previsto un suo utilizzo, senza la necessità di dover apportare modifiche allo stesso;
					\item \textbf{Sostituibilità}: il prodotto deve poter sostituire un altro software che ha lo stesso scopo e lavora nel medesimo ambiente.
				\end{itemize}

				\subsubsection{Metriche}
					\paragraph{Versioni dei browser supportate}
					Questa metrica serve per indicare la percentuale di versioni di \gl{browser} attualmente supportate, fra quelle individuate dai requisiti. \\La formula per calcolarne il valore è la seguente:
					\begin{equation*}
						S = \frac{N_{VS} * 100}{N_{VS}}
					\end{equation*}
					Dove:
					\begin{itemize}
						\item \textbf{$N_{VS}$}: rappresenta il numero di versioni di browser supportare dal prodotto;
						\item \textbf{$N_{TE}$}: indica il numero di versioni di browser che devono essere supportare dal prodotto.
					\end{itemize}
					
					\paragraph{Inclusione di funzionalità da altri prodotti}
					Questa metrica serve per indicare la percentuale di funzionalità del software utilizzato in precedenza dall'utente che produce risultati simili a quelli ottenuti dal prodotto da sviluppare. La formula per calcolare questo valore è la seguente:
					\begin{equation*}
						I = \frac{N_{FPA}}{N_{FPP}} * 100
					\end{equation*}
					Dove:
					\begin{itemize}
						\item \textbf{$N_{FPA}$}: indica il numero di funzionalità del software utilizzato in precedenza dall'utente che produce risultati simili a quelli ottenuti dal prodotto in sviluppo;
						\item \textbf{$N_{FPP}$}: indica il numero di funzionalità offerte dal software utilizzato in precedenza dall'utente.
					\end{itemize}		
\end{document}
