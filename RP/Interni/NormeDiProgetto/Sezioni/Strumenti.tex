\documentclass[../NormeDiProgetto.tex]{subfiles}
\begin{document}

\section{Strumenti}
	\subsection{Strumenti per documenti}
    	\subsubsection{Texmaker}
            L'editor scelto dal team per sviluppare i documenti in \LaTeX\ è \gl{Texmaker}. Texmaker è un editor gratuito, moderno e multi-piattaforma per \gl{Linux}, sistemi \gl{MacOS} e \gl{Microsoft Windows} che integra molti strumenti utili per sviluppare documenti in \LaTeX. Texmaker include il supporto \gl{Unicode}, il controllo ortografico, il completamento automatico, il raggruppamento del codice e un visore incorporato \gl{PDF} con il supporto \gl{SyncTex} e modalità di visualizzazione continua. Texmaker è uno strumento facile da usare e da configurare.

            \subsubsection{Aspell}
            \gl{Aspell} è una utility per il controllo del testo. Permette di controllare automaticamente l'ortografia di un testo \LaTeX\ tramite un dizionario di lingua italiana.
            
            \subsubsection{GloMaker}
            GloMaker è un applicativo realizzato dal team per l'inserimento automatico dei termini, prelevati dai documenti, che vanno inseriti nel \glossario.


	\subsection{Strumenti per sviluppo}
			\subsubsection{Trender}
            Per mantenere univoco il modello con il quale i requisiti e i casi d'uso vengono descritti, si è deciso di utilizzare un software con \gl{licenza MIT} chiamato \gl{Trender}.
            Quest'ultimo è un software di supporto sia per la parte di analisi del problema, sia nella parte di progettazione logica.
            Quest'ultimo permette di tracciare e descrivere:
            \begin{itemize}
                  \item \gl{Casi d'uso};
                  \item \gl{Requisiti};
                  \item \gl{Attori};
                  \item \gl{Classi};
                  \item \gl{Package}.
            \end{itemize}
            e di legarli tra loro. In seguito può essere effettuata la stampa in \gl{\LaTeX} di tutti gli elementi sopra descritti con le relative tabelle che ne mostrano i legami. É possibile inoltre tracciare i verbali in modo da averne un'indicizzazione conforme.

            \subsubsection{Astah}
            \gl{Astah} è l'editor scelto dal team per la modellazione nel linguaggio UML. Astah è un editor gratuito, scaricabile online sia per Microsoft Windows, sia per Linux che per MacOS. È un editor abbastanza semplice e user-fiendly. Infatti ad un primo utilizzo, è facile capire come muoversi nell'ambiente di sviluppo, come aggiungere classi, diagrammi e relazioni. Strumento molto utile che l'editor mette a disposizione è quello di poter esportare il diagramma realizzato in un file immagine per una rapida visualizzazione. Astah supporta la funzionalità per la codifica dei diagrammi delle classi in un linguaggio di programmazione. Dunque, è possibile disegnare un diagramma delle classi, generare il codice e poi sistemare a mano le varie molteplicità delle relazioni.

            \subsubsection{Microsoft Excel}
            \gl{Excel} è un programma di fogli di calcolo che permette di gestire dati di vario genere, eseguire operazioni su di essi e mostrarli attraverso grafici. L'utilizzo di Excel sarà limitato alla produzione di grafici per la documentazione.
            
            \subsubsection{Gantt Chart}
            \gl{Gantt Chart} o diagramma di Gantt è uno strumento di supporto alla gestione dei progetti che permette di visualizzare in un solo schema l'insieme dei task in cui un progetto è suddiviso e il tempo previsto e impiegato nella loro realizzazione. L'utilizzo di questo tipo di diagramma facilita la suddivisione dei task tra i membri del team e permette di identificare più facilmente l'andamento del progetto e il soddisfacimento delle date di consegna dei contenuti.
            
            \subsubsection{GanttProject}
            \gl{GanttProject} è un software di modellazione di diagrammi di Gantt open source, rilasciato sotto licenza \gl{GNU GPL 3}. Questo software viene utilizzato per l'organizzazione dei task all'interno del team e la previsione dei tempi di sviluppo.

            \subsubsection{Google Chrome}
            \gl{Google Chrome} è il browser sviluppato dalla software house Google ed è tuttora uno di più affidabili ed efficienti browser web, inoltre è attualmente il browser che supporta al meglio gli standard web HTML5, \gl{CSS3} e JavaScript. Fornisce inoltre ottimi strumenti integrati per lo sviluppo.

            \subsubsection{Swagger}
            \gl{Swagger} è un insieme di specifiche e di strumenti che mirano a semplificare e standardizzare i processi di documentazione di API per servizi web \gl{RESTful}. Il cuore di Swagger consiste in un file testuale dove sono descritte tutte le funzionalità di un'applicazione web e i dettagli di input e output in un formato studiato per essere interpretabile correttamente.

            \subsubsection{NPM}
            \gl{NPM} è il principale software utilizzato per maneggiare i moduli di Node.js e consente di condividere il codice per problemi tipici tra gli sviluppatori JavaScript. NPM suddivide il codice in package o moduli e si dimostra utilissimo poiché ne facilita il riuso e la manutenzione.

            \subsubsection{Amazon DynamoDB}
            \gl{Amazon DynamoDB} è un servizio che fa parte del pacchetto di Amazon Web Services offerti da Amazon. DynamoDB è una tipologia di database che supporta lo storage, le query e l'aggiornamento di documenti. La funzionalità che suscita il nostro interesse è la possibilità di creare applicazioni, attraverso l'SDK AWS, che memorizzano documenti JSON direttamente nelle tabelle Amazon DynamoDB. 

            \subsubsection{AWS APIGateway}
            \gl{AWS APIGateway} è un servizio che fa parte del pacchetto di Amazon Web Services offerti da Amazon. APIGateway è un'interfaccia per applicazioni di tipo Rest e permette la semplice interazione con Amazon Lambda. 
            
	\subsection{Strumenti per test}
            \subsubsection{AWS CodePipeline}
            \gl{AWS CodePipeline} è un servizio di integrazione continua e distribuzione continua per aggiornare applicazione e infrastruttura in modo rapido. CodePipeline crea, esegue il testing e distribuisce il codice ogni volta che viene modificato, in base a modelli e processi personalizzati configurati.
            
            \subsubsection{JSCover}
            JSCover è un applicativo che permette di svolgere branch coverage sul codice che verrà prodotto durante le sessioni di codifica.
            
      \subsection{Tecnologie utilizzate}
            \subsubsection{JavaScript}
            JavaScript è un linguaggio di scripting orientato agli oggetti e agli eventi, comunemente utilizzato nella programmazione Web lato client per aggiungere funzionalità dinamiche alle pagine.
            
            \subsubsection{Node.js}
            \gl{Node.js} è un runtime JavaScript costruito sul motore JavaScript V8 di Google Chrome. Node.js usa un modello I/O non bloccante e ad eventi, che lo rende efficace ed efficiente.
            
            \subsubsection{HTML5}
            \gl{HTML5} è l'ultima versione standard di \gl{HTML} ed è ormai supportata da tutti i maggiori browser. Dato il supporto, e le funzionalità che offre è adatto alla struttura dell'applicativo.
            
            \subsubsection{CSS3}
            \gl{CSS} necessario per separare i contenuti delle pagine HTML dalla loro formattazione e permettere una programmazione più chiara e facile da utilizzare, sia per gli autori delle pagine stesse sia per gli utenti, garantendo contemporaneamente anche il riutilizzo di codice ed una sua più facile manutenzione.

            \subsubsection{AWS Lambda}
            \gl{AWS Lambda} è un servizio di elaborazione serverless che esegue il codice in risposta a determinati eventi su un'infrastruttura di calcolo ad alta disponibilità e amministra automaticamente le risorse di elaborazione in uso.

            \subsubsection{TypeScript}
            \gl{TypeScript} è un linguaggio di programmazione libero ed Open Source sviluppato da Microsoft e che estende la base esistente JavaScript aggiungendo:
                  \begin{itemize}
                        \item Firma digitale;
                        \item Classi;
                        \item Interfacce;
                        \item Moduli;
                        \item Operatore =>;
                        \item Tipi di dato;
                        \item Tipo Enum;
                        \item Mixin tra classi.
                  \end{itemize}

      \subsection{Librerie utilizzate}

            \subsubsection{Angular 2}
            \gl{Angular 2} è la versione più recente del framework AngularJS per lo sviluppo di applicazioni web \gl{single page} in JavaScript supportato da Google. Questo strumento permette la rapida gestione del \gl{Two Way Data binding} mediante l'utilizzo del pattern \gl{MVC}.

            \subsubsection{Total.js}
            \gl{Total.js} è un framework per Node.js che permette una semplice gestione dei path di richieste per applicazione Rest, e che è pienamente compatibile con Angular 2.

            \subsubsection{AWS SKD per Node.js}
            \gl{AWS SKD per Node.js} sono un'insieme di strumenti atti alla realizzazione di codice per Node.js nell'ambiente di Amazon Web Services.

\end{document}