\newglossaryentry{AngularJS} {
	name=AngularJS,
	description={\invisiblesection{AngularJS}Framework JavaScript per lo sviluppo di applicazioni Web client side, rilasciato da Google in versione stabile nel 2012}
}
\newglossaryentry{SWEDesigner} {
	name=SWEDesigner,
	description={\invisiblesection{SWEDesigner}Riferimento al nome del progetto del capitolato C6. Per ulteriori info si rimanda alla \href{http://www.math.unipd.it/~tullio/IS-1/2016/Progetto/C6.pdf}{documentazione ufficiale}}
}
\newglossaryentry{PostgreSQL} {
	name=PostgreSQL,
	description={\invisiblesection{PostgreSQL}Sistema di gestione di basi di dati che consente la manipolazione e l'interrogazione efficiente di collezioni di dati strutturati}
}
\newglossaryentry{Fault} {
	name=Fault,
	description={\invisiblesection{Fault}Un processo o un dato non corretto all'interno di un software}
}
\newglossaryentry{GanttProject} {
	name=GanttProject,
	description={\invisiblesection{GanttProject}Software gratuito scritto in Java, multi piattaforma, che permette la gestione di progetti aiutando il team nel pianificare ed organizzare attività e gestire le risorse}
}
\newglossaryentry{LaTeX} {
	name=LaTeX,
	description={\invisiblesection{LaTeX}Linguaggio di markup usato per la preparazione di documenti testuali basato sul programma di composizione tipografica TEX}
}
\newglossaryentry{Proponente} {
	name=Proponente,
	description={\invisiblesection{Proponente}Chi propone un qualcosa. Nel corso di Ingegneria del Software, rappresenta l'azienda che propone il capitolato}
}
\newglossaryentry{Design} {
	name=Design,
	description={\invisiblesection{Design}In ambito scientifico, è un processo che a partire da determinate norme tecniche permette la realizzazione di un qualsiasi oggetto complesso, sia software che hardware}
}
\newglossaryentry{SiriKit} {
	name=SiriKit,
	description={\invisiblesection{SiriKit}Framework di proprietà Apple che permette di estendere le funzionalità delle proprie applicazioni implementando l'assistente virtuale Siri}
}
\newglossaryentry{eBread} {
	name=eBread,
	description={\invisiblesection{eBread}Applicazione di lettura per dislessici, riferita al nome del progetto del capitolato C4. Per ulteriori info si rimanda alla \href{http://www.math.unipd.it/~tullio/IS-1/2016/Progetto/C4.pdf}{documentazione ufficiale}}
}
\newglossaryentry{Spice} {
	name=Spice,
	description={\invisiblesection{Spice}Strumento per valutare la qualità dei processi di un software, nato nel 1992 in contrapposizione a CMM}
}
\newglossaryentry{Ciclo di vita del software} {
	name=Ciclo di vita del software,
	description={\invisiblesection{Ciclo di vita del software}Tutti gli stati che il prodotto assume dal concepimento al ritiro}
}
\newglossaryentry{Rocket.Chat} {
	name=Rocket.Chat,
	description={\invisiblesection{Rocket.Chat}Servizio di chat gratuito, open source, che permette di scrivere messaggi tra più persone o effettuare chiamate video}
}
\newglossaryentry{API key} {
	name=API key,
	description={\invisiblesection{API key}Chiave di sicurezza segreta necessaria ad autenticarsi presso un server in modo da accedere ai servizi offerti da un set di API}
}
\newglossaryentry{SyncTex} {
	name=SyncTex,
	description={\invisiblesection{SyncTex}Utility che permette di sincronizzare il codice sorgente di un documento \LaTeX\ con il relativo output PDF}
}
\newglossaryentry{DeGeOP} {
	name=DeGeOP,
	description={\invisiblesection{DeGeOP}A Designer and Geo-localizer Web App for Organizational Plants, riferita al nome del capitolato C3. Per ulteriori info si rimanda alla \href{http://www.math.unipd.it/~tullio/IS-1/2016/Progetto/C3.pdf}{documentazione ufficiale}}
}
\newglossaryentry{RR} {
	name=RR,
	description={\invisiblesection{RR}Acronimo di Revisione dei Requisiti, fa parte delle revisioni di avanzamento di progetto utilizzate nel corso di Ingegneria del Software}
}
\newglossaryentry{RP} {
	name=RP,
	description={\invisiblesection{RP}Acronimo di Revisione di Progettazione, fa parte delle revisioni di avanzamento di progetto utilizzate nel corso di Ingegneria del Software}
}
\newglossaryentry{RQ} {
	name=RQ,
	description={\invisiblesection{RQ}Acronimo di Revisione di Qualifica, fa parte delle revisioni di avanzamento di progetto utilizzate nel corso di Ingegneria del Software}
}
\newglossaryentry{RA} {
	name=RA,
	description={\invisiblesection{RA}Acronimo di Revisione di Accettazione, fa parte delle revisioni di avanzamento di progetto utilizzate nel corso di Ingegneria del Software}
}
\newglossaryentry{Libreria} {
	name=Libreria,
	description={\invisiblesection{Libreria}Collezione di risorse utilizzate da un software per aumentarne le potenzialità}
}
\newglossaryentry{Logger} {
	name=Logger,
	description={\invisiblesection{Logger}Software che tiene traccia scrivendo in un file di testo tutti gli eventi accaduti dal primo avvio}
}
\newglossaryentry{Font} {
	name=Font,
	description={\invisiblesection{Font}Insieme di caratteri tipografici accomunati da un certo stile grafico}
}
\newglossaryentry{NoSQL} {
	name=NoSQL,
	description={\invisiblesection{NoSQL}Dati modellati in base al significato, e non in base alla relazione fra essi come avviene nei database relazionali}
}
\newglossaryentry{Bug} {  
  	name=Bug, 
  	description={\invisiblesection{Bug}In informatica, identifica un errore nella scrittura del codice sorgente di un programma software. Meno comunemente, il termine bug può indicare un difetto di progettazione in un componente hardware, che ne causa un comportamento imprevisto o comunque diverso da quello specificato dal produttore}  
}  
\newglossaryentry{Swagger} {  
  	name=Swagger,  
  	description={\invisiblesection{Swagger}Framework Open Source caratterizzato da un vasto ecosistema di strumenti che consentono di progettare, costruire, documentare ed utilizzare le vostre API per un'architettura di tipo REST}  
}  
\newglossaryentry{EPub} {  
  	name=EPub,   
  	description={\invisiblesection{EPub}Abbreviazione di \textit{Eletronic Publication}, indicato anche come EPub, epub, o EPUB è uno standard aperto specifico per la pubblicazione di libri digitali come gli \textit{eBook} basato sul linguaggio XML}  
}  
\newglossaryentry{Amministratore} {  
	name=Amministratore,   
	description={\invisiblesection{Amministratore}Colui che, all'interno del gruppo di sviluppo del progetto didattico, è responsabile dell'efficienza e dell'operatività dell'ambiente di sviluppo, della redazione e attuazione di piani e procedure di gestione per la qualità, di controllo di versioni e configurazioni del prodotto, di gestione ed archiviazione della documentazione di progetto. Egli inoltre collabora alla redazione del \textit{Piano di Progetto} e redige le \textit{Norme di Progetto} per conto del Responsabile}  
}
\newglossaryentry{Microservizio} {
	name=Microservizio,
	description={\invisiblesection{Microservizio}Piccola applicazione indipendente dalle altre, che insieme ad altre fornisce uno specifico servizio}
}
\newglossaryentry{Java} {
	name=Java,
	description={\invisiblesection{Java}Linguaggio di programmazione orientato agli oggetti, specificatamente progettato per essere il più possibile indipendente dalla piattaforma di esecuzione}
}
\newglossaryentry{Diagramma di attivita} {
	name=Diagramma di attività,
	description={\invisiblesection{Diagramma di attivita}Diagramma definito all'interno dello \textit{Unified Modeling Language} (UML) che definisce le attività da svolgere per realizzare una data funzionalità. Più in dettaglio, un diagramma di attività definisce dei flussi di compiti, i termini di relazioni tra di essi, i responsabili e i punti di decisione}
}
\newglossaryentry{Express} {
	name=Express,
	description={\invisiblesection{Express}Framework per applicazioni web NodeJS flessibile e leggero che fornisce una serie di funzioni avanzate per le applicazioni web e per dispositivi mobili}
}
\newglossaryentry{Classe} {
	name=Classe,
	description={\invisiblesection{Classe}Nella programmazione orientata agli oggetti è un costrutto di un linguaggio di programmazione usato come modello per creare oggetti. Una classe è identificabile come un tipo di dato astratto che può rappresentare molteplici entità ed è quindi l'astrazione di un concetto, implementata in un software}
}
\newglossaryentry{Interlocutore} {
	name=Interlocutore,
	description={\invisiblesection{Interlocutore}All'interno del progetto didattico si riferisce al membro del personale dell'azienda \textit{zero12} che l'ospite desidera incontrare}
}
\newglossaryentry{Microsoft Windows} {
	name=Microsoft Windows,
	description={\invisiblesection{Microsoft Windows}Famiglia di ambienti operativi e sistemi operativi ideati e prodotti dalla \textit{Microsoft} dedicati ai personal computer, alle workstation, ai server e agli smartphone. Il sistema operativo è così chiamato per via della sua interfaccia di visualizzazione a finestre}
}
\newglossaryentry{Repository} {
	name=Repository,
	description={\invisiblesection{Repository}Locazione di memoria nella quale i pacchetti necessari a comporre un software o parte di esso, possono essere recuperati e installati in un altro ambiente di lavoro. Il repository utilizzato dal gruppo \textit{Kern3lP4nic} è fornito dalla piattaforma GitHub}
}
\newglossaryentry{Google Chrome DevTools} {
	name=Google Chrome DevTools,
	description={\invisiblesection{Google Chrome DevTools}Insieme di strumenti web per i test ed il debug di codice integrati nel browser Google Chrome. Questi strumenti offrono agli sviluppatori web accesso in profondità ai meccanismi interni del browser e alla loro applicazione nel web}
}
\newglossaryentry{Merge} {
	name=Merge,
	description={\invisiblesection{Merge}Nel controllo distribuito di versione, l'operazione di merge (chiamata anche integrazione) è una operazione fondamentale che riconcilia più modifiche apportate ad una collezione di file con controllo di versione}
}
\newglossaryentry{Applicazione web} {
	name=Applicazione web,
	description={\invisiblesection{Applicazione web}Applicazione accessibile/fruibile via web per mezzo di un network, all'interno di un sistema informatico o attraverso la Rete Internet}
}
\newglossaryentry{E-commerce} {
	name=E-commerce,
	description={\invisiblesection{E-commerce}Acquisto di beni e/o servizi attraverso il World Wide Web ricorrendo a server sicuri (caratterizzati dall'indirizzo HTTPS, un apposito protocollo che crittografa i dati sensibili dei clienti contenuti nell'ordine di acquisto allo scopo di tutelare il consumatore), con servizi di pagamento in linea}
}
\newglossaryentry{Pull request} {
	name=Pull request,
	description={\invisiblesection{Pull request}All'interno del controllo distribuito di versione è una azione/richiesta che consente di comunicare le proprie modifiche all'interno di una repository di lavoro}
}
\newglossaryentry{Sintesi vocale} {
	name=Sintesi vocale,
	description={\invisiblesection{Sintesi vocale}Tecnica per la riproduzione artificiale della voce umana. Un sistema usato per questo scopo è detto sintetizzatore vocale e può essere realizzato tramite software o via hardware. I sistemi di sintesi vocale sono noti anche come sistemi \textit{text-to-speech} (TTS) per la loro possibilità di convertire il testo in parlato}
}
\newglossaryentry{Excel} {
	name=Excel,
	description={\invisiblesection{Excel}Programma prodotto da Microsoft, dedicato alla creazione, alla produzione ed alla gestione dei fogli elettronici tra i più diffusi ed utilizzati. Oltre alle funzionalità classiche del foglio di calcolo fornisce anche utili strumenti di impaginazione, grafica e testi}
}
\newglossaryentry{Google Chrome} {
	name=Google Chrome,
	description={\invisiblesection{Google Chrome}Navigatore web sviluppato da Google nel 2008, basato a partire dalla versione 28, sul motore di esecuzione Blink e successore del web browser open source chiamato Chromium}
}
\newglossaryentry{SCSS} {
	name=SCSS,
	description={\invisiblesection{SCSS}Estensione per i tipi di file CSS3}
}
\newglossaryentry{Procedure} {
	name=Procedure,
	description={\invisiblesection{Procedure}Serie di azioni atte a perseguire un determinato scopo}
}
\newglossaryentry{Unicode} {
	name=Unicode,
	description={\invisiblesection{Unicode}Unicode è un sistema di codifica che assegna un numero univoco ad ogni carattere usato per la scrittura di testi, in maniera indipendente dalla lingua, dalla piattaforma informatica e dal programma utilizzato}
}
\newglossaryentry{Integrazione continua} {
	name=Integrazione continua,
	description={\invisiblesection{Integrazione continua}Nell'ingegneria del software, l'integrazione continua (Continuous Integration in inglese, spesso abbreviato in CI) è una pratica che si applica in contesti in cui lo sviluppo del software avviene attraverso un sistema di versionamento. Consiste nell'allineamento frequente dai vari ambienti di lavoro degli sviluppatori verso il principale ambiente condiviso (master)}
}
\newglossaryentry{Modello Incrementale} {
	name=Modello Incrementale,
	description={\invisiblesection{Modello Incrementale}Procedura che procura un'aggiunta ad un impianto base, con il risultato di un avvicinamento alla conclusione del progetto}
}
\newglossaryentry{Milestone} {
	name=Milestone,
	description={\invisiblesection{Milestone}Termine inglese che letteralmente significa pietra miliare. Tipicamente utilizzato nella pianificazione e gestione di progetti complessi per indicare il raggiungimento di obiettivi stabiliti in fase di definizione del progetto stesso}
}
\newglossaryentry{Monolith} {
	name=Monolith,
	description={\invisiblesection{Monolith}Riferimento al nome del progetto del capitolato C5. Per ulteriori info si rimanda alla \href{http://www.math.unipd.it/~tullio/IS-1/2016/Progetto/C5.pdf}{documentazione ufficiale}}
}
\newglossaryentry{Miglioramento Continuo} {
	name=Miglioramento Continuo,
	description={\invisiblesection{Miglioramento Continuo}Il modello di generazione incrementale è un metodo di sviluppo software in cui il prodotto è progettato, implementato e testato in maniera incrementale fino al raggiungimento del prodotto finito. Essa coinvolge sia lo sviluppo che la manutenzione. Il prodotto risulta finito quando soddisfa tutti i requisiti. Questo modello combina gli elementi del modello a cascata con la filosofia iterativa di prototipazione}
}
\newglossaryentry{Inspection} {
	name=Inspection,
	description={\invisiblesection{Inspection}Metodo di verifica con l'obiettivo di identificare difetti da sottoporre a correzione. La fase del controllo sarà effettuata da un verificatore, che parte per parte analizzerà il lavoro svolto dall'autore del prodotto. Successivamente indicherà i difetti trovati all'autore in modo da fargli effettuare le correzioni necessarie. Queste fase di verifica e correzione può essere iterata più volte. 
	Il prodotto redatto passerà alla fase di controllo solo dopo aver superato determinati criteri in modo da impedire controlli di prodotti ancora incompleti}
}
\newglossaryentry{CSS3} {
	name=CSS3,
	description={\invisiblesection{CSS3}Aggiornamento di CSS che introduce nuove funzionalità ed effetti grafici} 
}
\newglossaryentry{Linux} {
	name=Linux,
	description={\invisiblesection{Linux}Linux è una famiglia di sistemi operativi di tipo Unix-like, rilasciati sotto varie possibili distribuzioni, aventi la caratteristica comune di utilizzare come nucleo il kernel Linux}
}
\newglossaryentry{UML} {
	name=UML,
	description={\invisiblesection{UML}In ingegneria del software, UML (Unified Modeling Language, "linguaggio di modellizzazione unificato") è un linguaggio di modellizzazione e specifica basato sul paradigma orientato agli oggetti}
}
\newglossaryentry{MongoDB} {
	name=MongoDB,
	description={\invisiblesection{MongoDB}DBMS non relazionale, orientato ai documenti}
}
\newglossaryentry{Tomcat} {
	name=Tomcat,
	description={\invisiblesection{Tomcat}Apache Tomcat (o semplicemente Tomcat) è un application server nella forma di contenitore servlet open source sviluppato dalla Apache Software Foundation. Implementa le specifiche JavaServer Pages (JSP) e Servlet, fornendo quindi una piattaforma software per l'esecuzione di applicazioni Web sviluppate in linguaggio Java. La sua distribuzione standard include anche le funzionalità di web server tradizionale, che corrispondono al prodotto Apache}
}
\newglossaryentry{AtAVi} {
	name=AtAVi,
	description={\invisiblesection{AtAVi}Acronimo di Accoglienza tramite Assistente Virtuale, riferito al nome dato al capitolato C2. Per ulteriori info si rimanda alla \href{http://www.math.unipd.it/~tullio/IS-1/2016/Progetto/C2.pdf}{documentazione ufficiale}}
}
\newglossaryentry{GitHub} {
	name=GitHub,
	description={\invisiblesection{GitHub}Servizio di hosting per progetti software. Il nome deriva dal fatto che GitHub è un servizio sostitutivo del software dell'omonimo strumento di controllo versione distribuito, Git}
}
\newglossaryentry{Driver} {
	name=Driver,
	description={\invisiblesection{Driver}Un driver, in informatica, indica l'insieme di procedure, spesso scritte in assembly, che permette ad un sistema operativo di pilotare un dispositivo hardware}
}
\newglossaryentry{PDF} {
	name=PDF,
	description={\invisiblesection{PDF}Acronimo di Portable Document Format. È un formato di file sviluppato da Adobe per creare dei documenti che consentissero di essere visualizzati allo stesso modo su diversi dispositivi indipendentemente dalle componenti hardware e software presenti sul dispositivo}
}
\newglossaryentry{Branch} {
	name=Branch,
	description={\invisiblesection{Branch}Un punto che rappresenta un potenziale cambiamento nel flusso del progetto. Viene scelto un percorso fra diversi percorsi possibili. Quando si lavora ad un progetto, si possono lavorare contemporaneamente a diversi aspetti dello stesso in contemporanea, i quali poi, potenzialmente, verranno unificati}
}
\newglossaryentry{Bootstrap} {
	name=Bootstrap,
	description={\invisiblesection{Bootstrap}Bootstrap è una raccolta di strumenti liberi per la creazione di siti e applicazioni per il Web. Essa contiene modelli di progettazione basati su HTML e CSS, sia per la tipografia, che per le varie componenti dell'interfaccia, come moduli, pulsanti e navigazione, così come alcune estensioni opzionali di JavaScript}
}
\newglossaryentry{Licenza MIT} {
	name=Licenza MIT,
	description={\invisiblesection{Licenza MIT}La Licenza MIT  o Licenza Expat è una licenza di software libero creata dal Massachusetts Institute of Technology. È una licenza permissiva, cioè permette il riutilizzo del software proprietario sotto la condizione che la licenza sia distribuita con tale software. È anche una licenza GPL-compatibile, cioè la GPL permette di combinare e ridistribuire tale software con altro che utilizza la Licenza MIT}
}
\newglossaryentry{Requisiti} {
	name=Requisiti,
	description={\invisiblesection{Requisiti}Descrizione dei servizi che un sistema software deve fornire, insieme ai vincoli da rispettare sia in fase di sviluppo che durante la fase di operatività del software}
}
\newglossaryentry{AWS} {
	name=AWS,
	description={\invisiblesection{AWS}AWS, acronimo Amazon Web Services, è una piattaforma completa, che eroga servizi software forniti da Amazon.com}
}
\newglossaryentry{Diagrammi di sequenza} {
	name=Diagrammi di sequenza,
	description={\invisiblesection{Diagrammi di sequenza}È un diagramma previsto dall'UML utilizzato per descrivere uno scenario}
}
\newglossaryentry{Assistente virtuale} {
	name=Assistente virtuale,
	description={\invisiblesection{Assistente virtuale}Il termine viene spesso utilizzato per indicare programmi che interpretano il linguaggio naturale (Natural Language Processing). Il processo di elaborazione viene suddiviso in fasi diverse: analisi lessicale, analisi grammaticale, analisi sintattica e analisi semantica}
}
\newglossaryentry{Report} {
	name=Report,
	description={\invisiblesection{Report}Relazione, per lo più particolareggiata, intesa a informare il pubblico o un organo superiore}
}
\newglossaryentry{Stub} {
	name=Stub,
	description={\invisiblesection{Stub}Tipo di test che fornisce risposte preconfezionate alle chiamate effettuate durante le prove, generalmente non risponde a nulla che stia fuori da ciò per cui è stato programmato}
}
\newglossaryentry{Server-side} {
	name=Server-side,
	description={\invisiblesection{Server-side}L'espressione lato server (server-side in inglese) fa riferimento a operazioni compiute dal server in un ambito client-server contrapponendosi a tutto ciò che viene eseguito sul client (lato client)}
}
\newglossaryentry{Aspell} {
	name=Aspell,
	description={\invisiblesection{Aspell}GNU Aspell è un correttore ortografico libero e open source; può essere usato sia come libreria da altri programmi, sia come programma a sé stante. È anche in grado di suggerire il termine più corretto.\url{http://aspell.net}}
}
\newglossaryentry{PascalCase} {
	name=PascalCase,
	description={\invisiblesection{PascalCase}In programmazione, PascalCase denota la pratica di scrivere parole composte o frasi in modo tale che la prima lettera di ciascuna parola concatenata è maiuscola. Nessun altro simbolo è usato per separare}
}
\newglossaryentry{Editor} {
	name=Editor,
	description={\invisiblesection{Editor}Un editor è un programma di videoscrittura: facilita la scrittura del testo e, in base al contesto, può fornire suggerimenti e aiuti controlli sul testo}
}
\newglossaryentry{APIM} {
	name=APIM,
	description={\invisiblesection{APIM}API Market è il progetto cui ItalianaSoftware richiede la realizzazione nel capitolato C2}
}
\newglossaryentry{NodeJS} {
	name=NodeJS,
	description={\invisiblesection{NodeJS}È un ambiente JavaScript multipiattaforma ed open-source per lo sviluppo di una grande varietà di strumenti e applicazioni}
}
\newglossaryentry{Template} {
	name=Template,
	description={\invisiblesection{Template}È un modello predefinito che consente di creare o inserire contenuti di diverso tipo in un documento o in una pagina web}
}
\newglossaryentry{JavaScript} {
	name=JavaScript,
	description={\invisiblesection{JavaScript}JavaScript è un linguaggio di scripting orientato agli oggetti e agli eventi, comunemente utilizzato nella programmazione Web lato client per la creazione di siti e applicazioni web. È stato standardizzato con il nome ECMAScript, ed è alla base di molti framework moderni, tra i quali AngularJS e Ionic}
}
\newglossaryentry{Base di dati} {
	name=Base di dati,
	description={\invisiblesection{Base di dati}Indica un insieme di dati, omogeneo per contenuti e per formato, memorizzati in un elaboratore elettronico e interrogabili via terminale utilizzando le chiavi di accesso previste}
}
\newglossaryentry{Client} {
	name=Client,
	description={\invisiblesection{Client}Programma o parte di un programma (per es. un browser) che permette di scambiare dati con un server}
}
\newglossaryentry{Shiny} {
	name=Shiny,
	description={\invisiblesection{Shiny}È un'applicazione web che permette di creare applicazioni web interattive attraverso l'uso del linguaggio R}
}
\newglossaryentry{Script} {
	name=Script,
	description={\invisiblesection{Script}Programma o sequenza di istruzioni che viene interpretata o portata a termine da un altro programma}
}
\newglossaryentry{Interfacce} {
	name=Interfacce,
	description={\invisiblesection{Interfacce}L'interfaccia è il componente fisico o logico che permette a due o più sistemi elettronici di comunicare e interagire. Interfacciare vuol dire quindi collegare, seguendo un formato standard che consenta lo scambio di dati, due o più dispositivi eterogenei in modo da permettere loro lo scambio di informazioni}
}
\newglossaryentry{GIF} {
	name=GIF,
	description={\invisiblesection{GIF}GIF (Graphics Interchange Format) è un formato per immagini digitali di tipo bitmap molto utilizzato nel World Wide Web}
}
\newglossaryentry{Heroku} {
	name=Heroku,
	description={\invisiblesection{Heroku}Heroku è un Platform as a service (PaaS) sul cloud che supporta diversi linguaggi di programmazione}
}
\newglossaryentry{SQL} {
	name=SQL,
	description={\invisiblesection{SQL}SQL (Structured Query Language) è un linguaggio standardizzato per database basati sul modello relazionale (RDBMS)}
}
\newglossaryentry{Android} {
	name=Android,
	description={\invisiblesection{Android}Sistema operativo, basato su Linux, per dispositivi mobili lanciato da Google nel 2007 e destinato a smartphone e tablet}
}
\newglossaryentry{Jenkins} {
	name=Jenkins,
	description={\invisiblesection{Jenkins}È uno strumento open source scritto in linguaggio Java che fornisci servizi di integrazione continua per lo sviluppo del software}
}
\newglossaryentry{Git} {
	name=Git,
	description={\invisiblesection{Git}Git è un software di controllo versione distribuito utilizzabile da interfaccia a riga di comando, creato da Linus Torvalds nel 2005}
}
\newglossaryentry{Slack} {
	name=Slack,
	description={\invisiblesection{Slack}Piattaforma di messaggistica istantanea dedicata a organizzazioni ed aziende. Quest'ultima offre anche servizi di condivisione file ed è strutturata a canali. Ogni canale ha un nome, che non può essere modificato, e una lista di persone appartenenti all'organizzazione, che possono leggerlo e scriverci}
}
\newglossaryentry{Python3} {
	name=Python3,
	description={\invisiblesection{Python3}Versione 3 del linguaggio di programmazione Python. Questo è un linguaggio di programmazione ad alto livello, orientato agli oggetti, adatto, tra gli altri usi, per sviluppare applicazioni distribuite, scripting, computazione numerica e system testing}
}
\newglossaryentry{Software} {
	name=Software,
	description={\invisiblesection{Software}Insieme delle procedure e delle istruzioni in un sistema di elaborazione dati che si identifica con un insieme di programmi}
}
\newglossaryentry{W3C markup} {
	name=W3C markup,
	description={\invisiblesection{W3C markup}Servizio offerto dal w3c(ente che regola lo standard HTML) che permette di verificare la validità di di un documento scritto in un linguaggio markup}
}
\newglossaryentry{Astah} {
	name=Astah,
	description={\invisiblesection{Astah}Software di modellazione grafica utilizzato per la redazione di diagrammi, quali Use Case, classi e sequenza. Questo software mette a disposizione la versione professionale gratuitamente agli studenti}
}
\newglossaryentry{Diagrammi delle classi} {
	name=Diagrammi delle classi,
	description={\invisiblesection{Diagrammi delle classi}È una tipologia di diagrammi che permette di descrivere come è organizzata una classe logica in modo semplice e preciso}
}
\newglossaryentry{Progetto} {
	name=Progetto,
	description={\invisiblesection{Progetto}Insieme di operazioni eseguite da un gruppo per svolgere il lavoro richiesto}
}
\newglossaryentry{Source} {
	name=Source,
	description={\invisiblesection{Source}In informatica e il codice dal quale viene generato il software eseguibile distribuito}
}
\newglossaryentry{Google Drive} {
	name=Google Drive,
	description={\invisiblesection{Google Drive}Servizio di salvataggio remoto e condivisione documenti gratuito fino a 15GB di spazio offerti da Google}
}
\newglossaryentry{Halstead} {
	name=Halstead,
	description={\invisiblesection{Halstead}La metrica Halstead è un'unità di misurazione della complessità del codice introdotta da Maurice Howard Halstead nel 1977. Questo tipo di misurazione della complessità, si basa sull'analisi statica del codice, per non essere influenzata dall'esecuzione nella specifica macchina}
}
\newglossaryentry{Compilatore} {
	name=Compilatore,
	description={\invisiblesection{Compilatore}Software che svolge una funzione di traduzione e composizione di file sorgenti in codice funzionante}
}
\newglossaryentry{Release} {
	name=Release,
	description={\invisiblesection{Release}Ciascuna versione di un software, messa in commercio o comunque diffusa, contraddistinta da un numero}
}
\newglossaryentry{Framework} {
	name=Framework,
	description={\invisiblesection{Framework}È un'architettura logica di supporto (spesso un'implementazione logica di un particolare design pattern) su cui un software può essere progettato e realizzato}
}
\newglossaryentry{CSS} {
	name=CSS,
	description={\invisiblesection{CSS}CSS, Cascading Style Sheets è un linguaggio di markup che descrive lo stile di un documento di una pagina web. In particolare è usato per definire la formattazione di documenti HTML, XHTML e XML ad esempio i siti web e relative pagine web}
}
\newglossaryentry{Telegram} {
	name=Telegram,
	description={\invisiblesection{Telegram}Piattaforma di messaggistica gratuita}
}
\newglossaryentry{Specifica Tecnica} {
	name=Specifica Tecnica,
	description={\invisiblesection{Specifica Tecnica}Documento che descrive strumenti e tecnologie utilizzate durante lo sviluppo del software}
}
\newglossaryentry{Front-end} {
	name=Front-end,
	description={\invisiblesection{Front-end}Porzione del software che gestisce l'interazione tra l'utente e il sistema. Può anche essere considera la parte visibile del software}
}
\newglossaryentry{Mocha} {
	name=Mocha,
	description={\invisiblesection{Mocha}Libreria utilizzata per effettuare i test di funzioni NodeJS}
}
\newglossaryentry{React} {
	name=React,
	description={\invisiblesection{React}Framework JavaScript, che permette l'interazione con aggiornamenti in tempo reale con l'utente}
}
\newglossaryentry{Processo} {
	name=Processo,
	description={\invisiblesection{Processo}Insieme di operazioni eseguite al fine di svolgere un'attività}
}
\newglossaryentry{Commit} {
	name=Commit,
	description={\invisiblesection{Commit}È un'operazione nativa del sistema di versionamento Git che permette, si salvare lo stato dei file dopo una modifica e coincide con le unità più piccole gestite da Git}
}
\newglossaryentry{HTML5} {
	name=HTML5,
	description={\invisiblesection{HTML5}L'HTML5 è un linguaggio di markup per la strutturazione delle pagine web, pubblicato come W3C Recommendation da ottobre 2014, che amplia le funzionalità di HTML introducendo nuove feature, tra le quali l'asincronia}
}
\newglossaryentry{Cloud storage} {
	name=Cloud storage,
	description={\invisiblesection{Cloud storage}Il Cloud Storage è un modello di conservazione dati su computer in rete dove i dati stessi sono memorizzati su molteplici server virtuali generalmente ospitati presso strutture di terze parti o su server dedicati}
}
\newglossaryentry{API} {
	name=API,
	description={\invisiblesection{API}Acronimo di Application Programming Interface. Indica un insieme di procedure rese disponibili al programmatore, di solito raggruppate a formare un set di strumenti specifici per l'espletamento di un determinato compito}
}
\newglossaryentry{Maintainable JavaScript} {
	name=Maintainable JavaScript,
	description={\invisiblesection{Maintainable JavaScript}È un manuale, scritto da Nicholas C. Zakas, sui metodi per mantenere il proprio codice JavaScript in ordine, semplice da comprendere e facile da editare}
}
\newglossaryentry{SLA} {
	name=SLA,
	description={\invisiblesection{SLA}I Service Level Agreement (in italiano: accordo sul livello del servizio), in sigla SLA, sono strumenti contrattuali attraverso i quali si definiscono le metriche di servizio (es. qualità di servizio) che devono essere rispettate da un fornitore di servizi (provider) nei confronti dei propri clienti/utenti}
}
\newglossaryentry{Siri} {
	name=Siri,
	description={\invisiblesection{Siri}Creato da Apple Inc. Siri è l'assistente digitale presente nei dispositivi iOS, macOS, watchOS e tvOS quali iPhone, iPad, Mac, Apple Watch e Apple TV}
}
\newglossaryentry{Prodotto} {
	name=Prodotto,
	description={\invisiblesection{Prodotto}Da Swebook v3: \textit{A product is an economic good (or output) that is created in a process that transforms product factors (or inputs) to an output. When sold, a product is a deliverable that creates both a value and an experience for its users. A product can be a combination of systems, solutions, materials, and services delivered internally (e.g., in-house IT solution) or externally (e.g., software application), either as-is or as a component for another
product (e.g., embedded software)}}
}
\newglossaryentry{Database} {
	name=Database,
	description={\invisiblesection{Database}Archivio di dati strutturato in modo da razionalizzare la gestione e l'aggiornamento delle informazioni e da permettere lo svolgimento di ricerche complesse}
}
\newglossaryentry{Failure} {
	name=Failure,
	description={\invisiblesection{Failure}Da Swebook v3: \textit{Undesired effect observed in the system’s delivered service}, in particolare è il mancato raggiungimento di uno o più requisiti}
}
\newglossaryentry{Moduli} {
	name=Moduli,
	description={\invisiblesection{Moduli}Un modulo (software) è un componente software autonomo e ben identificato, e quindi facilmente riusabile. La programmazione modulare è un paradigma di programmazione nato con lo scopo di favorire la riusabilità dei componenti software}
}
\newglossaryentry{EBook} {
	name=EBook,
	description={\invisiblesection{EBook}Un ebook (scritto anche e-book o eBook), in italiano libro elettronico, è un libro in formato digitale a cui si può avere accesso mediante computer e dispositivi mobili, come smartphone, tablet PC e dispositivi appositamente ideati per la lettura di testi lunghi in digitale, detti eReader (ebook reader)}
}
\newglossaryentry{DynamoDB} {
	name=DynamoDB,
	description={\invisiblesection{DynamoDB}Dalla documentazione di Amazon DynamoDB: \textit{Amazon DynamoDB is a fully managed NoSQL database service that provides fast and predictable performance with seamless scalability. You can use Amazon DynamoDB to create a database table that can store and retrieve any amount of data, and serve any level of request traffic}}
}
\newglossaryentry{Diagrammi UML} {
	name=Diagrammi UML,
	description={\invisiblesection{Diagrammi UML}UML (Unified Modeling Language, "linguaggio di modellizzazione unificato") è un linguaggio di modellizzazione e specifica basato sul paradigma orientato agli oggetti. L'UML svolge un'importantissima funzione di "lingua franca" nella comunità della progettazione e programmazione a oggetti}
}
\newglossaryentry{Perl} {
	name=Perl,
	description={\invisiblesection{Perl}Perl è un linguaggio di programmazione ad alto livello, dinamico, procedurale e interpretato, creato nel 1987 da Larry Wall. Perl ha un singolare insieme di funzionalità ereditate da C, scripting shell Unix (sh), Awk, sed e in diversa misura da molti altri linguaggi di programmazione, compresi alcuni linguaggi funzionali}
}
\newglossaryentry{RStudio} {
	name=RStudio,
	description={\invisiblesection{RStudio}RStudio è un IDE open source multi piattaforma per il linguaggio R, attualmente il più diffuso e completo nel suo settore}
}
\newglossaryentry{GNU GPL v3} {
	name=GNU GPL v3,
	description={\invisiblesection{GNU GPL v3}GNU GPL v3 è una licenza per programmi liberi e open source. \url{https://www.gnu.org/licenses/gpl-3.0.en.html}}
}
\newglossaryentry{Web} {
	name=Web,
	description={\invisiblesection{Web}Abbreviazione di World Wide Web, è uno dei principali servizi di Internet che permette di navigare e usufruire di un insieme vastissimo di contenuti collegati tra loro attraverso legami (link)}
}
\newglossaryentry{Texmaker} {
	name=Texmaker,
	description={\invisiblesection{Texmaker}Texmaker editor \LaTeX cross-platform open source con un lettore PDF integrato. Texmaker è interamente realizzato in Qt e integra molti strumenti necessari per lo sviluppo di documenti \LaTeX}
}
\newglossaryentry{Risorse hardware} {
	name=Risorse hardware,
	description={\invisiblesection{Risorse hardware}Le risorse hardware sono delle risorse di un sistema legate alla sua parte elettronica e fisica, come la velocità del processore o la quantità di memoria RAM}
}
\newglossaryentry{MacOS} {
	name=MacOS,
	description={\invisiblesection{MacOS}Precedentemente noto come OS X e successivamente come Mac OS X, è il sistema operativo sviluppato da Apple Inc. per i computer Macintosh.}
}
\newglossaryentry{Swift} {
	name=Swift,
	description={\invisiblesection{Swift}Linguaggio di programmazione object-oriented per sistemi macOS, iOS, watchOS, tvOS e Linux, presentato da Apple durante la WWDC 2014. Swift è concepito per coesistere con il linguaggio Objective-C, tipico del software sviluppato per i sistemi operativi Apple, semplificando la scrittura del codice}
}
\newglossaryentry{Amazon Alexa} {
	name=Amazon Alexa,
	description={\invisiblesection{Amazon Alexa}Assistente personale virtuale sviluppato da Lab126 di Amazon.com, reso popolare da Amazon Echo. È in grado di effettuare interazione vocale, riproduzione di musica, creare liste di cose da fare, impostare sveglie, riprodurre podcast e audiobook, e fornire informazioni su meteo, traffico e altre informazioni in tempo reale}
}
\newglossaryentry{Package} {
	name=Package,
	description={\invisiblesection{Package}Raggruppamento logico di classi e metodi nel codice sorgente di un programma}
}
\newglossaryentry{Capitolato} {
	name=Capitolato,
	description={\invisiblesection{Capitolato}Atto allegato a un contratto d'appalto che intercorre tra il cliente ed una ditta appaltatrice in cui vengono indicate modalità, costi e tempi di realizzazione dell'opera oggetto del contratto}
}
\newglossaryentry{Trender} {
	name=Trender,
	description={\invisiblesection{Trender}Un sistema che permette di tracciare le necessarie dipendenze durante la produzione di software}
}
\newglossaryentry{Browser} {
	name=Browser,
	description={\invisiblesection{Browser}Un'applicazione per il recupero, la presentazione e la navigazione di risorse sul web. Il programma implementa da un lato le funzionalità di client per il protocollo HTTP, che regola lo scaricamento delle risorse dai server web a partire dal loro indirizzo URL, dall'altro quelle di visualizzazione dei contenuti ipertestuali (solitamente all'interno di documenti HTML) e di riproduzione di contenuti multimediali}
}
\newglossaryentry{Task} {
	name=Task,
	description={\invisiblesection{Task}Lavoro, mansione il cui lo svolgimento viene assegnato ad una persona}
}
\newglossaryentry{Automatizzare} {
	name=Automatizzando,
	description={\invisiblesection{Automatizzare}Significa rendere automatico un determinato movimento o una prefissata operazione di un dispositivo, di una macchina, ecc}
}
\newglossaryentry{JSHint} {
	name=JSHint,
	description={\invisiblesection{JSHint}È uno strumento di analisi statica di codice JavaScript per controllarne la compilazione}
}
\newglossaryentry{HTML} {
	name=HTML,
	description={\invisiblesection{HTML}L'HyperText Markup Language (traduzione letterale: linguaggio a marcatori per ipertesti), in informatica è il linguaggio di markup solitamente usato per la formattazione e impaginazione di documenti ipertestuali disponibili nel World Wide Web sotto forma di pagine web, nato assieme al web 1.0}
}
\newglossaryentry{Linguaggio di markup} {
	name=Linguaggio di markup,
	description={\invisiblesection{Linguaggio di markup}In generale un linguaggio di markup è un insieme di regole che descrivono i meccanismi di rappresentazione (strutturali, semantici o presentazionali) di un testo che, utilizzando convenzioni standardizzate, sono utilizzabili su più supporti. La tecnica di formattazione per mezzo di marcatori (o espressioni codificate) richiede quindi una serie di convenzioni, ovvero appunto di un linguaggio a marcatori di documenti}
}
\newglossaryentry{Piattaforma} {
	name=Piattaforma,
	description={\invisiblesection{Piattaforma}Base software e/o hardware su cui sono sviluppate e/o eseguite applicazioni}
}
\newglossaryentry{Attori} {
	name=Attori,
	description={\invisiblesection{Attori}Entità che prendono decisioni nei diagrammi di casi d'uso. Non è necessario che gli attori siano umani: \textit{An actor might be a person, a company or organization, a computer program, or a computer system-hardware, software, or both}}
}
\newglossaryentry{Meeting} {
	name=Meeting,
	description={\invisiblesection{Meeting}Riunione, convegno svolti al fine di discutere argomenti d'interesse. In lingua inglese la parola viene definita come un atto o processo per giungere assieme, come in una assemblea, ad uno scopo comune}
}
\newglossaryentry{Jolie} {
	name=Jolie,
	description={\invisiblesection{Jolie}Jolie è un linguaggio di programmazione open-source orientato ai microservizi sviluppato da italianaSoftware}
}
\newglossaryentry{Casi d'uso} {
	name=Casi d'uso,
	description={\invisiblesection{Casi d'uso}Un caso d'uso è un insieme di scenari (sequenze di azioni) che hanno in comune uno scopo finale (obiettivo) per un utente (attore)}
}
\newglossaryentry{Teamwork} {
	name=Teamwork,
	description={\invisiblesection{Teamwork}Teamwork è un'applicazione web che permette la pianificazione e la coordinazione di varie attività fra i componenti di un team per la realizzazione di un progetto. L'applicazione è stata sviluppata da Open Lab}
}
\newglossaryentry{MeteorJS} {
	name=MeteorJS,
	description={\invisiblesection{MeteorJS}Meteor, o MeteorJS, è un'applicazione web gratuita ed open-source, scritta usando Node.js, che permette la realizzazione di prototipi di programmi e produce codice multipiattaforma}
}
\newglossaryentry{Gulpease} {
	name=Gulpease,
	description={\invisiblesection{Gulpease}L'Indice Gulpease è un indice di leggibilità di un testo tarato sulla lingua italiana. Quest'indice utilizza la lunghezza delle parole in lettere anziché in sillabe, semplificandone il calcolo automatico}
}
\newglossaryentry{Gantt Chart} {
	name=Gantt Chart,
	description={\invisiblesection{Gantt Chart}Il diagramma di Gantt è uno strumento di supporto alla gestione dei progetti, realizzato da Henry Laurence Gantt. Esso è costruito partendo da un asse orizzontale che rappresenta l'arco temporale totale del progetto, suddiviso in fasi incrementali, e da un asse verticale che rappresenta le mansioni o attività che costituiscono il progetto}
}
\newglossaryentry{Alexa Skills Kit} {
	name=Alexa Skills Kit,
	description={\invisiblesection{Alexa Skills Kit}Alexa Skills Kit è una collezione di API ad uso personale, strumenti, documentazioni e campioni di righe di codice che semplificano e velocizzano l'aggiunta di nuove funzionalità ad Alexa}
}
\newglossaryentry{Chat} {
	name=Chat,
	description={\invisiblesection{Chat}Conversazione in tempo reale fra due o più interlocutori che avviene per mezzo di una comunicazione telefonica o via internet}
}
\newglossaryentry{Open source} {
	name=Open source,
	description={\invisiblesection{Open source}In informatica, il termine inglese open source indica un software di cui gli autori (più precisamente, i detentori dei diritti) rendono pubblico il codice sorgente, favorendone il libero studio e permettendo a programmatori indipendenti di apportarvi modifiche ed estensioni}
}
\newglossaryentry{Risorse software} {
	name=Risorse software,
	description={\invisiblesection{Risorse software}Componenti software che costituiscono un sistema}
}
\newglossaryentry{Flowchart} {
	name=Flowchart,
	description={\invisiblesection{Flowchart}Il termine Flowchart (diagramma di flusso) indica una rappresentazione grafica, o diagramma, che in informatica e altre discipline, viene usata per gestire il flusso di controllo negli algoritmi e nei processi}
}
\newglossaryentry{Walkthrough} {
	name=Walkthrough,
	description={\invisiblesection{Walkthrough}Analisi informale del codice svolta dai vari componenti del team, i quali scelgono alcuni casi di test e simulano l'esecuzione del codice a mano}
}
\newglossaryentry{Django} {
	name=Django,
	description={\invisiblesection{Django}Django è un web framework open source per lo sviluppo di applicazioni web, scritto in linguaggio Python, seguendo il pattern Model-View-Controller}
}
\newglossaryentry{PDCA} {
	name=PDCA,
	description={\invisiblesection{PDCA}Acronimo di Plan-Do-Check-Act, detto anche Ciclo di Deming o Ciclo di Miglioramento Continuo, è un metodo che permette di perseguire un continuo miglioramento della qualità nei processi}
}
\newglossaryentry{Ospite} {
	name=Ospite,
	description={\invisiblesection{Ospite}Persona o gruppo di persone in visita all'ufficio di zero12}
}
\newglossaryentry{Diagrammi dei package} {
	name=Diagrammi dei package,
	description={\invisiblesection{Diagrammi dei package}Diagrammi che descrivono la struttura dei package ponendo evidenza alle dipendenze fra le classi. Essi sono utili in sistemi medio‐grandi per tenere sotto controllo la complessità strutturale}
}
\newglossaryentry{AWS SDK per Node.js} {
	name=AWS SDK per Node.js,
	description={\invisiblesection{AWS SDK per Node.js}Strumenti che fanno parte degli Amazon Web Services utili a sviluppare codice per Node.js}
}
\newglossaryentry{AWS Lambda} {
	name=AWS Lambda,
	description={\invisiblesection{AWS Lambda}Servizio di elaborazione di tipo serverless offerto da Amazon}
}
\newglossaryentry{SonarQube} {
	name=SonarQube,
	description={\invisiblesection{SonarQube}Piattaforma open source per la gestione della qualità del codice che produce reports sul codice duplicato, sugli standards di programmazione, i test di unità, il code coverage, la complessità, i bugs potenziali, i commenti, la progettazione e l’architettura}
}
\newglossaryentry{NPM} {
	name=NPM,
	description={\invisiblesection{NPM}Principale software utilizzato per maneggiare i moduli di Node.js e consente di condividere il codice per problemi tipici tra gli sviluppatori JavaScript}
}
\newglossaryentry{TypeScript} {
	name=TypeScript,
	description={\invisiblesection{TypeScript}Linguaggio di programmazione libero ed Open Source che estende la sintassi di JavaScript e sviluppato da Microsoft}
}
\newglossaryentry{Amazon DynamoDB} {
	name=Amazon DynamoDB,
	description={\invisiblesection{Amazon DynamoDB}Tipologia di database che supporta lo storage, le query e l'aggiornamento di documenti e fa parte degli Amazon Web Services}
}
\newglossaryentry{LWA} {
	name=LWA,
	description={\invisiblesection{LWA} LWA (Login With Amazon) servizio che permette di effettuare il login con le credenziali Amazon in modo da poter accedere a tutti i servizi messi a disposizione}
}