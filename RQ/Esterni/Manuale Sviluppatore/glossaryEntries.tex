\newglossaryentry{Admin}
{
  name=Admin,
  description={Tipologia di utente con maggiori privilegi rispetto ad un utente normale utilizzatore del software}
}

\newglossaryentry{token}
{
  name=Token,
  description={Stringa alfanumerica pseudo-casuale utilizzata per l'autenticazione di un amministratore nel sistema}
}

\newglossaryentry{ospite}
{
  name=ospite,
  description={Utente generale utilizzatore del sistema senza privilegi di amministrazione}
}

\newglossaryentry{tecnologie di sintetizzazione vocale}
{
  name=Tecnologie di sintetizzazione vocale,
  description={Tecnologie per la riproduzione artificiale della voce umana. Un sistema usato per questo scopo è detto sintetizzatore vocale e può essere realizzato tramite software o via hardware. I sistemi di sintesi vocale sono noti anche come sistemi text-to-speech (TTS) (in italiano: da testo a voce) per la loro possibilità di convertire il testo in parlato}
}

\newglossaryentry{interlocutore}
{
  name=Interlocutore,
  description={Rapprensenta la persona che l'ospite intende incontrare all'interno dell'azienda}
}

\newglossaryentry{Slack}
{
  name=Slack,
  description={Piattaforma cloud-based di messaggistica istantanea dedicata a organizzazioni ed aziende}
}

\newglossaryentry{Javascript}
{
  name=Javascript,
  description={Linguaggio di scripting orientato agli oggetti e agli eventi, comunemente utilizzato nella programmazione Web lato client per la creazione di siti o applicazioni web}
}

\newglossaryentry{SuperAdmin}
{
  name=SuperAdmin,
  description={Tipologia di utente che ha privilegi aggiuntivi sugli admin (modifica dei suoi dati)}
}

\newglossaryentry{framework}
{
  name=Framework,
  description={In informatica e specificatamente nello sviluppo software, è un'architettura logica di supporto (spesso un'implementazione logica di un particolare design pattern) su cui un software può essere progettato e realizzato, spesso facilitandone lo sviluppo da parte del programmatore}
}

\newglossaryentry{front-end}
{
  name=Front-end,
  description={Rappresenta la parte visibile all'utente con cui egli può interagire}
}

\newglossaryentry{runtime}
{
  name=Runtime,
  description={Momento in cui un programma per computer viene eseguito, in contrapposizione ad altre fasi del ciclo di vita del software}
}

\newglossaryentry{features}
{
  name=Features,
  description={Proprietà o funzionalità distintiva del software}
}

\newglossaryentry{thread}
{
  name=Thread,
  description={Suddivisione di un processo in due o più  sottoprocessi che vengono eseguiti concorrentemente da un sistema di elaborazione monoprocessore (multithreading), multiprocessore o multicore}
}

\newglossaryentry{modello I/O}
{
  name=Modello I/O,
  description={Modelli che prevedono metodi per eseguire operazioni di input/output (I/O) in un computer, tra la CPU ed un dispositivo di I/O. Un altro metodo è quello di usare dei processori di I/O dedicati (canali)}
}

\newglossaryentry{event-driven}
{
  name=Event-driven,
  description={Paradigma di programmazione dell'informatica. Mentre in un programma tradizionale l'esecuzione delle istruzioni segue percorsi fissi, che si ramificano soltanto in punti ben determinati predefiniti dal programmatore, nei programmi scritti utilizzando la tecnica a eventi il flusso del programma è largamente determinato dal verificarsi di eventi esterni}
}

\newglossaryentry{funzioni di callback}
{
  name=Funzioni di callback,
  description={In programmazione, un callback (o, in italiano, richiamo) è, in genere, una funzione, o un "blocco di codice" che viene passata come parametro ad un'altra funzione}
}

\newglossaryentry{asincrona}
{
  name=Asincrona,
  description={La comunicazione asincrona è quella comunicazione in cui il mittente invia il messaggio e poi continua la propria esecuzione; asincrona sta proprio a significare l'asincronicità che vi è tra l'invio di un messaggio e la risposta al messaggio stesso}
}

\newglossaryentry{serverless}
{
  name=Serverless,
  description={Con il termine serverless (dall'inglese senza server) si intende un network la cui gestione non viene incentrata su dei server, come spesso accade, ma viene dislocata fra i vari utenti che utilizzano il network stesso, quindi il lavoro necessario di gestione del network viene eseguito dagli stessi utilizzatori. In questo modo non sarà possibile chiudere un intero network disattivando i soli server, ma la rete sarà attiva fin quando ci saranno persone che la utilizzeranno}
}

\newglossaryentry{Database Management System}
{
  name=Database Management System,
  description={Abbreviato anche con l'acronimo DBMS o Sistema di gestione di basi di dati, è un sistema software progettato per consentire la creazione, la manipolazione (da parte di un amministratore DBA) e l'interrogazione efficiente (da parte di uno o più utenti client) del database}
}

\newglossaryentry{ASK}
{
  name=ASK,
  description={Features di terze parti sviluppate per l'esperienza vocale che aggiungono nuove funzionalità ai dispositivi che utilizzano Alexa}
}

\newglossaryentry{parsing}
{
  name=Parsing,
  description={Processo che analizza un flusso continuo di dati in ingresso (input) in modo da determinare la sua struttura grazie ad una data grammatica formale}
}

\newglossaryentry{matching}
{
  name=Matching,
  description={Azione di controllo della presenza di un certo pattern all'interno di un oggetto composito}
}

\newglossaryentry{report}
{
  name=Report,
  description={Resoconto informativo}
}

\newglossaryentry{Promise}
{
  name=Promise,
  description={Un oggetto di tipo Promise è usato nella programmazione asincrona. Un oggetto di questo tipo rappresenta un valore che può essere disponibile adesso, nel futuro o mai}
}

\newglossaryentry{Homebrew}
{
  name=Homebrew,
  description={È uno strumento gratuito ed open-source usato per semplificare l'installazione di software nei sistemi operativi macOS}
}

\newglossaryentry{top-down}
{
  name=Top-down,
  description={Nel modello top-down si formula inizialmente una visione generale del sistema, ovvero se ne descrive la finalità principale senza scendere nel dettaglio delle sue parti. Ogni parte del sistema è successivamente rifinita (decomposizione, specializzazione e specificazione o identificazione) aggiungendo maggiori dettagli della progettazione}
}

\newglossaryentry{design pattern architetturali}
{
  name=Design pattern architetturali,
  description={Nell'ambito dell'ingegneria del software, un design pattern (traducibile in lingua italiana come schema progettuale, schema di progettazione, schema architetturale), è una descrizione o modello logico da applicare per la risoluzione di un problema che può presentarsi in diverse situazioni durante le fasi di progettazione e sviluppo del software, prima ancora della fase di codifica}
}






