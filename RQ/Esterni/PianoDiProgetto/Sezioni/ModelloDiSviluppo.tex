\documentclass[../PianoDiProgetto_v3.0.0.tex]{subfiles}

\begin{document}

\section{Modello di Sviluppo}
Il Modello di Sviluppo scelto per il prodotto è il \gl{Modello Incrementale}: il progetto viene suddiviso in periodi, ognuno delimitato da una milestone. In questo modo viene fornita al proponente la possibilità di valutare lo stato del prodotto al termine di ogni periodo, inoltre permette di mitigare i rischi esposti nella sezione precendente Analisi dei rischi perché in caso di mancanze di informazioni o errori da correggere dopo una revisione sarà possibile rivedere il soggetto in esame senza dover rifare anche ciò che invece era corretto. Il committente potrà così produrre dei responsi valutativi utili a facilitare l'orientamento del lavoro nei periodi successivi. Nel caso in cui il soddisfacimento dei requisiti obbligatori richieda più tempo di quello previsto, il Quinto e Sesto Periodo, rispettivamente il periodo di Codifica dei Requisiti Desiderabili ed il periodo di Codifica dei Requisiti Opzionali, verranno ridimensionati ed, eventualmente, neanche avviati, al fine di focalizzare meglio l'attenzione sul soddisfacimento dei vincoli obbligatori. I sette periodi descritti successivamente verranno suddivisi in attività meno onerose, per permettere un maggior controllo sull’avanzamento del progetto e dare la possibilità di applicare più frequentemente il modello di Deming del \gl{Miglioramento Continuo}\ (PDCA).\\
I periodi sono organizzati in modo tale da permettere di presentarsi alle consegne delle revisioni previste, le scadenze che verranno rispettate per queste sono le seguenti:
	\begin{itemize}
		\item \revisionedeirequisiti: 2017/01/24;
		\item \revisionediprogettazione: 2017/03/13 presentandosi con \revisionediprogettazionemax;
		\item \revisionediqualifica: 2017/04/18;
		\item \revisionediaccettazione: 2017/05/15.
	\end{itemize}

	\subsection{Primo Periodo - Analisi}
	Questo periodo comprende quattro attività principali:

	\begin{itemize}
		\item Individuazione degli strumenti necessari alla comunicazione fra i componenti del team;
		\item Individuazione degli strumenti utili alla redazione dei documenti;
		\item Scelta del progetto da svolgere;
		\item Analisi dei requisiti specificati nel Capitolato del progetto che si intende realizzare.
	\end{itemize}
			
	Questo periodo si concluderà con la \revisionedeirequisiti\ (\gl{RR}) che permette al proponente di verificare che siano stati analizzati tutti i requisiti richiesti.
			
	\subsection{Secondo Periodo - Analisi di Dettaglio}
	In questo periodo verrà effettuata un'analisi più dettagliata dei requisiti. Grazie anche al feedback ricevuto dal committente in seguito alla RR, saranno aggiunti nuovi eventuali requisiti, oppure verranno modificati quelli trovati in precedenza, nel caso non rispecchino appieno le esigenze del proponente.
		
	\subsection{Terzo Periodo - Terzo Periodo - Progettazione Architetturale e Progettazione di Dettaglio}
	Periodo di importanza fondamentale. Durante la Progettazione Architetturale inizierà a prendere forma la struttura logica del prodotto, realizzata grazie alle specifiche individuate e descritte durante i periodi precedenti. Al termine di questo periodo verranno incrementate le versioni dei documenti e sarà prodotta la \gl{\definizionediprodotto}. Questo periodo terminerà con la \revisionediprogettazione\ (\gl{RP}), che garantirà che tutti i vincoli stabiliti dal proponente sono stati rispettati. Il numero di versione dei documenti prodotti nei periodi precedenti verrà incrementato.
		
	\subsection{Quarto Periodo - Revisione di Progettazione e Codifica dei Requisiti Obbligatori}
	Nella Progettazione di Dettaglio verrà descritta la struttura del sistema diviso nelle sue componenti minimali. In seguito si procederà alla realizzazione di un software che soddisfi tutti requisiti obbligatori individuati. Verranno inoltre redatti il \manualeutente\ ed il \manualesviluppatore\ parallelamente all'attività di codifica ed il numero di versione di alcuni documenti prodotti nei periodi precedenti verrà ulteriormente incrementato.
		
	\subsection{Quinto Periodo - Codifica dei Requisiti Desiderabili}
	Periodo che segue quella di Progettazione di Dettaglio e Codifica dei Requisiti Obbligatori. Questo periodo è molto simile al precedente, con la differenza che verranno trattati i requisiti desiderabili, ovvero dei requisiti aggiuntivi ma non strettamente necessari. Sarà compito del committente, al termine della Revisione di Progettazione, fornire le specifiche di eventuali nuovi requisiti desiderabili. Verrà inoltre aumentato il numero di versione dei rimanenti documenti prodotti. Questo periodo termina con la \revisionediqualifica\ (\gl{RQ}), nella quale verrà presentato un software che soddisfi i requisiti obbligatori, i requisiti desiderabili definiti dagli Analisti.
		
	\subsection{Sesto Periodo - Codifica dei Requisiti Opzionali}
	In questo periodo verranno implementati gli eventuali requisiti opzionali, ovvero i requisiti che non sono stati specificati nel Capitolato Tecnico, ma che sono stati decisi in seguito, in comune accordo con il proponente. Il numero di versione dei documenti prodotti nei periodi precedenti verrà ulteriormente incrementato.
		
	\subsection{Settimo Periodo - Validazione}
	Periodo conclusivo dello Sviluppo del prodotto. Durante questo periodo verranno eseguiti la validazione ed il successivo collaudo del software. Alla fine si procederà con la \revisionediaccettazione (\gl{RA}). \\

\end{document}