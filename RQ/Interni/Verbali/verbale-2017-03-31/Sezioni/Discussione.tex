\documentclass[../verbale-2017-03-31.tex]{subfiles}

\begin{document}
	\section{Discussione dell'ordine del giorno}
	\begin{itemize}
		\item Sono stati riscontrati dei problemi dei problemi riguardanti l'invio dell'audio ad Alexa. Dopo un'attenta discussione ed un confronto con il gruppo concorrente AnSWEr, anche loro incappati nello stesso problema, si è deciso di non svolgere ulteriori tentativi per evitare un eccessivo ritardo delle attività. Si è dunque optato di testare il funzionamento della Skill attraverso l'utilizzo di un applicativo esterno chiamato Echosim.io. Inoltre a causa di ciò è stato rivisto la tipologia di due requisiti; ora RFF1(attivazione tramite keyword) è un requisito opzionale e ROF2 (attivazione tramite pulsante) è diventato un requisito obbligatorio;
		\item Sono stati rilevati dei ritardi nella codifica del Front-End, verrà fatta, quindi, una riassegnazione della sezione dell'architettura da codificare. Le modifiche vengono riportate di seguito:
			\begin{itemize}
				\item \textbf{Front-End}
					\begin{itemize}
						\item \textbf{Antonino, Daniele, Marco}: UI di AdminPage;
						\item \textbf{Antonino, Daniele, Marco}: Service di AdminPage.
					\end{itemize}
			\end{itemize}
		\item L'attività di redazione, verifica e approvazione del \manualeutente\ viene suddivisa in questo modo:
			\begin{itemize}
				\item \textbf{Redazione}: Francesco, Antonino;
				\item \textbf{Verifica}: Giacomo, Daniele, Marco;
				\item \textbf{Approvazione}: Edoardo.
			\end{itemize}
		\item L'attività di redazione, verifica e approvazione del \manualesviluppatore\ viene suddivisa in questo modo:
			\begin{itemize}
				\item \textbf{Redazione}: Giacomo, Filippo, Marco, Daniele;
				\item \textbf{Verifica}: Francesco, Antonino;
				\item \textbf{Approvazione}: Edoardo.
			\end{itemize}
		\item L'approvazione dei documenti viene assegnata a Edoardo, \responsabilediprogetto\ del Quinto Periodo.
		\item È stato deciso di utilizzare nuovi strumenti per la verifica dei test, ovvero Mocha e Chai.
	\end{itemize}
\end{document}