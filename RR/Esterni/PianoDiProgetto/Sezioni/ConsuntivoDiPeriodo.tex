\documentclass[../PianoDiProgetto.tex]{subfiles}

\begin{document}
	\section{Consuntivo di periodo}
	Di seguito verranno indicate le spese effettivamente sostenute, sia per ruolo che per persona.\\
	Il bilancio potrà risultare:
	\begin{itemize}
		\item \textbf{Positivo}: se il preventivo supera il consuntivo;
		\item \textbf{In pari}: se il preventivo e il consuntivo sono equivalenti;
		\item \textbf{Negativo}: se il consuntivo supera il preventivo.
	\end{itemize}
	
	\subsection{Fase 1}
		\subsubsection{Consuntivo}
		Essendo \kpanic\ non ancora stato scelto come fornitore ufficiale per il progetto \progetto, i suoi componenti hanno svolto le ore di lavoro come approfondimento personale.\\
		I dati riportati sulla seguente tabella riguardano le ore non rendicontate.
		\begin{table}[h]
		\centering
		\begin{tabular}{l * {2}{c}}
			\toprule
			\textbf{Ruolo} & \textbf{Ore} & \textbf{Costo (\euro{})} \\
			\midrule
			\responsabilediprogetto &	32 (+3) & 1050,00 \\

			\amministratore & 64 (+2) & 1320,00 \\
			
			\progettista & 0 & 0,00 \\
			
			\analista & 68 (+5) & 1825,00 \\
			
			\programmatore & 0 & 0,00 \\
			
			\verificatore & 46 (+4) & 7500,00 \\
			
			\midrule
			\textbf{Totale Preventivo} & 210 & 4.630,00 \\		
			\textbf{Totale Consuntivo} & 223 & 4945,00 \\
			\midrule
			\textbf{Differenza} & +14 & +315,00 \\
			\bottomrule
		\end{tabular}
		\caption{Fase 1 - Consuntivo}
		\label{tab:consuntivo1}
	\end{table}	
		\subsubsection{Conclusioni}
		Come si può vedere dalla tabella \ref{tab:consuntivo1}, che contiene i dati riguardanti il consuntivo della fase 1, è stato impiegato più tempo rispetto a quanto previsto in tutti i ruoli, di conseguenza il bilancio risulta \textbf{negativo}.
		
			Vista la mancanza di esperienza nella pianificazione e nella conoscienza di progetti sui quali basare la preventivazione dei costo, da parte dei componenti del gruppo che hanno svolto il ruolo di \responsabilediprogetto, sono state necessarie più ore di lavoro.
			
			Il lavoro svolto dagli \amministratori\ è rimasto in linea quanto preventivato con quanto era stato previsto, visto che chi ha svolto questo ruolo aveva già conoscenza della piattaforma su cui sono stati basati i software di supporto utilizzati, come quello per il tracciamento dei requisiti.
			
			L'attività degli \analisti\ ha impiegato più ore di quanto preventivato poiché il capitolato scelto ha richiesto una buona dose di innovazione e di ricerca che, in questa fase, ha impattato sulla specifica dei casi d'uso e dei requisiti.
			
			Per quanto riguarda i \verificatori, le ore aggiuntive sono state necessarie per le numerose modifiche fatte sui documenti che andavano verificati. 
\end{document}