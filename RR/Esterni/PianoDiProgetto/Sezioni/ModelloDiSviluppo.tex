\documentclass[../PianoDiProgetto.tex]{subfiles}

\begin{document}

\section{Modello di Sviluppo}
Il Modello di Sviluppo scelto per il prodotto è il \gl{Modello Incrementale}: il progetto viene suddiviso in fasi, ognuna delimitata da una milestone. In questo modo viene fornita al proponente la possibilità di valutare lo stato del prodotto al termine di ogni fase. Il committente potrà così produrre dei responsi valutativi utili a facilitare l'orientamento del lavoro nelle fasi successive. Nel caso in cui il soddisfacimento dei requisiti obbligatori richieda più tempo di quello previsto, le fasi 5 e 6, rispettivamente la fase di Progettazione di Dettaglio e codifica dei Requisiti Desiderabili e la fase di Progettazione di Dettaglio e codifica dei Requisiti Opzionali, verranno ridimensionate ed, eventualmente, neanche avviate, al fine di focalizzare meglio l'attenzione sul soddisfacimento dei vincoli obbligatori. Le sette fasi descritte precedentemente verranno suddivise in sottofasi meno onerose, per permettere un maggior controllo sull’avanzamento del progetto e dare la possibilità di applicare più frequentemente il modello di Deming del \gl{Miglioramento Continuo}\ (PDCA).

	\subsection{Fase 1 - Analisi}
	Questa fase si suddivide ulteriormente in quattro sottofasi:

	\begin{itemize}
		\item Individuazione degli strumenti necessari alla comunicazione fra i componenti del team;
		\item Individuazione degli strumenti utili alla redazione dei documenti;
		\item Scelta del progetto da svolgere;
		\item Analisi dei requisiti specificati nel Capitolato del progetto che si intende realizzare.
	\end{itemize}
			
	Questa fase si concluderà con la \revisionedeirequisiti\ (\gl{RR}) che permette al proponente di verificare che siano stati analizzati tutti i requisiti richiesti.
			
	\subsection{Fase 2 - Analisi di Dettaglio}
	In questa fase verrà effettuata un'analisi più dettagliata dei requisiti. Grazie anche al feedback ricevuto dal committente in seguito alla RR, saranno aggiunti nuovi eventuali requisiti, oppure verranno modificati quelli trovati in precedenza, nel caso non rispecchino appieno le esigenze del proponente.
		
	\subsection{Fase 3 - Progettazione Architetturale}
	Fase di importanza fondamentale. Durante la Progettazione Architetturale inizierà a prendere forma la struttura logica del prodotto, realizzata grazie alle specifiche individuate e descritte durante le fasi precedenti. Al termine di questa fase verranno incrementate le versioni dei documenti e sarà prodotta la \gl{Specifica Tecnica}.
		
	\subsection{Fase 4 - Progettazione di Dettaglio e Codifica dei Requisiti Obbligatori}
	Nella Progettazione di Dettaglio verrà descritta la struttura del sistema diviso nelle sue componenti minimali. In seguito si procederà alla realizzazione di un software che soddisfi tutti requisiti individuati. Questa fase terminerà con la \revisionediprogettazione\ (\gl{RP}), che garantirà che tutti i vincoli stabiliti dal proponente sono stati rispettati. Il numero di versione dei documenti prodotti nelle fasi precedenti verrà incrementato. Alla RP si prevede la consegna del documento Definizione di \gl{Prodotto} .
		
	\subsection{Fase 5 - Progettazione di Dettaglio e Codifica dei Requisiti Desiderabili}
	Fase che segue quella di Progettazione di Dettaglio e Codifica dei Requisiti Obbligatori. Questa fase è molto simile alla precedente, con la differenza che verranno trattati i requisiti desiderabili, ovvero dei requisiti aggiuntivi ma non strettamente necessari. Sarà compito del committente, al termine della Revisione di Progettazione, fornire le specifiche di eventuali nuovi requisiti desiderabili. Il numero di versione dei documenti prodotti nelle fasi precedenti verrà incrementato.
		
	\subsection{Fase 6 - Progettazione di Dettaglio e Codifica dei Requisiti Opzionali}
	In questa fase verranno implementati gli eventuali requisiti opzionali, ovvero i requisiti che non sono stati specificati nel Capitolato Tecnico, ma che sono stati decisi in seguito, in comune accordo con il proponente. Il numero di versione dei documenti prodotti nelle fasi precedenti verrà ulteriormente incrementato. Questa fase termina con la \revisionediqualifica\ (\gl{RQ}), nella quale verrà presentato un software che soddisfi i requisiti obbligatori, i requisiti desiderabili e i requisiti opzionali definiti dagli Analisti.
		
	\subsection{Fase 7 - Validazione}
	Fase conclusiva dello Sviluppo del prodotto. Durante questa fase verranno eseguiti la validazione ed il successivo collaudo del software. Alla fine si procederà con la \revisionediaccettazione. \\

\end{document}