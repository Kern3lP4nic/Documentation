\documentclass[../PianoDiProgetto.tex]{subfiles}

\begin{document}

\section{Pianificazione}
In questa sezione verranno elencate le attività che costituiscono le varie fasi di sviluppo del prodotto. Ogni attività sarà accompagnata da diagrammi che mostrano le tempistiche con le quali deve essere svolta.
	
	\subsection{Fase 1 - Analisi}
Questa fase inizia con la formazione del gruppo \kpanic\ e si conclude con la \textbf{\revisionedeirequisiti}\ (RR).\\
\\
\textbf{Periodo: dal 2016/11/28 al 2017/01/11}
\\
\\
Le attività sono le seguenti:
\begin{enumerate}
	\item Scelta degli Strumenti: in questa fase vengono concordati gli strumenti da utilizzare per la comunicazione  fra i membri del gruppo e quelli da utilizzare per la stesura della documentazione;
	\item Stesura delle \normediprogetto: in questo documento verranno specificate tutte le norme concordate fra i membri del gruppo una volta che sono stati scelti gli strumenti alla fase precedente. Questo documento viene redatto indipendentemente dal tipo di capitolato scelto;
	\item Si procede alla stesura dei seguenti documenti:
		\begin{itemize}
			\item \studiodifattibilita: si procede alla valutazione tecnica di tutti i sei Capitolati Tecnici proposti, analizzando per ognuno i vari pro e contro. Si procede con la redazione della prima versione del documento \textbf{Studio di Fattibilità}. Al termine di questa fase viene deciso di comune accordo fra i vari membri del gruppo il Capitolato Tecnico da svolgere;
			\item \pianodiprogetto: si procede alla stesura della prima versione del documento \textbf{\pianodiprogetto}\ nel quale vengono specificate le attività che dovranno essere svolte dal team;
			\item \pianodiqualifica: si procede alla stesura della prima versione del documento \textbf{\pianodiqualifica}\ nel quale si elencano gli obiettivi che il team intende raggiungere e le modalità con le quali intende perseguirli;
			\item \glossario: la stesura della prima versione del documento \textbf{\glossario}\ viene eseguita in modo automatico per quanto riguarda la scrittura dei termini, mentre la loro definizione viene redatta manualmente;
			\item \analisideirequisiti: la prima versione del documento \textbf{\analisideirequisiti}\ viene redatta specificando sia i requisiti tecnici carpiti dal Capitolato scelto, sia i requisiti specificati in seguito dal proponente durante gli incontri pianificati o la comunicazione telematica.
		\end{itemize}	 
\end{enumerate}

	\begin{figure}[!h]
		\centering
		\includegraphics[width=\textwidth]{Pianificazione/Immagini/GanttChart01.png}
		\caption{GanttChart 1}
	\end{figure}	

	\subsection{Fase 2 - Analisi di Dettaglio}
	Questa fase inizia dopo la \revisionedeirequisiti\ e termina con la Milestone prevista per la fase 2.
	\\
	\\
	\textbf{Periodo: dal 2017/01/25 al 2017/02/03}
	\\
	\\
	Si procede con un incremento della documentazione a seguito della RR. Nei documenti verranno aggiunte anche le specifiche riguardanti eventuali nuove richieste del proponente. Si procede quindi con l'analisi dei requisiti appena individuati.
	
	\begin{figure}[!h]
		\centering
		\includegraphics[width=\textwidth]{Pianificazione/Immagini/GanttChart02.png}
		\caption{GanttChart 2}
	\end{figure}	
		
	\subsection{Fase 3 - Progettazione Architetturale}
	Questa fase inizia dopo la Milestone programmata per la fase 2 e termina con la Milestone programmata per la fase 3.
	\\
	\\
	\textbf{Periodo: dal 2017/02/06 al 2017/02/09}
	\\
	\\
	Si procede con le seguenti attività:
	\begin{itemize}
		\item \normediprogetto: il documento verrà modificato e corretto. La versione raggiunta in quel momento sarà la terza;
		\item \analisideirequisiti: il documento verrà modificato e corretto. La versione raggiunta in quel momento sarà la terza;
		\item \specificatecnica: la stesura di questo documento è l'attività principale della terza fase. La \specificatecnica\ viene scritta dal Progettista, il quale descriverà ad alto livello le scelte procedurali prese per il progetto. Saranno quindi descritti quali design pattern verranno implementati, l'architettura logica del software, i principali flussi di controllo, il tracciamento dei requisiti e i componenti hardware per effettuare i successivi test di sistema del prodotto;
		\item \pianodiqualifica: il documento verrà modificato inserendo i dettagli aggiuntivi specificati durante la \revisionedeirequisiti. La versione raggiunta in quel momento sarà la terza;
		\item \pianodiprogetto: il documento verrà modificato apportando correzioni riguardanti la divisione delle attività ed aggiungendo la parte di consuntivo che riguarda quel periodo. La versione raggiunta in quel momento sarà la terza;
		\item \glossario: il documento verrà modificato inserendo i nuovi vocaboli usati durante l'aggiornamento dei documenti sovra descritti, di cui si ritiene opportuno fornire una definizione. La versione raggiunta in quel momento sarà la terza.
	\end{itemize}
	
	\begin{figure}[!h]
		\centering
		\includegraphics[width=\textwidth]{Pianificazione/Immagini/GanttChart03.png}
		\caption{GanttChart 3}
	\end{figure}	
	
\subsection{Fase 4 - Progettazione di Dettaglio e Codifica dei Requisiti Obbligatori}
	Questa fase inizia dopo la Milestone programmata per la fase 3 e termina con la \revisionediprogettazione\ (RP).
	\\
	\\
	\textbf{Periodo: dal 2017/02/20 al 2017/03/04}
	\\
	\\
	Si procede con le seguenti attività:
	\begin{itemize}
		\item I seguenti documenti verranno modificati e corretti:
			\begin{enumerate}
				\item \normediprogetto;
				\item \analisideirequisiti;
				\item \pianodiqualifica;
				\item \pianodiprogetto;
				\item \glossario.
			\end{enumerate}
		La versione raggiunta in quel momento dai documenti elencati precedentemente sarà la quarta, mentre la \specificatecnica\ avanzerà alla seconda versione;
		\item Definizione di prodotto: viene scritto il documento \definizionediprodotto, il quale definisce la struttura interna e le relazioni fra i vari componenti che costituiscono il prodotto, relativi ai requisiti obbligatori;
		\item Codifica dei Requisiti Obbligatori: durante questa attività i \programmatori\ inizieranno lo sviluppo della parte del prodotto che concerne i requisiti obbligatori, attenendosi a ciò che sarà scritto nella \definizionediprodotto;
		\item Esecuzione dei test: verranno eseguiti sia i test d'integrazione che i test di sistema, secondo quanto previsto dal \pianodiqualifica;
		\item Manuali: si procederà alla stesura del Manuale Utente, documento che fornirà un aiuto consistente agli utilizzatori finali del prodotto. Parallelamente verrà redatto anche il Manuale Sviluppatore.
	\end{itemize}
	
	\newpage
	\begin{figure}[!h]
		\centering
		\includegraphics[width=\textwidth]{Pianificazione/Immagini/GanttChart04.png}
		\caption{GanttChart 4}
	\end{figure}	
	
	\newpage
	\subsection{Fase 5 - Progettazione di Dettaglio e Codifica dei Requisiti Desiderabili}
	Questa fase inizia dopo l'esito della \revisionediprogettazione\ e termina con la Milestone prevista per la fase 5.
	\\
	\\
	\textbf{Periodo: dal 2017/03/14 al 2017/03/30}
	\\
	\\
	Si procede con le seguenti attività:
	\begin{itemize}
		\item I seguenti documenti verranno modificati e corretti:
			\begin{enumerate}
				\item \normediprogetto;
				\item \analisideirequisiti;
				\item \pianodiqualifica;
				\item \pianodiprogetto;
				\item \glossario.
			\end{enumerate}
		La versione raggiunta in quel momento dai documenti elencati precedentemente sarà la quinta, mentre la \specificatecnica\ avanzerà alla terza versione e la \definizionediprodotto\ avanzerà alla seconda versione;
		\item Codifica dei Requisiti Desiderabili: durante questa attività i \programmatori\ inizieranno lo sviluppo della parte del prodotto che concerne i requisiti desiderabili, attenendosi a ciò che sarà scritto nella \definizionediprodotto;
		\item Esecuzione dei test: verranno eseguiti sia i test d'integrazione che i test di sistema, secondo quanto previsto dal \pianodiqualifica;
		\item Manuali: si procederà all'aggiornamento del Manuale Utente e del Manuale Sviluppatore.
	\end{itemize}
	
	\newpage
	\begin{figure}[!h]
		\centering
		\includegraphics[width=\textwidth]{Pianificazione/Immagini/GanttChart05.png}
		\caption{GanttChart 5}
	\end{figure}	
	
	\subsection{Fase 6 - Progettazione di Dettaglio e Codifica dei Requisiti Opzionali}
	Questa fase inizia dopo la Milestone prevista per la fase 5 e termina con la \revisionediqualifica.
	\\
	\\
	\textbf{Periodo: dal 2017/03/31 al 2017/04/11}
	\\
	\\
	Si procede con le seguenti attività:
	\begin{itemize}
		\item I seguenti documenti verranno modificati e corretti:
			\begin{enumerate}
				\item \normediprogetto;
				\item \analisideirequisiti;
				\item \pianodiqualifica;
				\item \pianodiprogetto;
				\item \glossario.
			\end{enumerate}
		La versione raggiunta in quel momento dai documenti elencati precedentemente sarà la sesta, mentre la \specificatecnica\ avanzerà alla quarta versione e la \definizionediprodotto\ avanzerà alla terza versione;
		\item Codifica dei Requisiti Opzionali: durante questa attività i \programmatori\ inizieranno lo sviluppo della parte del prodotto che concerne i requisiti opzionali, attenendosi a ciò che sarà scritto nella \definizionediprodotto;
		\item Esecuzione dei test: verranno eseguiti sia i test d'integrazione che i test di sistema, secondo quanto previsto dal \pianodiqualifica;
		\item Manuali: si procederà all'aggiornamento del Manuale Utente e del Manuale Sviluppatore.
	\end{itemize}
	
	\begin{figure}[!h]
		\centering
		\includegraphics[width=\textwidth]{Pianificazione/Immagini/GanttChart06.png}
		\caption{GanttChart 6}
	\end{figure}	
	
	\subsection{Fase 7 - Validazione}
	Questa fase inizia dopo la \revisionediqualifica\ e termina con la \revisionediaccettazione.
	\\
	\\
	\textbf{Periodo: dal 2017/04/19 al 2017/05/13}
	\\
	\\
	Si procede con le seguenti attività:
	\begin{itemize}
		\item I seguenti documenti verranno modificati e corretti:
			\begin{enumerate}
				\item \normediprogetto;
				\item \analisideirequisiti;
				\item \pianodiqualifica;
				\item \pianodiprogetto.
			\end{enumerate}
		La versione raggiunta in quel momento dai documenti elencati precedentemente sarà la sesta, mentre la \definizionediprodotto\ avanzerà alla quarta versione;
		\item Validazione: tramite il tracciamento si verifica di aver soddisfatto i requisiti specificati nell'\analisideirequisiti;
		\item Esecuzione dei test: verranno eseguiti i test finali di sistema, secondo quanto previsto dal \pianodiqualifica;
		\item Collaudo: viene eseguito il collaudo definitivo del sistema creato.
	\end{itemize}
	
	\begin{figure}[!h]
		\centering
		\includegraphics[width=\textwidth]{Pianificazione/Immagini/GanttChart07.png}
		\caption{GanttChart 7}
	\end{figure}	
	
\end{document}