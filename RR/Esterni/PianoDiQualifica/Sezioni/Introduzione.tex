\documentclass[../PianoDiQualifica.tex]{subfiles}

\begin{document}

\section{Introduzione}

	\subsection{Scopo del documento}
	In questo documento sono illustrate le strategie di verifica e validazione che \kpanic\ ha deciso di adottare per raggiungere gli obiettivi qualitativi prefissati, cercando di mitigare il più possibile il rischio di un progressivo deterioramento del materiale prodotto.
	
	\subsection{Scopo del prodotto}
	Lo scopo del prodotto è quello di fornire assistenza ad un \gl{ospite} in visita alla sede dell'azienda attraverso un sistema d'interazione che utilizza tecnologie di sinterizzazione vocale.
	\\Il prodotto finale dovrà offrire le seguenti funzionalità:
	\begin{itemize}
		\item Registrazione dei dati personali dell'ospite all'interno di un \gl{database} di supporto;
		\item Identificazione \gl{interlocutore} all'interno dell'azienda;
		\item Inoltro delle informazioni tramite un messaggio \gl{Slack} all'interlocutore;
		\item Accoglienza virtuale dell'ospite durante l'attesa.
	\end{itemize}

	\subsection{Glossario}
	Con lo scopo di rendere più chiara e semplice la lettura e la comprensione di questo documento, viene allegato il \glossariov, nel quale vengono raccolti termini, anche tecnici, abbreviazioni ed acronimi. Per evidenziare un termine presente in tale documento, esso verrà marcato con il \gl{pedice}, e solo alla sua prima istanza.
	
	\subsection{Riferimenti utili}
		\subsubsection{Riferimenti normativi}
		\begin{itemize}
			\item \textbf{\normediprogettov};
		\end{itemize}
	
		\subsubsection{Riferimenti Informativi}	
		\begin{itemize}
			\item \textbf{Glossario:} \glossariov;
			\item\textbf{Capitolato d'appalto C2:} \progetto\ (Accoglienza tramite Assistente Virtuale \href{http://www.math.unipd.it/~tullio/IS-1/2016/Progetto/C2.pdf}{(http://www.math.unipd.it/~tullio/IS-1/2016/Progetto/C2.pdf)};
		\end{itemize}
			
\end{document}