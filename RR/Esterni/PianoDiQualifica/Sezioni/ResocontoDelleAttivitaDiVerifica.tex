\documentclass[../PianoDiQualifica.tex]{subfiles}

\begin{document}
	\appendix
	\section{Resoconto delle attività di verifica}
	Di seguito verranno descritti e analizzati gli esiti delle attività di verifica svolte su tutti i documenti che vengono consegnati nelle varie revisioni di avanzamento del progetto.
	
		\subsection{Revisione dei requisiti}
			\subsubsection{Tracciamento dei casi d'uso e dei requisiti}
			\kpanic\ ha deciso di utilizzare l'applicativo \gl{Trender} per facilitare il tracciamento sia delle relazioni fra casi d'uso e relazioni, che delle relazioni fra requisiti e fonti.
			
			\subsubsection{Analisi statica dei documenti}
			L'analisi statica dei documenti è stata fatta mediante Walkthrough che ha portato all'individuazione di alcuni errori. Tra gli errori individuati quelli più frequenti sono stati:
			\begin{itemize}
				\item Errori nei concetti esposti;
				\item Violazioni di quanto stabilito nelle norme tipografiche;
				\item Aggettivi o verbi utilizzati in modo scorretto;
				\item Periodi troppo lunghi o complessi da capire e interpretare.
			\end{itemize}
			Durante la verifica manuale sono stati anche individuati i termini da aggiugere al \glossariov. Nello specifico sono stati individuati 161 termini.
			
			\subsubsection{Verifiche automatiche}
			Come spiegato nelle \normediprogettov\ sono stati utilizzati degli applicativi software per l'individuazione di errori ortografici e per il calcolo dell'indice Gulpease oltre al controllo dei comandi \LaTeX\ permesso da \gl{Texmaker}.\\
			Il primo ha permesso di individuare tutti gli errori che non sono stati individuati nell'analisi statica dei documenti.\\
			Come spiegato alla sezione 3.4.2 "Metriche per i documenti", è stato definito un indice Gulpease da rispettare per avere un documento accettabile. In seguito ai test effettuati sui documenti sono stati ottenuti i seguenti risultati:
			\newpage
			\begin{table}[h]
				\centering
				\begin{tabular}{l * {2}{c}}
					\toprule
						\textbf{Documento} & \textbf{Gulpease} & \textbf{Esito} \\
					\midrule
						\analisideirequisitiv & 54 & Ottimale \\
						\glossariov & 60 & Ottimale \\
						\normediprogettov & 54 & Ottimale \\
						\pianodiprogettov & 53 & Ottimale \\
						\pianodiqualificav & 53 & Ottimale \\
						\studiodifattibilitav & 57 & Ottimale \\		
					\bottomrule
				\end{tabular}
				\caption{Fase 1- Indice Gulpease dei documenti}
				\label{tab:esiti_gulpease}
			\end{table}
\end{document}