\documentclass[../PianoDiQualifica.tex]{subfiles}

\begin{document}
\section{Visione generale della strategia di gestione della qualità}
	\subsection{Obiettivi di qualità}
	Per garantire la qualità del \gl{prodotto}, il gruppo \kpanic\ si impegna a perseguire determinati obiettivi di qualità durante l'intero svolgimento del progetto. Vengono usati standard, modelli e metriche per la verifica degli obiettivi preposti. 
	\subsection{Qualità di processo}
	La qualità dei processi è un fattore fondamentale durante lo svolgimento del progetto. Per favorire quest'ultima, il gruppo \kpanic\ utilizza gli standard ISO/IEC 15504, denominato \gl{Spice}, e lo standard \gl{PDCA} (Plan-Do-Check-Act).
		\subsubsection{Standard ISO/IEC 15504}
		Questo standard ci garantisce un miglioramento continuo dei processi, e ne verifica l'adeguatezza secondo il relativo obiettivo. Il team ha scelto di seguire questo modello di riferimento per la valutazione del livello di maturità dei processi. Il livello di maturità viene valutato secondo attributi di \gl{processo}, ed è definito secondo questa scala:
		\begin{itemize}
			\item Level 0 - Incompleto: il processo non è implementato o non raggiunge i suoi obiettivi;
			\item Level 1 - Eseguito: il processo è implementato e raggiunge i suoi obiettivi.
			Misurato secondo:
				\begin{itemize}
					\item Performance: capacità di ottenere risultati identificabili.
				\end{itemize}
			\item Level 2 - Gestito: il processo agisce in base ad una pianificazione ed ogni sua azione è tracciata.
			Misurato secondo:
				\begin{itemize}
					\item Gestione delle performance: capacità di elaborare un prodotto coerente con gli obiettivi attesi;
					\item Gestione delle performance: capacità di elaborare un prodotto documentato, controllato e verificato.
				\end{itemize}
			\item Level 3 - Stabilito: il processo agisce in base a linee guida uniformi nell'intera organizzazione.
			Misurato secondo:
				\begin{itemize}
					\item Definizione: capacità di elaborare un prodotto seguendo gli standard preposti;
					\item Risorse: capacità di sfruttare le risorse a disposizione così da venir attuato al meglio.
				\end{itemize}
			\item Level 4 - Predicibile: il processo agisce entro certi limiti.
			Misurato secondo:
				\begin{itemize}
					\item Misurazioni: capacità di sfruttare le misure ricavate durante l'esecuzione così da raggiungere i propri obiettivi;
					\item Controllo: capacità di sfruttare le misure ricavate durante l'esecuzione così da migliorarsi e correggersi, se necessario.
				\end{itemize}
			\item Level 5 - Ottimizzato: il processo viene misurato e quindi ottimizzato.
			Misurato secondo:
				\begin{itemize}
					\item Cambiamenti: capacità di supportare cambiamenti strutturali e di esecuzione;
					\item Miglioramento continuo: capacità di sfruttare i cambiamenti strutturali e di esecuzione così da migliorarsi continuamente nel raggiungimento dei propri obiettivi.
				\end{itemize}
			\end{itemize}
			Ogni attributo di un processo viene valutato in una scala metrica di quattro unità:
			\begin{itemize}
				\item N - Non posseduto [0 - 15\%]
				\item P - Parzialmente posseduto ]15\% - 50\%]
				\item L - Largamente posseduto ]50\% - 85\%]
				\item F - Completamente posseduto ]85\% - 100\%]
			\end{itemize}		
			\subsubsection{PDCA}
			Kern3lP4nic attua una strategia di miglioramento continuo della qualità dei processi di sviluppo utilizzando il modello PDCA, noto anche come Ciclo di Deming. In questo modo il team cerca di ottimizzare l'uso delle risorse durante l'intero ciclo di vita del prodotto puntando ad un risultato di qualità. Questo modello si suddivide in quattro iterazioni, e assicura un miglioramento progressivo ad ogni ciclo. Nello specifico: 
			\begin{itemize}
				\item Plan: vengono stabiliti obiettivi e processi necessari per raggiungere i risultati aspettati;
				\item Do: implementazione del punto precedente ed attuazione dei processi, il tutto finalizzato alla creazione del prodotto;
				\item Check: vengono comparati i risultati ottenuti con quelli attesi, e raccolti in grafici e tabelle per uno studio approfondito del traguardo raggiunto. Se si sono raggiunti gli obiettivi preposti si può passare alla fase successiva, altrimenti è necessario ripetere il ciclo PDCA tenendo conto delle cause che hanno contribuito al fallimento;
				\item Act: vengono attuate azioni di aggiustamento e correzione. La soluzione individuata diventa la nuova baseline, e si può quindi ripetere l'intero ciclo.
			\end{itemize}
		\subsection{Qualità di prodotto}
		Oltre che nei processi, \kpanic\ si prefigge di mantenere una buona qualità del prodotto cercando di seguire al meglio lo standard di qualità ISO/IEC 9126. Con prodotto, il team intende l'insieme di documenti e del software realizzato.
			\subsubsection{Qualità dei documenti}
			Gli obiettivi di qualità riguardanti i documenti che il gruppo \kpanic\ desidera raggiungere nell'arco dell'intero progetto sono i seguenti:
			\begin{itemize}
				\item I documenti devono essere comprensibili, e per far ciò il team si appoggia all'utilizzo dell'indice \gl{Gulpease};
				\item I documenti devono essere corretti a livello ortografico e semantico;
				\item I documenti devono essere attinenti all'argomento di interesse.
			\end{itemize}
			\subsubsection{Qualità del software}
			Assimilato lo standard ISO/IEC 9126, gli obiettivi di qualità riguardanti il software che il gruppo Kern3lP4nic desidera raggiungere nell'arco dell'intero progetto sono i seguenti:
			\begin{itemize}
				\item Funzionalità: il prodotto funzionerà sulla base dei requisiti indicati nel documento \analisideirequisitiv;
				\item Affidabilità: il prodotto dovrà essere robusto in presenza di eventuali situazioni di difficoltà;
				\item Usabilità: il prodotto dovrà risultare semplice nell'utilizzo per gli utenti a cui è destinato;
				\item Efficienza: il prodotto dovrà essere performante e rispondere alle richieste dell'utente utilizzando il minor numero di risorse e tempo;
				\item Manutenibilità: il prodotto dovrà essere verificato, stabile e testato ad ogni estensione o modifica;
				\item Portabilità: il prodotto si adatterà con facilità ad ogni trasferimento in diversi sistemi.
			\end{itemize}
	\subsection{Scadenze temporali}
	Le scadenze che il gruppo Kern3lP4nic ha deciso di rispettare sono le seguenti:
	\begin{itemize}
		\item Revisione dei requisiti: 24/01/2017;
		\item Revisione di progettazione: 13/03/2017 presentandosi con Revisione di Progettazione Max;
		\item Revisione di qualifica: 18/04/2017;
		\item Revisione di accettazione: 15/05/2017.
	\end{itemize}
\end{document}
