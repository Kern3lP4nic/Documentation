\documentclass[../NormeDiProgetto.tex]{subfiles}
\begin{document}

\section{Introduzione}

    \subsection{Scopo del documento}
    Questo documento definisce le norme, gli strumenti e le \gl{procedure} che tutti i membri del gruppo devono adottare per la realizzazione del progetto. Tutti i membri sono tenuti a leggere il documento e ad applicare le regole qui definite, per garantire la conformità del materiale prodotto e di ridurre il numero di errori o malfunzionamenti. Qualora siano necessarie modifiche a questo documento è richiesto l'immediato avviso agli altri membri del gruppo mediante lo strumento di comunicazione interna \gl{Slack}.

    \subsection{Scopo del prodotto}
	Lo scopo del prodotto è quello di fornire assistenza ad un \gl{ospite} in visita alla sede dell'azienda attraverso un sistema d'interazione che utilizza tecnologie di sinterizzazione vocale.
	\\Il prodotto finale dovrà offrire le seguenti funzionalità:
	\begin{itemize}
		\item Registrazione dei dati personali dell'ospite all'interno di un \gl{database} di supporto;
		\item Identificazione \gl{interlocutore} all'interno dell'azienda;
		\item Inoltro delle informazioni tramite un messaggio \gl{Slack} all'interlocutore;
		\item Accoglienza virtuale dell'ospite durante l'attesa.
	\end{itemize}

    \subsection{Glossario}
	Con lo scopo di rendere più chiara e semplice la lettura e la comprensione di questo documento, viene allegato il \glossariov, nel quale vengono raccolti termini, anche tecnici, abbreviazioni ed acronimi. Per evidenziare un termine presente in tale documento, esso verrà marcato con il \gl{pedice}, e solo alla sua prima istanza.

    \subsection{Riferimenti}
    \subsubsection{Riferimenti normativi}
    \begin{itemize}
    \item Normativa ISO 12207 per \gl{ciclo di vita del software} (\url{https://en.wikipedia.org/wiki/ISO/IEC_12207}(2017-01-29));
    \item Swebok v3 (\url{http://www.computer.org/web/swebok/v3}(2017-01-29)).
    \end{itemize}

    \subsubsection{Riferimenti informativi}
    \begin{itemize}
        \item \textbf{Glossario:} \glossariov;
        \item\textbf{Capitolato d'appalto C2:} AtAVi (Accoglienza tramite Assistente Virtuale) (\url{http://www.math.unipd.it/~tullio/IS-1/2016/Progetto/C2.pdf}(2017-01-29));
        \item Diapositive del Modulo A del corso (\url{http://www.math.unipd.it/~tullio/IS-1/2016/}(2017-01-29));
        \item Diapositive del Modulo B del corso (\url{http://www.math.unipd.it/~rcardin/sweb.html}(2017-01-29));
        \item Gerard J. Holzmann, \textit{The Power of 10: Rules for Developing Safety-Critical Code}(\url{https://spinroot.com/gerard/pdf/Power_of_Ten.pdf}(2017-01-29)).
    \end{itemize}
\end{document}
