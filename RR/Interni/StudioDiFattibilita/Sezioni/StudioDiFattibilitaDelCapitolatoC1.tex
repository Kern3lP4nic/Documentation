\documentclass[../StudioDiFattibilita.tex]{subfiles}

\begin{document}
\section{Studio di fattibilità del capitolato C1}
	\subsection{Descrizione}
	Il \gl{capitolato} \gl{APIM}, acronimo di An API Market Platform, presentato da italianaSoftware, consiste nel creare un'applicazione \gl{Web} che gestisca un market di \gl{API} per \gl{microservizi}, nello specifico permette di registrare API a microservizi, consultare le loro \gl{interfacce} e la loro documentazione, monitorare il loro utilizzo ed associare diverse \gl{API key} d'uso per ogni API e limitarne quindi l'accesso solo in caso di validità della stessa. In base al numero di gruppi che svolgono questo capitolato, vengono assegnati obiettivi extra, quali definizione di policy di vendita delle API, con relative gestione di una moneta virtuale per la loro compravendita, definizione di un \gl{SLA} per ciascuna API e relativo monitoraggio ed infine gestire un piccolo social network con il profilo degli utilizzatori di queste API, con relative valutazioni e/o recensioni. Il \gl{proponente} offre anche una formazione minima sull'utilizzo di \gl{Jolie}, tecnologia che sta sviluppando internamente, utile alla definizione delle interfacce delle API.

	\subsection{Studio di dominio}
	  	\subsubsection{Dominio applicativo}
		Il progetto è dedicato a tutte quelle aziende che vogliono vendere una propria API in un \gl{e-commerce}, monitorarandone l'utilizzo ai fini commerciali.

	 	\subsubsection{Dominio tecnologico}
		Le tecnologie consigliate dal proponente per lo sviluppo del progetto sono:
			\begin{itemize}
				\item \textbf{\gl{HTML}, \gl{CSS3} e \gl{JavaScript}:} sviluppo dell'interfaccia \gl{Web};
				\item \textbf{\gl{Bootstrap}:} \gl{framework} CSS;
				\item \textbf{\gl{NoSQL} o \gl{SQL} \gl{Database}:} da utilizzare a livello di \gl{base di dati};
				\item \textbf{\gl{Swagger}:} tool online utile per il \gl{design}, la creazione, la documentazione e lo sviluppo di API;
				\item \textbf{\gl{Jolie}:} linguaggio di programmazione utile alla definizione delle interfacce delle API.
			\end{itemize}
			Per ogni altra esigenza, il proponente lascia libera scelta.

		\subsection{Potenziali criticità}
		Il team ha riscontrato alcuni punti potenzialmente critici, quali:
			\begin{itemize}
				\item non tutti i membri del gruppo hanno piena conoscenza delle tecnologie da utilizzare, specialmente Bootstrap, Swagger, Jolie;
        		\item introdurre un Service Level Agreement per i servizi  necessita di massima attenzione e una conoscenza di base di alcune policy di sicurezza.
			\end{itemize}

	\subsection{Valutazione finale}
	\kpanic\ ha tenuto questo capitolato in gioco fino all'ultimo, in concorrenza con il C2. Alla fine C1 è stato scartato perché, dopo un'attenta analisi sulle tecnologie utilizzate, abbiamo deciso che Jolie, ancora in fase di beta e non ancora ampiamente diffusa nell'ambito delle tecnologie dedicate allo sviluppo di API, non suscita interesse alla maggior parte del gruppo e quindi non ci porterebbe a sviluppare questo progetto al meglio delle nostre capacità. Inoltre l'obiettivo del capitolato, un market di API, è qualcosa di già presente nel mercato globale e quindi secondo il team non ha molta prospettiva di crescita una volta avviato.

\end{document}
