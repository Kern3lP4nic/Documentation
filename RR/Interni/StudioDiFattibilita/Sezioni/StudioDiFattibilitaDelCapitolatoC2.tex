\documentclass[../StudioDiFattibilita.tex]{subfiles}

\begin{document}
\section{Studio di fattibilità del capitolato C2}
	\subsection{Descrizione}
	Il capitolato \atavi\, acronimo di \acratavi\, presentato da \prop\, consiste nel creare un'\gl{applicazione web} che permetta agli ospiti di un'azienda di interrogare un \gl{assistente virtuale} dedicato all'accoglienza. Tutte le risposte del visitatore verranno inviate in un canale \gl{Slack} dedicato, in modo da mantenere un log delle conversazioni. La WebApp dovrà fornire alcune funzionalità obbligatorie, tra le quali la richiesta di alcuni dati personali dell'ospite e di sue eventuali necessità (moduli, informazioni, ecc), che verranno inoltrate all'interessato tramite Slack. Il \gl{software} dovrà utilizzare la lingua inglese per adattarsi agli assistenti virtuali già disponibili. Con questo progetto \prop\ cerca di semplificare le \gl{procedure} intraprese solitamente da personale dedicato nell'accogliere di nuove persone, \gl{automatizzando} il tutto e rendendolo più veloce ed innovativo.

	\subsection{Studio di dominio}
	  	\subsubsection{Dominio applicativo}
		Il progetto è particolarmente dedicato a chi utilizza sistemi di comunicazione aziendale come Slack, ma si rivolge anche a chi vuole trovare nuovi affascinanti modi di accogliere ospiti nella propria azienda. Il progetto risulta interessante soprattutto per le innovative tecnologie utilizzate, visto il recente interesse globale nel campo degli assistenti virtuali.

	 	\subsubsection{Dominio tecnologico}
		Le tecnologie consigliate dal proponente per lo sviluppo del progetto sono:
		\begin{itemize}
			\item \textbf{\gl{HTML5}, CSS3 e JavaScript:} sviluppo dell'interfaccia Web;
			\item \textbf{Bootstrap 3:} framework CSS;
			\item \textbf{\gl{NodeJS} o \gl{Swift}:} linguaggio di programmazione per sviluppare l'applicativo, in base all'assistente virtuale utilizzato, relativamente \gl{Amazon Alexa} e \gl{Siri};
			\item \textbf{\gl{SiriKit} o \gl{Alexa Skills Kit}:} assistenti virtuali da utilizzare;
			\item \textbf{\gl{Express}:} eventuale framework NodeJS;
			\item \textbf{\gl{AWS}:} servizi per l'interazione con le API dell'assistente virtuale;
			\item \textbf{\gl{MongoDB} o \gl{DynamoDB}:} NoSQL Database;
			\item \textbf{Slack API:} API da utilizzare per la comunicazione con Slack;
			\item \textbf{Swagger:} tool online utile per il design, la creazione, la documentazione e lo sviluppo di API.
		\end{itemize}

		\subsection{Potenziali criticità}
		Il team ha riscontrato alcuni punti potenzialmente critici, tra i quali:
		\begin{itemize}
			\item non tutti i membri del gruppo hanno piena conoscenza delle tecnologie da utilizzare;
			\item l'analisi del comportamento degli utenti richiede uno studio accurato dei possibili scenari che possono venirsi a creare;
			\item il punto precedente non deve trascurare le necessità dello sviluppatore, quali ad esempio risposte brevi e facilmente comprensibili.
		\end{itemize}
		Più dettagli e ulteriori punti critici possono essere trovati nel documento \pianodiprogettov.

	\subsection{Valutazione finale}
	Il team ha scelto questo capitolato perché le tecnologie usate e il dominio applicativo risultano molto interessanti. Inoltre \kpanic\ considera un aspetto positivo l'acquisizione di conoscenze riguardanti gli assistenti virtuali, ambito tecnologico in forte crescita e con molto potenziale a livello di mercato. Il prodotto ha capacità di rivolgersi ad un ampio numero di aziende, con alta probabilità di avere successo e di offrire un importante contributo perché attualmente non presente sul mercato.
\end{document}
