\documentclass[../StudioDiFattibilita.tex]{subfiles}

\begin{document}
\section{Studio di fattibilità del capitolato C5}
	\subsection{Descrizione}
	Il capitolato \gl{Monolith} presentato da Red Babel richiede la realizzazione di un framework impiegabile sulla \gl{piattaforma} \gl{Rocket.Chat}. L'applicativo desiderato deve permettere la creazione di particolari messaggi chiamati bubble che permettano la condivisione di contenuti (link, \gl{GIF} o video) all'interno di una \gl{chat}. Queste bubble devono avere la possibilità di aggiornare automaticamente le informazioni al proprio interno, come ad esempio la variazione dell'orario di un volo, delle condizioni atmosferiche o il prezzo di un prodotto. Inoltre, anche il ricevente del messaggio può modificarne il contenuto, marcando ad esempio la sua risposta all'interno della bubble ricevuta, spuntare un elemento da un elenco o caricare un file.

	\subsection{Studio di dominio}
	  	\subsubsection{Dominio applicativo}
		Essendo \gl{Rocket.Chat} rivolto principalmente ad un impiego in ambiente lavorativo, questo progetto cerca di facilitare la comunicazione in determinate situazioni dove si necessita ad esempio di avere risposte rapide e precise, eseguire un sondaggio, oppure discutere tra colleghi o con clienti di determinati argomenti in maniera semplice ed intuitiva.

	 	\subsubsection{Dominio tecnologico}
		Le tecnologie consigliate dal proponente per lo sviluppo del progetto sono:
            	\begin{itemize}
            		\item\textbf{\gl{NodeJS}:} sviluppo del framework;
            		\item\textbf{\gl{MeteorJS}:} estensione di NodeJS ;
            		\item\textbf{\gl{AngularJS}:} framework JavaScript per lo sviluppo intero di Applicazioni Web cross-platform\gl;
            		\item\textbf{\gl{React}:} libreria JavaScript per lo sviluppo del \gl{front-end};
            		\item\textbf{\gl{SCSS}:} estensione di CSS3;
            		\item\textbf{\gl{Heroku}:} piattaforma di appoggio su cui implementare il progetto;
            		\item\textbf{\gl{Rocket.Chat} API:} API da utilizzare per l'integrazione con Rocket.Chat.
				\end{itemize}

		\subsection{Potenziali criticità}
		Il gruppo ha evidenziato alcune potenziali difficoltà critiche, tra le quali:
        	\begin{itemize}
        		\item non tutti i membri del team hanno piena conoscenza delle tecnologie da utilizzare, come \gl{NodeJS}, \gl{AngularJS} e \gl{React};
        		\item le possibili interazioni con le bubble vanno limitate in base al contenuto;
        		\item gli aggiornamenti automatici richiedono un'attenta gestione delle risorse;
        	\end{itemize}

	\subsection{Valutazione finale}
	Malgrado questo capitolato risulti essere interessante per le tecnologie impiegate che oggigiorno trovano un grande utilizzo, \kpanic\ ha scelto di scartarlo perché l'impiego del prodotto si limita solamente a \gl{Rocket.Chat} che, rispetto a concorrenti, risulta meno diffuso avendo quindi un bacino di utenti nettamente inferiore e suscitando nel gruppo poco interesse. Di conseguenza anche ciò che viene richiesto di realizzare avrà un impiego numericamente inferiore e non è ciò a cui punta il team.

\end{document}
