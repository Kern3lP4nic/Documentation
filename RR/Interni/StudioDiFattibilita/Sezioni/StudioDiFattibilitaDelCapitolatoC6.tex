\documentclass[../StudioDiFattibilita.tex]{subfiles}

\begin{document}
\section{Studio di fattibilità del capitolato C6}
	\subsection{Descrizione}
	\gl{SWEDesigner} è il progetto proposto da Zucchetti richiede la realizzazione di un applicativo Web per la realizzazione di diagrammi \gl{UML} con la relativa generazione di codice \gl{Java} e JavaScript. L'\gl{editor} richiesto deve prevedere la realizzazione di \gl{diagrammi delle classi} e di \gl{diagrammi di attività}. Successivamente si dovrà ricavare lo scheletro delle classi, e il corpo dei metodi. Nei casi in cui non è possibile la generazione del codice sorgente il sistema deve essere in grado di segnarlo. Inoltre è a discrezione dei gruppi che decidono di realizzare questo capitolato se implementare anche la gestione di \gl{flowchart} o di \gl{diagramma di sequenza}.
	
	\subsection{Studio di dominio}	
	  	\subsubsection{Dominio applicativo}
		Questo progetto ha come utilizzatori tutti coloro che vogliono velocizzare la scrittura di codice Java o JavaScript tramite l'utilizzo di digrammi UML o che ne fanno uso per un'attività. Il capitolato risulta molto interessante visto l'obiettivo finale richiesto; impiegabile anche in futuri progetti, nei linguaggi indicati, previo realizzazione di un diagramma UML.
		
	 	\subsubsection{Dominio tecnologico}
		Per lo sviluppo del progetto sono richieste le seguenti tecnologie:
       		\begin{itemize}
           		\item\textbf{HTML5, CSS3 e JavaScript:} sviluppo dell'interfaccia Web ;
            	\item\textbf{Java o JavaScript:} parte \gl{server-side} scritta relativamente con \gl{Tomcat} e NodeJS;
      		\end{itemize}
		
		\subsection{Potenziali criticità}
		Il team ha riscontrato alcune criticità per la realizzazione di questo progetto, tra le quali:
        	\begin{itemize}
        		\item non tutti i membri del team hanno piena conoscenza delle tecnologie da utilizzare, come Java;
        		\item il riconoscimento dei diagrammi UML da parte di strumenti automatici può risultare difficile da comprendere e causare risultato indesiderati;
        	\end{itemize}

	\subsection{Valutazione finale}  
	Il gruppo ha deciso di non interessarsi alla realizzazione di questo progetto perché, oltre ad essere poco interessante dal punto dell'impiego dello stesso limitandolo ai soli utilizzatori di diagrammi UML, risulta essere tra i più complessi capitolati proposti.
	
\end{document}