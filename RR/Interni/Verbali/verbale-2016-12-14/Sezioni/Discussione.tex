\documentclass[../verbale-2016-12-14.tex]{subfiles}

	\begin{document}
	\section{Discussione dell'ordine del giorno}
	Tutti i punti hanno avuto un'analisi attenta delle caratteristiche positive e negative che comportavano ogni scelta, ed una approvazione data dall’assenso dei singoli membri del gruppo. Le norme approvato sono le seguenti:
			\begin{itemize}
				\item I nomi di file e cartelle deve essere nella forma PascalCase;
				\item La versione, che sarà presente solo all'interno del documento stesso e non nel nome, sarà scritta nella forma vR.V.X con:
					\begin{itemize}
						\item R= +1 se incremento versione approvata dal responsabile
						\item V=+1 se incremento versione  approvata dal verificatore
						\item X= +1 se incremento versione  approvata dagli altri ruoli che redigono il documento
					\end{itemize}
		\item Per la creazione dei diagrammi dei casi d’uso viene scelto all’unanimità Astah professional con licenza studenti;
		\item Viene approvato all’unanimità l’uso dello strumento Trender;
		\item Gantt project viene approvato all’unanimità dal gruppo come strumento per la creazione dei diagrammi di Gantt;
		\item Approvazione dello strumento Trender per la gestione degli use case e dei requisiti;
		\item Il gruppo ha stilato una lista di domande per l’incontro con l’azienda. Verrà inviata una e-mail al referente della \prop con il documento delle domande come da loro richiesto;
		\item I membri che parteciperanno alla conference call con l’azienda: Antonino Macrì, Francesco Fasolato, Giacomo Zecchin.
	\end{itemize}
\end{document}